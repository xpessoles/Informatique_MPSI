% Déclaration des titres
% -------------------------------------


\graphicspath{{../../style/png/}{images/}{../../exos//}}
\lstinputpath{{../../exos//}}

\def\discipline{Informatique}
\def\xxtete{Informatique}

\def\classe{\textsf{MPSI}}
\def\xxnumpartie{3}
\def\xxpartie{Simulation numérique}
\def\xxdate{25 Mars 2020}

\def\xxchapitre{04}
\def\xxnumchapitre{04}
\def\xxnomchapitre{Systèmes linéaires}
\def\xxnumactivite{04}

\def\xxposongletx{2}
\def\xxposonglettext{1.45}
\def\xxposonglety{19}%16

\def\xxonglet{\textsf{Cycle 01}}
\def\xxauteur{\textsl{Émilien Durif \\ Xavier Pessoles}}


\def\xxpied{%
Cycle \xxnumpartie -- \xxpartie\\
Chapitre \xxnumchapitre -- \xxactivite -\xxnumactivite -- \xxnomchapitre%
}

\setcounter{secnumdepth}{5}
\chapterimage{Fond_SIMU}
\def\xxfigures{}

\def\xxcompetences{%
\textsl{%
\textbf{Savoirs et compétences :}\\
\begin{itemize}[label=\ding{112},font=\color{ocre}]
\item SN.C1 : Réaliser un programme complet structuré
\item SN.C2 : Étudier l'effet d'une variation des paramètres sur le temps de calcul, sur la précision des résultats, sur la forme des solutions pour des programmes d'ingénierie numérique choisis, tout en contextualisant l'observation du temps de calcul par rapport à la complexité algorithmique de ces programmes
\item SN.C3 : Utiliser les bibliothèques de calcul standard
\item SN.C4 : Utiliser les bibliothèques standard pour afficher les résultats sous forme graphique
\item SN.C5 : Tenir compte des aspects pratiques comme l'impact des erreurs d'arrondi sur les résultats, le temps de calcul ou le stockage en mémoire.
\item SN.S1 : Bibliothèques logicielles
\item SN.S4 : Problème discret multidimensionnel linéaire. Méthode de Gauss avec recherche partielle du pivot.
\end{itemize}
}}


%Infos sur les supports
\def\xxtitreexo{Systèmes linéaires}
\def\xxsourceexo{\hspace{.2cm} \footnotesize{
}}
%\def\xxtitreexo{Titre EXO}
%\def\xxsourceexo{\hspace{.2cm} \footnotesize{Source EXO}}


%---------------------------------------------------------------------------


