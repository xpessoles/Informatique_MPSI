%%%% Paramétrage du cours %%%%
\def\xxactivite{\ifprof TP -- Corrigé  \else  Interro B \fi}
\def\xxauteur{\textsl{Émilien Durif -- Xavier Pessoles}}

%\fichefalse
%\proftrue
%\tdfalse
%\courstrue

% Déclaration des titres
% -------------------------------------

\def\discipline{Informatique}
\def\xxtete{Informatique}

\def\classe{\textsf{MPSI}}
\def\xxnumpartie{3}
\def\xxpartie{Simulation numérique}
\def\xxdate{11 Mars 2020}

\def\xxnumchapitre{3}
\def\xxnomchapitre{Problèmes stationnaires}

\def\xxposongletx{2}
\def\xxposonglettext{1.45}
\def\xxposonglety{19}%16

\def\xxonglet{\textsf{Cycle 01}}
\def\xxauteur{\textsl{Émilien Durif \\ Xavier Pessoles}}


\def\xxpied{%
\xxnumpartie -- \xxpartie\\
\xxnumchapitre -- \xxactivite -- \xxnomchapitre%
}

\setcounter{secnumdepth}{5}
\chapterimage{Fond_SIMU}
\def\xxfigures{}

\def\xxcompetences{%
\textsl{%
\textbf{Savoirs et compétences :}\\
\begin{itemize}[label=\ding{112},font=\color{ocre}]
\item SN.C1; SN.C2; SN.C3; SN.C4; SN.C5; SN.S1; SN.S2
\item ....
\end{itemize}
}}


%---------------------------------------------------------------------------





\def\xxfigures{
%\includegraphics[width=.6\linewidth]{fig_00}
}%figues de la page de garde


\iflivret
\input{../../../style/pagegarde_info}
\else
\input{../../../style/pagegarde_info}
\fi
\setlength{\columnseprule}{.1pt}

\pagestyle{fancy}
\thispagestyle{plain}

\ifprof
\vspace{4.5cm}
\else
\vspace{4.5cm}
\fi

\def\columnseprulecolor{\color{ocre}}
\setlength{\columnseprule}{0.4pt} 

%%%%%%%%%%%%%%%%%%%%%%%

\setcounter{exo}{0}
\vspace{2cm}
\begin{multicols}{2}
%\subparagraph{}\textit{Donner l'algorithme permettant de calculer l'intégrale d'une fonction sur un intervalle $[a,b]$ en utilisant la méthode des rectangles à gauche. Pour cela on implémentera la fonction \texttt{def IntRectGauche(f:function, a:float, b:float, n:int) -> float : }.}

\subparagraph{}\textit{Donner l'algorithme permettant de calculer l'intégrale d'une fonction sur un intervalle $[a,b]$ en utilisant la méthode des rectangles à droite. Pour cela on implémentera la fonction \texttt{def IntRectDroite(f:function, a:float, b:float, n:int) -> float : }.}


\subparagraph{}\textit{Donner l'algorithme permettant de calculer l'intégrale d'une fonction sur un intervalle $[a,b]$ en utilisant la méthode des trapèzes. Pour cela on implémentera la fonction \texttt{def IntRectTrapeze(f:function, a:float, b:float, n:int) -> float : }.}

%\subparagraph{}\textit{Donner les instructions (et les fonctions) permettant de déterminer $\int\limits_0^1 x^2 \dd x$. en utilisant chacune des méthodes.}
%
%
\subparagraph{}\textit{Donner les instructions (et les fonctions) permettant de déterminer $\int\limits_0^1 \sqrt{x} \dd x$. en utilisant chacune des méthodes.}


%\subparagraph{}
%\textit{Résoudre numériquement l'équation différentielle $\dot{y}+\alpha y= 0$ avec $y(0)=1$ et $\alpha=2$. On définira : 
%\begin{itemize}
%\item la fonction $F$ définie dans le problème de Cauchy;
%\item la fonction \texttt{def euler(F, a, b, y0, h) -> list,list:}.
%\end{itemize}h
%Enfin, on tracera la solution de l'équation différentielle.}



\subparagraph{}
\textit{Résoudre numériquement l'équation différentielle $\beta \dot{\theta}+ \theta= 0$ avec $y(0)=1$ et $\beta=0,1$. On définira : 
\begin{itemize}
\item la fonction $F$ définie dans le problème de Cauchy;
\item la fonction \texttt{def euler(F, a, b, y0, h) -> list,list:}.
\end{itemize}
Enfin, on tracera la solution de l'équation différentielle.}


%\subparagraph{}
%\textit{Vectoriser l'équation différentielle suivante en la mettant sous la forme  $f:\begin{pmatrix} a \\ b \end{pmatrix},t \mapsto \begin{pmatrix} g(a,b) \\ h(a,b) \end{pmatrix} $:
%$a \ddot{u}(t) + \dot{u}(t)  + {u}(t) = 1$. On explicitera les fonctions $g$ et $h$.}

\subparagraph{}
\textit{Vectoriser l'équation différentielle suivante en la mettant sous la forme  $f:\begin{pmatrix} a \\ b \end{pmatrix},t \mapsto \begin{pmatrix} g(a,b) \\ h(a,b) \end{pmatrix} $:
$m \ddot{y}(t) + c\dot{y}(t)  + k{y}(t) = 0$. On explicitera les fonctions $g$ et $h$.}
%


%%$\ddot{\theta}(t) + k_1\sin \left(\theta(t)\right)  + k_2 \dot{\theta}(t) = 0$
%%avec $\theta(0)=1$ et $\dot{\theta}(0)=2$. 
%%
%%$a \ddot{u}(t) + b\dot{u}(t)  + {u}(t) = e(t)$
%%avec $\theta(0)=1$ et $\dot{\theta}(0)=2$. 
%%
%%Reformuler l'équation différentielle sous la forme $f:\begin{pmatrix} a \\ b \end{pmatrix},t \mapsto \begin{pmatrix} g(t) \\ h(t) \end{pmatrix} $

\end{multicols}




