\begin{itemize}
\item Il est nécessaire de comprendre le sujet dans sa globalité afin de comprendre la successions des questions et des fonctions. 
\item Il faut bien penser à réutiliser les fonctions précédentes.
\item Il faut penser au rebouclage de l'alphabet en utilisant la division Euclidienne pour éviter les dépassements. 
\item Attention aux confusions entre les \texttt{print} et les \texttt{return}. Le \texttt{print} ne permet pas de réutiliser le résultat affiché.
\item Il y a confusion entre la gestion des chaînes de caractères et les listes. Les chaînes de caractères ne sont pas mutables. On ne peut pas changer un caractère avec l'instruction \texttt{chaine[i]='a'}. 
\item Attention à la création de listes vides :  \texttt{t=[0]*10} puis \texttt{tt=10*[t]} ne crée pas une liste de listes indépendantes...
\item Ne pas utiliser les fonctions \texttt{sum}.
\end{itemize}
