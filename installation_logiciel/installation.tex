\documentclass[francais,a4paper,DIV=16]{scrartcl}


\textbf{Consignes}

\input{../../../exos/consignes/consignes_TP4}


\activite{Bataille navale}

\question{} Dans le jeu de la bataille navale, on représente chaque case par un couple d'entiers entre 0 et~9.

Un navire a ses extrémités sur les cases \pyv{a} et \pyv{b}. Un joueur tire sur la case \pyv{x}. 

\'Ecrire une fonction \pyv{touche(a,b,x)} qui renvoie un booléen indiquant si le navire est touché ou non.


\activite{Simulation d'un prêt immobilier}

Un banquier vous  propose un prêt de $400\,000$ euros  sur $40$ ans «à
$3\%$ par  an» ---  ce qui, dans  le langage commercial  des banquiers,
veut  dire $0,25\%$  par mois  ---  avec des  mensualités de  $1431,93$
euros.  Autrement  dit,  vous   contractez  une  dette  de  $400\,000$
euros. Chaque mois, cette dette  augmente de $0,25\%$ puis est diminuée
du  montant  de  votre  mensualité.  À  la  fin  des  $40  \times  12$
mensualités, il  ne vous  reste plus qu'à  vous acquitter  d'une toute
petite dette, que vous rembourserez aussitôt.


\question{} Écrire  une fonction  \pyv{reste_a_payer(p, t,  m, d)}
renvoyant  le montant  de cette  somme à  rembourser  immédiatement
après le paiement de la dernière
mensualité, où  $p$ est le  montant total du  prêt en euros (dans l'exemple, $400\,000$), $t$ son  taux mensuel (dans l'exemple, 
$0,25 \times 10^{-2}$), $m$ le montant d'une mensualité en euros (dans l'exemple, $1431,93$) et $d$ la
durée en années (dans l'exemple, $40$).

\emph{Indice : dans le cas donné dans cet énoncé, vous devez trouver un montant
restant d'un peu moins de $7,12$ euros.}


\question{} Écrire une fonction \pyv{somme_totale_payee(p, t, m,
  d)} renvoyant la somme totale (mensualités plus le dernier
paiement) que vous aurez payé au banquier.


\question{}Écrire une fonction \pyv{cout_total(p, t, m, d)} renvoyant
  le coût total  du crédit, c'est-à-dire le total de  ce que vous avez
  payé moins le montant du prêt.
  
Un banquier vous propose de vous prêter $p$ euros, à un taux de $12t\%$ par an ---  ce qui, dans  le langage commercial  des banquiers,
veut  dire $t\%$  par mois  --- avec des  mensualités de  $m$ euros. Autrement  dit,  vous   contractez  une  dette  de  $p$
euros. Chaque mois, cette dette  augmente de $t\%$ puis est diminuée du  montant  de  votre  mensualité. Lorsque votre dette, augmentée du taux, est inférieure à la mensualité, il suffit de régler le solde en une fois.

\question{} \'Ecrire une fonction \pyv{duree_mensualite(p,t,m)} renvoyant le nombre de mensualités nécessaires au remboursement total du prêt.

\question{} Attention : que se passe-t-il si la mensualité est trop petite ? 

\emph{Indice : dans le cas où le prêt est $\texttt{p}=4\times10^5$, le taux est $\texttt{t}=0,25\times10^{-2}$ et la mensualité est $\texttt{m}=1431,93$, on trouvera une durée de remboursement de $480$ mois.}

\question{} \'Ecrire une fonction \pyv{tracer_amortissement(p,t,m)} permettant de tracer pour les tous mois jusqu'à ce que le prêt soit remboursé le capital restant du ainsi que la somme cumulé des intérêts versé à la banque.



\activite{Suites}

 On pose $u_{0} = 1$ et pour tout $n\in \N$,
\begin{align*}
  u_{n+1} &= \frac{1}{2}\p{u_{n}+\frac{n+1}{u_{n}}}\\
\text{et } v_{n} &= \sum_{k=0}^{n} \frac{1}{u_{k}^{5}}
\end{align*}

\question{} Écrire une fonction \pyv{f(n)} renvoyant la valeur de
  $v_{n}$.

On peut montrer que $\p{v_{n}}_{n\in \N}$ converge.

\textbf{Attention}: on fera attention à ce que le calcul de \pyv{f}
de demande  pas trop de (re)calculs inutiles. Pour fixer les idées,
vous  pouvez considérer  que  \pyv{f(10**6)} doit être calculé  en
(largement) moins d'une minute.

\question{} Vérifier que vous pouvez calculer $v_{n}$ pour de grandes valeurs de $n$.




\activite{ (Facultative) Sommes}

\question{} Écrire une fonction \pyv{somme1(n)} et une fonction \pyv{somme2(n)} prenant en argument
un entier naturel \pyv{n} et renvoyant respectivement
\begin{gather}
  \sum_{1\leq i,j\leq n} \frac{1}{i+j^{2}}~, \\
  \sum_{1\leq i<j\leq n} \frac{1}{i+j^{2}}~.
\end{gather}

Au besoin, on introduira des fonctions auxiliaires.

\title{Fiche d'aide à l'installation des programmes}
\date{\today}
% \date{5 septembre 2017}
\begin{document}
\maketitle{}

Pour pouvoir travailler chez vous en informatique, vous aurez besoin :
\begin{itemize}
  \item d'utiliser \texttt{Python} (version 3) dans un IDE, avec plusieurs modules ; 
  \item pouvoir manipuler des fichiers textes et des tableurs ; 
  \item pouvoir manipuler une base de données \texttt{sqlite}.
\end{itemize}
Comme IDE pour \texttt{Python}, nous vous conseillons {IDLE}, mais tout IDE pour \texttt{Python} conviendra. 

Voici quelques indications pour installer les logiciels nécessaires sur les différentes plateformes. 


\section{Windows.}

Vous pouvez installer pyzo sous windows. Vous trouverez les fichiers d'installation sur le lien suivant : \url{http://www.pyzo.org/start.html}. Il faut suivre les étapes précisées sur le site pour installer : 
\begin{itemize}
\item l'environnement de développement intégré (IDE)
\item l'environnement python avec miniconda.
\end{itemize}
%
%Installer la bibliothèque de calcul scientifique \texttt{scipy} séparément de \texttt{Python} est très difficile sous Windows.
%
Si vous souhaitez installer une distribution Linux directement sur votre ordinateur, en parallèle de Windows, une option simple est d'installer une machine virtuelle (nous vous conseillons {VirtualBox}\footnote{\texttt{https://www.virtualbox.org/}}), puis d'installer une distribution Linux\footnote{\texttt{https://www.ubuntu-fr.org/telechargement/} pour une distribution grand public} dans cette machine virtuelle. 
Il suffit ensuite de suivre les instructions de la partie~\ref{installation:sec:linux}.
%
%Dans le cas contraire, vous pouvez installer Anaconda\footnote{\texttt{https://www.continuum.io/downloads}}, qui comporte l'IDE spyder et contient toutes les bibliothèques usuelles.  
%
%Vous aurez aussi besoin d'un éditeur de texte proposant des colorations syntaxiques (par exemple, Notepad++\footnote{\texttt{https://notepad-plus-plus.org/}}), ainsi que d'un tableur (par exemple, LibreOffice\footnote{\texttt{https://fr.libreoffice.org/}}).

Pour utiliser SQLite, nous vous conseillons d'installer «DB Browser for SQLite»\footnote{\texttt{https://sqlitebrowser.org/}}. Comme dans linux et MacOS il faut télécharger le logiciel et télécharger la version "Windows.exe". Il faudra ensuite exécuter ce fichier et suivre les instructions.

\section{Linux.}\label{installation:sec:linux}

Vous pouvez installer un paquet en tapant la commande suivante dans un terminal (le système vous demandra votre mot de passe). 
\begin{verbatim}
sudo apt-get install nom_du_paquet
\end{verbatim}
Alternativement, vous pouvez utiliser un gestionnaire graphique de paquets (par exemple, Synaptic via le paquet \texttt{synaptic}).

Pour pouvoir utiliser \texttt{Python} et \texttt{sqlite} correctement, installez les paquets 
\begin{itemize}
  \item \texttt{python3}
  \item \texttt{idle-python3.XX} (le plus récent disponible, avec \texttt{XX} valant \texttt{4}, \texttt{5} ou \texttt{6}).
  \item \texttt{python3-numpy}
  \item \texttt{python3-scipy}
  \item \texttt{python3-matplotlib}
  \item \texttt{sqlite3}
  \item \texttt{sqlitebrowser}
\end{itemize}

Vous aurez aussi besoin d'un éditeur de texte proposant des colorations syntaxiques (par exemple, gedit via le paquet \texttt{gedit} ou emacs via le paquet \texttt{emacs}), ainsi que d'un tableur (LibreOffice via le paquet \texttt{libreoffice-calc} ou Gnumeric via le paquet \texttt{gnumeric}).

\section{Mac OS.}

Python est déjà installé de base sous Mac OS. Cependant par défaut il s'agit de la version 2.7 alors que nous allons travailler avec la version 3.

Les fichiers d'installation de Python 3.6 se situent sur le lien URL suivant : \url{https://www.python.org/downloads/} et vous pouvez cliquer sur \texttt{"Download Python 3.6.2"}. Il vous suffira ensuite d'exécuter le fichier \texttt{python-3.6.2-macosx10.6.pkg} et de suivre les instructions.

Un dossier \texttt{Python 3.6} sera alors créé dans le dossier \texttt{Applications} et vous pourrez lancer \texttt{IDLE} qui vous donnera accès à une console et à un éditeur. 

Comme dans \texttt{Linux} (Partie \ref{installation:sec:linux}) il faut installer les paquets. Pour cela on précèdera de la manière suivante :

\begin{itemize}
\item Installer l'utilitaire d'installation \texttt{pip} : 
\begin{itemize}
\item Ouvrir un terminal (Dans Application/Utilitaires/Terminal).
\item Taper l'instruction : \texttt{sudo easy$\_$install pip} puis saisir votre mot de passe administrateur.
\end{itemize}
\item Installer les paquets les uns après les autres avec la commande :
\texttt{python3 -m pip install nom$\_$du$\_$paquet} en remplaçant \texttt{nom$\_$du$\_$paquet} par : 
\begin{itemize}
  \item \texttt{numpy}
  \item \texttt{scipy}
  \item \texttt{matplotlib}.
\end{itemize}
\item Ici, il n'est pas utile d'installer \texttt{sqlite3} car il est déjà installé par défaut.
\end{itemize}

Vous pouvez également installer pyzo qui est un environnement assez convivial pour travailler avec python (\url{http://www.pyzo.org/start.html}).


Comme dans linux il faudra installer sqlitebrowser pour taiter les bases de données. Pour cela il faut télécharger le logiciel sur \url{https://sqlitebrowser.org/} et télécharger la version "Mac.dmg". Il faudra ensuite exécuter ce fichier et suivre les instructions.





\end{document}
