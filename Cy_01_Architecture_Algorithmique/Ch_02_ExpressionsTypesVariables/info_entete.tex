% Déclaration des titres
% -------------------------------------


\graphicspath{{../../style/png/}{images/}{../../exos/python_bases/PYB-000/}}
\lstinputpath{{../../exos/python_bases/PYB-000/}}

\def\discipline{Informatique}
\def\xxtete{Informatique}

\def\classe{\textsf{MPSI}}
\def\xxnumpartie{1}
\def\xxpartie{Architecture matérielle et initiation à l'algorithmique}
\def\xxdate{11 Septembre 2019}

\def\xxchapitre{02}
\def\xxnumchapitre{02}
\def\xxnomchapitre{Expressions, types et variables en Python}
\def\xxnumactivite{02}

\def\xxposongletx{2}
\def\xxposonglettext{1.45}
\def\xxposonglety{19}%16

\def\xxonglet{\textsf{Cycle 01}}
\def\xxauteur{\textsl{Émilien Durif \\ Xavier Pessoles}}


\def\xxpied{%
Cycle \xxnumpartie -- \xxpartie\\
Chapitre \xxnumchapitre -- \xxactivite -\xxnumactivite -- \xxnomchapitre%
}

\setcounter{secnumdepth}{5}
\chapterimage{Fond_ALG}
\def\xxfigures{}

\def\xxcompetences{%
\textsl{%
\textbf{Savoirs et compétences :}\\
\begin{itemize}[label=\ding{112},font=\color{ocre}]
\item AA.S1 : Se familiariser aux principaux composants d'une machine numérique
\item AA.C4 : Comprendre un algorithme et expliquer ce qu’il fait
\item AA.C5 : Modifier un algorithme existant pour obtenir un résultat différent
\item AA.C6 : Concevoir un algorithme répondant à un problème précisément posé
\item AA.C7 : Expliquer le fonctionnement d’un algorithme
\item AA.C9 : Choisir un type de données en fonction d’un problème à résoudre
\item AA.S6 : Variables : notion de type et de valeur d’une variable, types simples.
\item AA.S7 : Expressions et instructions simples
\item AA.S10 : Notion de fonction informatique
\end{itemize}
}}


%Infos sur les supports
\def\xxtitreexo{Types simples}
\def\xxsourceexo{\hspace{.2cm} \footnotesize{}}
%\def\xxtitreexo{Titre EXO}
%\def\xxsourceexo{\hspace{.2cm} \footnotesize{Source EXO}}


%---------------------------------------------------------------------------


