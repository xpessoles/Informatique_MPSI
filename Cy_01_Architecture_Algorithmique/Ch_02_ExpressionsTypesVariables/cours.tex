\section{Expressions}


Les \emph{valeurs} désignent les données manipulées par un algorithme ou une fonction. Une valeur 
peut ainsi être  : un nombre, un caractère, une chaîne de caractères, une valeur de vérité 
(\texttt{Vrai} ou \texttt{Faux}) et bien d'autres ... En Python, comme dans la plupart des 
langages 
de programmation, ces valeurs sont \emph{typées} selon l'objet qu'elles représentent : il y a ainsi 
des valeurs de type entier, flottant, chaîne de caractères, booléens ...   Leur représentation en 
mémoire varie beaucoup d'un langage à l'autre, mais ce sont souvent les mêmes objets que l'on 
cherche à traduire.\\

Une \emph{expression} est une suite de caractères faisant intervenir des valeurs et des 
opérations, afin d'obtenir une nouvelle valeur. Pour calculer cette nouvelle valeur,
la machine doit \emph{évaluer} l'expression. Voici des exemples d'expressions : \texttt{42 , 1+4 , 
1.2 / 3.0 , x+3 }.\\
En Python, pour évaluer une expression, il suffit de la saisir dans un interpréteur, qui
calcule et affiche alors la valeur qu'il a calculée :\\
\begin{pyconsole}
42
1+4
\end{pyconsole}

\section{Types simples}

En programmation, associer un type à une valeur permet :
\begin{enumerate}
\item de classer les valeurs en catégories similaires. Par exemple, tous les
entiers (type int), tous les flottants (type float)...
\item de contrôler quelles opérations sont faisables sur ces valeurs : par
exemple, tout entier (type int) pourra être utilisé dans une soustraction, ..., alors 
qu'une chaîne ne le pourra pas. Il sera également impossible d'additioner un entier et un booléen. 
Dans certains langages, par exemple en Caml, il est même impossible d'additioner un entier et un 
flottant.                                             
\end{enumerate}

Une expression n'a a priori pas de type, car le type de la valeur qu'elle renvoie dépend des types 
des sous-expressions. Ainsi \texttt{a+b} sera un entier (resp. un flottant, une chaîne) si 
\texttt{a} et \texttt{b} le sont, mais sera un flottant si \texttt{a} est un entier et \texttt{b} 
un flottant.\\

Pour afficher le type d'une valeur ou d'une expression après l'avoir évaluée, on utilise 
\texttt{type} :\\
\begin{pyconsole}
type(42)
type(1.2/3.0)
\end{pyconsole}
Le mot qui suit \texttt{class} indique le type de la valeur, entier (\texttt{int} en anglais) pour 
la première et flottant (\texttt{float} en anglais) pour la seconde. Le mot \texttt{class} fait 
référence au fait que Pytho, est un langage \emph{orienté objet}, ce que nous ignorerons pour 
l'instant.\\

Passons en revue les types et les opérations les plus courants en Python .

\subsection{Entiers}

Ce type est noté \texttt{int} en Python .\\

\noindent$\bullet$ \textbf{Constantes :} Ce sont les entiers écrits en base 10, et ils ne sont pas 
bornés en Python .\\

\noindent$\bullet$ \textbf{Opérateurs :}
\begin{enumerate}
 \item + : addition usuelle.
 \item - : soustraction usuelle (\texttt{15-9} renvoie \texttt{6}), mais aussi opposé 
(\texttt{-4}). 
\item  // : \emph{division entière} : \texttt{a//b} correspond au quotient de la 
division euclidienne de \texttt{a} par \textbf{b} si \texttt{b} est strictement positif 
(\texttt{256 // 3} renvoie \texttt{85} car $256 = 85*3 + 1$). Mais si \texttt{b} est strictement 
négatif, alors \texttt{a//b} renvoie ce quotient moins 1 ($15 = (-4)\times(-4)-1=(-4)\times(-3)+3$, 
le quotient de la division euclidienne de 15 par $-4$ est $-3$, mais \texttt{15//-4} renvoie 
\texttt{-4}). Cette division n'est pas définie si \texttt{b} est nul.
\item $\%$ : \emph{modulo} : même remarque que dans le point précédent, relativement au reste de la 
division euclidienne cette fois (\texttt{256 \% 3} renvoie 
\texttt{1}, \texttt{15\%-4} renvoie \texttt{-1}).
\item ** : exponentiation (\texttt{2\raisebox{0.3ex}{**}3} renvoie \texttt{8}).
\end{enumerate}

Les \emph{règles de précédence}, autrement dit les règles de priorité entre opérations, sont 
similaires aux règles mathématiques usuelles.

\subsection{Flottants}

Ce type est noté \texttt{float} en Python .\\

\noindent$\bullet$ \textbf{Constantes :} Ce sont les \emph{nombres à virgule flottante}. Nous en
donnerons une définition précise dans un chapitre ultérieur : pour simplifier, disons pour 
l'instant que ce sont des nombres à virgule, avec un nombre borné de chiffres dans leur écriture.\\

\noindent$\bullet$ \textbf{Opérateurs :}
\begin{enumerate}
 \item + : addition usuelle.
 \item - : soustraction usuelle, et aussi opposé.
\item  / : division usuelle.
\item ** : exponentiation. Remarquer la différence entre \texttt{2\raisebox{0.3ex}{**}100} et 
\texttt{2.\raisebox{0.3ex}{**}100} ou \texttt{2\raisebox{0.3ex}{**}100.}.
\end{enumerate}

\noindent\textbf{Avertissement :} ce qui précéde est valable en Python 3. Attention à l'Opérateur 
\texttt{/} en Python 2 !!!!\\

\subsection{Booléens}

En logique et mathématiques, le mot \emph{booléen} a été donné en hommage au mathématicien 
britannique George Boole. Ce type est noté \texttt{bool} en Python .\\

\noindent$\bullet$ \textbf{Constantes :} Il y en a deux : \texttt{True} et \texttt{False}.\\

\noindent$\bullet$ \textbf{Opérateurs :}
\begin{enumerate}
 \item \texttt{not} : négation usuelle.
 \item \texttt{and} : conjonction usuelle.
 \item \texttt{or} : disjonction usuelle.
\end{enumerate}

\begin{exemple}
\texttt{True or False} renvoie \texttt{True}, \texttt{not ( False or 
True) and True} renvoie \texttt{False}.
\end{exemple}

\noindent$\bullet$ \textbf{Opérateurs de comparaison :}
\begin{enumerate}
 \item \texttt{==} : test d'égalité : \texttt{2==3} renvoie \texttt{False}, \texttt{4==4} renvoie 
\texttt{True}. Problème : \texttt{0.1+0.2==0.3} renvoie \texttt{False} ! Nous y reviendrons plus 
tard. Il ne faut pas confondre \texttt{==} avec \texttt{=}, symbole de l'\emph{affectation}.
 \item \texttt{!=} : \texttt{a != b} est une écriture équivalente à \texttt{not (a == b)}.
 \item \texttt{<,>,<=,>=} : ce à quoi on s'attend.
\end{enumerate}


\begin{rem}
Python \ permet les chaînes de comparaisons : \texttt{1<2<3} renvoie
\texttt{True}, et \texttt{(1<2<-2) and (-5<2<6)} renvoie \texttt{False}.
\end{rem}

\subsection{Conversions}

Il est possible de convertir en entier en flottant avec la fonction \texttt{float}. La réciproque est possible dans une certaine 
mesure : \texttt{float(2)} renvoie \texttt{2.0}, \texttt{int(2.0)} renvoie \texttt{2}, mais \texttt{int(2.5)} renvoie \texttt{2} et 
\texttt{int(-3.5)} renvoie \texttt{-3}.

La fonction \texttt{int} appliquée à un nombre flottant procède donc à une troncature, ce n'est pas la fonction donnant la partie entière d'un nombre flottant (fonction \texttt{floor} du module \texttt{math}).

\section{Types composés}

On dit qu'une valeur est de \emph{type composé} si elles formées à partir de plusieurs autres 
valeurs de types plus simples. De nombreuses constructions sont définies sur tous les types 
composés.\\

\subsection{$n$-uplets ou tuples}

Ce type est noté \texttt{tuple} en Python .\\

\noindent$\bullet$ \textbf{Construction et composantes :}\\

Les $n$-uplets généralisent les couples. Un couple s'écrira \texttt{(a,b)}, un triplet 
\texttt{(a,b,c)} ... etc. Il existe même des 1-uplets : il faut les écrire \texttt{(a,)}, et non 
\texttt{(a)} : cette dernière écriture est équivalente à \texttt{a}, tout simplement.\\
Il n'est pas obligatoire que les éléments d'un tuple soient de même type. Par exemple, 
\texttt{(1,1.2,True)} est un triplet tout à fait valable.\\
On accède à la $k\ieme$ coordonnée d'un tuple \texttt{t} par la commande \texttt{t[k]}. Attention : 
\textbf{en Python , les coordonnées sont numérotées à partir de 0} ! On dit que $k$ est 
l'\emph{indice} de \texttt{t[k]}. Ainsi, si \texttt{t = (1,2,3)}, \texttt{t[0]} vaut 1, 
\texttt{t[2]} vaut 3, et \texttt{t[3]} n'existe pas.\\

On dit que les tuples sont \emph{immuables} : il n'est pas possible de changer un élément d'un 
tuple.\\

\noindent$\bullet$ \textbf{Opérateurs :}
\begin{enumerate}
 \item + : concaténation. \texttt{(1,2,3)+(1,4,5,6)} renvoie \texttt{1,2,3,1,4,5,6}.
 \item \texttt{len} : \emph{longueur} du tuple. \texttt{len(1,2,3)} renvoie \texttt{3}.
 \item \texttt{in} : test d'appartenance. \texttt{3 in (1,2,3)} renvoie \texttt{True}, alors que 
\texttt{3 in (1,2,4,5)} renvoie \texttt{False}.
\item le \emph{slicing}, qui permet de construire des sous-tuples : \texttt{(5,2,6,7,8,9)[1:5]} 
renvoie \texttt{(2,6,7,8)} (attention à la borne de droite !).\\
\end{enumerate}


\subsection{Chaînes de caractères}

Ce type est noté \texttt{str} en Python , pour \emph{string} en anglais.\\

\noindent$\bullet$ \textbf{Construction et composantes :}\\


Ce sont des suites finies de caractères, un caractère étant en python un caractère du clavier. On 
les note entre guillemets ou apostrophes : \texttt{``Ceci est un chaîne de caractères''} ou  
\texttt{'Ceci est une chaîne de caractères'}. La chaîne vide se note \texttt{``}\texttt{''} ou 
\texttt{''}. Un caractère est une chaîne de longueur 1.\\
On accède aux composantes d'une chaîne de la même manière que pour un tuple.\\

\noindent$\bullet$ \textbf{Opérateurs :}

Ce sont les mêmes que pour les tuples.

\subsection{Listes}

Ce type est noté \texttt{list} en Python .\\

Comme les tuples, les \emph{listes} sont des suites finies de valeurs arbitraires, mais cette fois 
ils sont \emph{mutables} : on peut changer la valeur d'une composante d'une liste.\\
Les listes s'écrivent entre crochets. On change la valeur de la $k\ieme$ composante d'une liste 
\texttt{L} grâce à la commande \texttt{L[k] = n}, où \texttt{n} est la nouvelle valeur. Ainsi :\\
\begin{pyconsole}
L = [1,2,3] 
L[0] = 4  
L 
\end{pyconsole}

Les autres opérateurs sont les mêmes que pour les tuples et les chaînes.

\subsection{Conversions}

On peut convertir des types simples en chaînes avec la fonction \texttt{str} : \texttt{str(2.3)} 
renvoie \texttt{'2.3'}.\\
À l'inverse, \texttt{int, float} et \texttt{bool} permettent de faire l'inverse quand cela est 
cohérent : \textbf{bool('True')} renvoie \texttt{True}, mais \texttt{int('2.3')} renvoie une 
erreur.\\

Il est également possible de passer d'un type composé à un autre grâce aux fonctions \texttt{str}, 
\texttt{tuple} et \texttt{list} : \texttt{str( (1,'a',[1,2]) )} renvoie \texttt{``(1,'a',[1,2])''}, 
\texttt{tuple( [1,2,3] )} renvoie \texttt{(1,2,3)} et \texttt{list('Coucou ?')} renvoie 
\texttt{['C', 'o', 'u', 'c', 'o', 'u', ' ', '?']}.


\section{Variables}

\subsection{Qu'est-ce qu'une variable ?}

Une \textbf{variable} désigne une zone mémoire de l'ordinateur dans la RAM.\\
Il s'agit d'un endroit où l'on peut \textbf{stocker} une valeur, y \textbf{accéder} et 
\textbf{changer} cette valeur.\\

Pour faire référence à une variable, on utilise un nom de variable, en général composé d'une ou 
plusieurs lettres. Dans la mesure du possible, on choisit un nom explicite, ce qui permet une 
meilleure lecture du programme.

On essaiera tant que possible d'éviter d'utiliser les caractères suivants comme noms de variables:
\begin{itemize}
  \item \texttt{O} (O majuscule), se confond avec $0$ ;
  \item \texttt{o} (O minuscule), se confond avec $0$ ;
  \item \texttt{l} (L minuscule), se confond avec $1$. 
\end{itemize}


\subsection{Affectation}

Quand on stocke une valeur \texttt{d} dans une variable \texttt{var}, on dit que l'on 
\textbf{affecte} \texttt{d} à la variable \texttt{var}. La valeur
\texttt{d} est encore appelée \textbf{la donnée}.

En Python , cette affectation est faite avec la commande \texttt{=}, comme suit.  

\begin{pyconsole}
>>> var  = d
\end{pyconsole}


Les variables (et donc l'affectation) sont incontournables : si vous ne stockez pas une donnée 
(résultat par exemple d'un calcul), vous ne pourrez pas la réutiliser, ni la modifier.



\begin{exemple}~
\begin{pyconsole}
>>> x = 42	# x prend la valeur 42.
>>> x = x + 2	# on additionne à la donnée stockée dans x  
		# le nombre 2 et on stocke le résultat de 
		# nouveau dans x
>>> x 		# on accède à la valeur qui est mémorisée 
		# dans x.
 44
\end{pyconsole}
\end{exemple}

Dans la variable affectée, la nouvelle donnée \og écrase\fg\ la donnée précédente : on perd son 
ancienne valeur.

Dans un programme Python \ , on utilisera l'instruction \texttt{print(x)}
pour afficher dans la console la valeur de la variable \texttt{x}.

Il est également possible d'affecter des valeurs à plusieurs variables en une fois, en utilisant 
des tuples. Ainsi : \texttt{a,b = (1,2)} affecte la valeur 1 à \texttt{a} et la valeur 2 à 
\texttt{b}.



%TODO parler de représentation en mémoire, cf. ancien chapitre 5.

\section{Fonctions}
\subsection{Objectif : modularité des programmes}

Pour répondre à un problème donné, il est souvent nécessaire d'enchaîner plusieurs voire un nombre 
important d'instructions.

Pour améliorer la lisibilité d'un programme et aussi pour pouvoir réutiliser 
ces instructions classiques (dans le même programme ou dans un autre),
on privilégie les programmes simples (ou sous-programmes) appelés \textbf{fonctions}.

Dans chaque langage, il y a déjà un nombre incalculable de fonctions déjà construites (en Python , 
par exemple, \texttt{print}) mais on s'aperçoit très vite que l'on a besoin de créer ses propres 
fonctions.


\subsection{\'Ecriture en langage naturel}

\fbox{\begin{minipage}{15cm}

On appelle \texttt{DefFonction} l'opération qui consiste à définir une nouvelle fonction nommée 
\textbf{nom-de-la-fonction} présentée comme suit : 
\begin{center}
\texttt{DefFonction} \begin{tabular}{|l}
					\textbf{nom\_ de\_la\_fonction}(\textit{paramètres})\\
 \hspace{5ex} \og commentaire expliquant la fonction \fg\\
\hspace{5ex}	bloc d'instructions\\
\hspace{5ex}	sortie du résultat\\
					\textbf{fin}
					\end{tabular}
\end{center}
\end{minipage}}

\vspace{0.5cm}
\begin{exemple}
On veut convertir en secondes une durée donnée en heures, minutes et secondes : 

\texttt{DefFonction} 
\begin{tabular}{|l}
				\textbf{conversion}(h,m,s)\\
\hspace{5ex}  \og convertit en secondes une durée donnée en heures, \\
  \hspace{10ex} minutes et secondes \fg\\
\hspace{5ex} $t \leftarrow 3600*h+60*m+s$\\
\hspace{5ex} Renvoyer t\\
				  \textbf{fin}
				  \end{tabular}

\vspace{2ex}									
\textbf{Une fois définie, pour l'utiliser, on écrit son nom suivi, entre parenthèses, des 
paramètres.} 

\vspace{0.2cm}
Pour convertir 2h35mn45s, on écrit \textbf{conversion}(2,35,45) (réponse : 9345).

\end{exemple}

\begin{rem}

\begin{itemize}
	\item Pour bien se faire comprendre, il est important de choisir un nom de fonction 
explicite. 
Il faut aussi mettre des commentaires pour expliquer ce que fait la fonction.
	\item Il est nécessaire de bien cibler les paramètres dont on a besoin 
c'est-à-dire les données nécessaires à l'exécution de la fonction.
	\item Il faut repérer ce que renvoie la fonction : rien (exemple : un simple affichage ou 
la modification d'un fichier...), un nombre entier, 
un flottant, une liste...

Reprenons l'exemple de la conversion 

\begin{description}
 \item[1er cas.] 
\

\texttt{DefFonction} 
\begin{tabular}{|l}
				\textbf{conversion}(h,m,s)\\
 \hspace{5ex}  \og convertit en secondes une durée donnée en heures, \\
  \hspace{5ex}  \hspace{5ex} minutes et secondes \fg\\
	\hspace{5ex} $t \leftarrow 3600*h+60*m+s$\\
	\hspace{5ex} Renvoyer t\\
				  \textbf{fin}
				  \end{tabular}

Si on écrit \texttt{rep}=\textbf{conversion}(h,m,s), on trouvera dans \texttt{rep} un nombre entier.

\item[2ème cas.] 
\

\texttt{DefFonction} 
\begin{tabular}{|l}
				\textbf{conversion}(h,m,s)\\
 \hspace{5ex}  \og convertit en secondes une durée donnée en heures, \\
  \hspace{10ex} minutes et secondes \fg\\
 \hspace{5ex} $t \leftarrow 3600*h+60*m+s$\\
 \hspace{5ex} Afficher t\\
				  \textbf{fin}
				  \end{tabular}

Si on écrit \texttt{rep}=\textbf{conversion}(h,m,s), on  ne trouvera rien dans \texttt{rep} mais la 
valeur s'affichera à l'écran.
\end{description}



	\item Enfin, il faut toujours tester sa fonction sur un ensemble significatif de valeurs 
pour repérer d'éventuelles erreurs ou manquements.
\end{itemize}
\end{rem}

\subsection{\'Ecriture en Python }

\begin{pyconsole}
 def nom_de_la_fonction(parametres):
     """ commentaire expliquant la fonction"""
     bloc d'instructions
     return(resultat)
 
 
\end{pyconsole}

\begin{rem}
 L'indentation des blocs n'est pas là que pour décorer. Dans la plupart des langages de 
programmation on \emph{conseille} d'indenter les différents blocs afin de faciliter la lecture du 
code. En Python , l'indentation est \emph{signifiante} : après le mot-clé \texttt{def}, chaque 
ligne indentée fait partie de la fonction. La première ligne non indentée rencontrée marque la 
fin de la fonction : cette ligne ne fait plus partie de la fonction, mais celles qui précéde en 
font partie. Il est donc \emph{impératif} d'indenter quand il le faut, et seulement 
quand il le faut. On rencontrera ce phénomène constamment en Python .
\end{rem}
