\section{L'instruction conditionnelle}

\subsection{Algorithme}

Quand on veut écrire un programme, on souhaite établir des connections logiques
entre les instructions. 
Ainsi, l'instruction conditionnelle a pour objet
d'intervenir dans le choix de l'instruction suivante en fonction d'une expression booléenne qu'on 
désignera par \textbf{condition} : 

%\vspace{1cm}
\begin{center}
\fbox{\begin{minipage}{12cm}

\begin{center}
\begin{tabular}{|l}
\textbf{Si} condition\\
\hspace{0.7cm}
\begin{tabular}{l}
\textbf{alors } bloc d'instructions 1\\
\textbf{sinon} bloc d'instructions 2\\
\end{tabular}\\
\textbf{Fin-du-Si}
\end{tabular}
\end{center}



\begin{center}
signifie que 
\end{center}

\begin{itemize}
\item \textbf{Si} la condition est vérifiée (expression booléenne=\texttt{True}) 

\textbf{alors} le programme exécute les instructions du bloc 1;

\item \textbf{si} la condition n'est pas vérifiée (expression booléenne=\texttt{False})

\textbf{alors }le programme exécute les instructions du bloc 2.
\end{itemize}
\end{minipage}}

\end{center}
%\newpage


\subsection{Syntaxe en Python}

%\begin{lstlisting}
\begin{pyverbatim}
if condition : 
	bloc d instructions 1
else :
	bloc d instructions 2
\end{pyverbatim}


 \begin{itemize}
 \item \textbf{Si} et \textbf{Sinon} se traduisent par \texttt{if} et \texttt{else}. 
 
 \item \textbf{Alors} se traduit par \og \textbf{:} \fg\ en bout de ligne et une indentation de toutes les 
lignes du bloc 1.
 
 \item \textbf{Fin-du-Si} se traduit par un retour à la ligne sans indentation. 
 
 \end{itemize}

 
\subsection{Exemple}

On veut tester si un nombre $x$ est proche de 3 à $10^{-4}$ près. On peut alors écrire la fonction suivante. 

\begin{pyverbatim}
def est_proche(x):
    """x est proche de 3 à 10**-4 près ?"""
    distance = abs(x-3)
    if distance <= 10**(-4) : 
        return True 
    else : 
        return False
\end{pyverbatim}



\begin{rem} 
La partie \textbf{sinon} est optionnelle. Sans elle, si la condition n'est pas vérifiée, alors la 
machine n'exécute rien.
\end{rem}

\begin{rem}
  On pouvait très bien remplacer cette boucle conditionnelle avec un usage astucieux des booléens, comme suit.
\begin{pyverbatim}
def est_proche(x):
    """x est proche de 3 à 10**-4 près ?"""
    distance = abs(x-3)
    return distance <= 10**(-4)
\end{pyverbatim}
La fonction est alors plus concise, mais plus difficilement lisible. 
\end{rem}


\subsection{\`A propos des conditions}

L'expression booléenne derrière le \textbf{si} joue le rôle de test. Pour exprimer cette condition, 
on a besoin des \textbf{opérateurs de comparaison} (inférieur strict, supérieur strict, inférieur ou 
égal, supérieur ou égal, égal à, différent de) et des \textbf{connecteurs logiques}
(non, et, ou).

%\begin{exs}
\newpage

\begin{multicols}{2}
\begin{itemize}

\item Calcul du carré d'un nombre positif. 

\begin{pyconsole}
x=4
if x >= 0 : 
    car = x**2

car
\end{pyconsole}

\vfill\null
\columnbreak 

\item Comprenez-vous le message d'erreur ? 

%\begin{lstlisting}
%>>> x = -5
%>>> if x>= 0 :
%	carr  = x**2
%>>> carr
%Traceback (most recent call last):
%  File "<pyshell#4>", line 1, in <module>
%    carr
%NameError: name 'carr' is not defined
%\end{lstlisting} 


\begin{pyconsole}
x = -5
if x>= 0 :
    carr  = x**2
carr
\end{pyconsole} 

\end{itemize}
\end{multicols}
\begin{itemize}

\item Condition avec un \og et \fg{}. 

\begin{pyconsole}
x = 0.5 
if x >= -1 and x<= 1 :
    print("Il existe un angle theta tel que x = cos(theta).") 

\end{pyconsole}
 \end{itemize}

% \end{exs}




%\newpage
\subsection{Imbrication de plusieurs conditions}

On peut se trouver face à un problème qui se scinde en plus de deux cas (par exemple, dans le cas 
des équations du second degré, on teste si le discriminant est strictement positif, nul ou 
strictement négatif).
Dans ce cas, voici comment procéder.

\begin{center}


\fbox{\begin{minipage}{15cm}

\begin{center}
\begin{tabular}{|l}
\textbf{Si} condition 1\\
\hspace{0.7cm}
\begin{tabular}{l}
\textbf{alors} bloc d'instructions 1\\
\textbf{sinon si} condition 2\\
\hspace{0.7cm}
\begin{tabular}{l}
\hspace{7ex}\textbf{alors} bloc d'instructions 2\\
\hspace{7ex}\textbf{sinon si} condition 3\\
\hspace{14ex}\textbf{alors} bloc d'instructions 3\\
\ldots\\
\hspace{14ex}\textbf{sinon} bloc final
\end{tabular}\\
\end{tabular}\\
\textbf{Fin-du-Si}
\end{tabular}
\end{center}

\begin{center}
signifie que 
\end{center}

\begin{itemize}
\item \textbf{Si} la condition 1 est vérifiée (expression booléenne=\textit{True}) 

\textbf{alors} le programme exécute les instructions du bloc 1.

\item \textbf{Si} la condition 1 n'est pas vérifiée (expression booléenne=\textit{False}) mais que 
la condition 2 est vérifiée

\textbf{alors} le programme exécute les instructions du bloc 2.

\item \textbf{Si} les conditions 1 et 2 ne sont pas vérifiées  mais que la condition 3 est vérifiée

\textbf{alors} le programme exécute les instructions du bloc 3...

\item \textbf{Si} aucune condition n'est vérifiée

\textbf{alors} le programme exécute les instructions du bloc final.
\end{itemize}
\end{minipage}}

\end{center}

\vspace{0.5cm}
\begin{rem}\label{rem-filtre}
\begin{itemize}
\item Si la \texttt{condition 1} est vérifiée, le programme ne se préoccupe pas des conditions 
suivantes.
\item Chaque \textbf{sinon si} agit comme un \textbf{filtre}. L'ordre de ces filtres est important. 
Ainsi, les séquences d'instructions :\\

\begin{tabular}{|l}
 $x \leftarrow$ 3 \\
\textbf{Si} $x >$ 6\\
\hspace{0.7cm}
\begin{tabular}{l}
\textbf{alors} renvoyer $x + 2$\\
\textbf{sinon si} $x > 4$\\
\hspace{0.7cm}
\begin{tabular}{l}
\hspace{7ex}\textbf{alors} renvoyer $x + 4$\\
\hspace{7ex}\textbf{sinon si}$ x > 2$\\
\hspace{14ex}\textbf{alors} renvoyer $x + 6$\\
\hspace{14ex}\textbf{sinon} renvoyer $x$
\end{tabular}\\
\end{tabular}\\
\textbf{Fin-du-Si}
\end{tabular}\\

et \\

\begin{tabular}{|l}
 $x \leftarrow 3$\\
\textbf{Si} $x > 6$\\
\hspace{0.7cm}
\begin{tabular}{l}
\textbf{alors} renvoyer $x + 2$\\
\textbf{sinon si} $x > 2$\\
\hspace{0.7cm}
\begin{tabular}{l}
\hspace{7ex}\textbf{alors} renvoyer $x + 6$\\
\hspace{7ex}\textbf{sinon si} $x > 4$\\
\hspace{14ex}\textbf{alors} renvoyer $x + 4$\\
\hspace{14ex}\textbf{sinon} renvoyer $x$
\end{tabular}\\
\end{tabular}\\
\textbf{Fin-du-Si}
\end{tabular}\\


ne renvoient pas la même chose.

\end{itemize}
\end{rem}

\subsection{Syntaxe en Python}

\begin{pyverbatim}
 if condition 1 :
	bloc d instructions 1
 elif condition 2 :
	bloc d instructions 2
 elif condition 3 :
	bloc d instructions 3
.
.
.
 else :
	bloc final
\end{pyverbatim}

\begin{itemize}
 \item \textbf{Sinon si} se traduit par \texttt{elif}. 
\end{itemize}

%\begin{exo}
\begin{exemple}
Écrire les deux séquences d'instructions de la remarque \textbf{\ref{rem-filtre}} en Python, et 
constater que l'on obtient bien deux résultats différents.
\end{exemple}
%\end{exo}

%\begin{ex}
Voici une fonction qui calcule le maximum de trois entiers \texttt{a, b, c} :
\begin{pyverbatim}
def max3 (a, b, c) :
    """ renvoie le maximum de a, b ,c.
    précondition : a, ,b et c sont 3 entiers """
    if a <= c and b <= c :
        return c
    elif a <= b and c < b :
        return b
    else :
        return a
\end{pyverbatim}

%\end{ex} 


\section{Boucles définies}

\subsection{Un programme très simple}

Écrivons un programme pour saluer la classe (disons les 8 premiers
élèves) :
\begin{pyverbatim}
print('Bonjour Baptiste')
print('Bonjour Lisa')
print('Bonjour Pierrick')
print('Bonjour Louise-Eugénie')
print('Bonjour Qasim')
print('Bonjour Lorenzo')
print('Bonjour Arthur')
print('Bonjour Ylies')
\end{pyverbatim}

Mais combien de travail faut-il faire si :

\begin{itemize}
\item on veut dire bonsoir et non bonjour?
\item on veut dire «Baptiste, comment vas-tu? \textit{etc.}» et non
  «Bonjour Baptiste, \textit{etc.}» ?
\end{itemize}

Et s'il y a 500 élèves ?


\subsection{Le principe DRY}

«\textbf{D}on't \textbf{R}epeat \textbf{Y}ourself» est une philosophie en programmation 
informatique consistant à éviter la redondance de code au travers de l'ensemble d'une application 
afin de faciliter la maintenance, le test, le débogage et les évolutions de cette dernière.\\
La philosophie DRY est explicitée par la phrase suivante :\\
  \textit{«Dans un système, toute connaissance doit avoir une représentation
   unique, non-ambiguë, faisant autorité ».}

Ici le programme devrait:
\begin{itemize}
\item définir la liste des prénoms auxquels dire bonjour;
\item dire qu'on veut effectuer un même traitement sur tous les prénoms;
\item dire que ce traitement consiste à afficher «Bonjour» suivi du prénom.
\end{itemize}


Nous avons déjà vu comment définir une liste :

\begin{pyverbatim}
prenoms = [ 'Baptiste', 'Lisa', 'Pierrick',\
'Louise-Eugénie', 'Qâsim', 'Lorenzo', 'Arthur', 'Ylies' ]
\end{pyverbatim}

Il faut maintenant s'occuper des deux points suivants : définir le traitement à effectuer sur 
chacun de ces 
prénoms, et l'\textbf{itérer}. Pour cela nous allons utiliser une \textbf{boucle itérative 
définie}, autrement dit nous allons \textbf{répéter} l'application d'une même séquences
d'instructions 
sur une liste \textbf{définie à l'avance}.


\fbox{\begin{minipage}{12cm}

\begin{center}
\begin{tabular}{|l}
\textbf{Pour} variable \textbf{dans} liste \textbf{répeter}\\
\hspace{0.7cm}
\begin{tabular}{l}
bloc d'instructions b\\
\end{tabular}\\
\textbf{Fin-de-la-boucle}
\end{tabular}
\end{center}



\begin{center}
signifie que 
\end{center}

\textbf{pour} chaque élément de la liste \texttt{liste},

le programme exécute les instructions du bloc b.

\end{minipage}}


\subsection{Syntaxe en Python}

\begin{pyverbatim}
for variable in liste :
    instructions
\end{pyverbatim}

Ici encore, la ligne contenant le mot-clé \textbf{for} doit se finir par un \og \textbf{:} \fg\ et les 
instructions du bloc doivent être indentées. La fin de la boucle est marquée par un retour à la 
ligne non indenté.\\

Maintenant que la liste \texttt{prenoms} est définie, dire bonjour à nos huit élèves s'écrit ainsi :


\begin{pyverbatim}
for x in prenoms:
    print('Bonjour ' + x)
\end{pyverbatim}


\subsection{Autre problème ...}
On souhaite calculer $u_{20}$ où $u$ est la suite définie par $\left\{
\begin{array} {l}
  u_{0}  = 5\\
  \forall n\in\N\ u_{n+1} = 2u_{n}- n - 3 \end{array}
  \right.
$.

Il y a deux approches possibles :

\begin{enumerate}
\item résoudre le problème mathématiquement, c'est-à-dire déterminer une expression de $u_n$ en fonction de 
$n$, avant de faire une application numérique.
\item  calculer de proche en proche $u_1$, $u_2$, $u_3$,$u_4\ \cdots$ et enfin $u_{20}$.
\end{enumerate}

La première méthode est la meilleure si on sait faire.\\
La seconde promet d'être longue et pénible, surtout si l'on vous demande de calculer $u_{10000}$. 
Mais là encore la boucle itérative va nous aider.\\


On peut proposer un algorithme du type :\\


\begin{tabular}{|l}
 $x \leftarrow5$\\
\textbf{Pour} k \textbf{de} 0 \textbf{à} \emph{???} \textbf{répéter}\\
\hspace{0.7cm}
\begin{tabular}{l}
$x \leftarrow 2x - k -3$\\
\end{tabular}\\
\textbf{Fin-de-la-boucle}
\end{tabular}\\

et on espère qu'après cette boucle, $x$ vaut $u_{20}$. \\

Mais il reste deux questions :
\begin{enumerate}
 \item Y a t'il un lien entre cette boucle et celle vue précédemment où on parcourt les éléments 
d'une liste ?
\item Jusqu'où aller dans cette boucle 
\emph{???} 18, 19, 20, 21, 22 ? Comment être sûr de ne pas se tromper ?\\
\end{enumerate}

\subsection{Les intervalles d'entiers en Python}

Pour répondre à la première question, il suffit de remarquer que les entiers de 0 à 19 par exemple, 
sont en fait les éléments d'une liste : \texttt{[0, 1, 2, $\cdots$ , 19]}. En Python, cette liste 
s'écrit de la manière suivante : \texttt{range(20)}.\\
Précisément, si $a$ et $b$ sont deux entiers, \texttt{range(a,b)} contient les éléments de 
l'intervalle semi-ouvert $\ii{a,b}$, dans l'ordre croissant. Avec un seul argument, 
\texttt{range(b)} signifie \texttt{range(0,b)}.\\
Redisons-le, car c'est un fait important en Python : \texttt{range(a,b)} est intervalle 
\textbf{fermé} à gauche, \textbf{ouvert} à
droite\footnote{Voir \textit{Why numbering should start at zero}, E. W. Dijkstra,
  EWD831. Disponible en ligne.}.\\
  
Avec \texttt{range}, nous pouvons maintenant itérer sur une liste d'entiers :

\begin{pyconsole}
x = 5
for k in range(20):
    x = 2 * x - k - 3
  
x
\end{pyconsole}
% Résultat obtenu : $1048600$.\\

\begin{rem}
Mais est-ce bien $u_{20}$ ? Pour cela on pourra montrer par récurrence que $x=u_k$ à l'entrée de la boucle et $x=u_{k+1}$ à la sortie de la boucle. C'est la notion \textbf{d'invariant de boucle} que l'on détaillera dans un chapitre ultérieur.
\end{rem}

%\subsection{Les invariants de boucle}
%
%Un \textbf{invariant de boucle} est une proposition vérifiée à chaque tour d'une boucle. Plus 
%précisément, on distingue les invariants d'\textbf{entrée de boucle} et ceux de \textbf{sortie de 
%boucle}. C'est en fait une proposition de récurrence, qui doit être vrai au début de la première 
%boucle, puis se propager de tour de boucle en tour de boucle, jusqu'au dernier.\\
%Les invariants de boucles ont essentiellement deux intérêts :
%\begin{enumerate}
% \item \textbf{Démontrer} l'algorithme, grâce au principe de récurrence ;
% \item Répondre à notre question précédente : où commencer et où finir une boucle ?
%\end{enumerate}
%
%Il est \textbf{fortement conseillé} de toujours indiquer les invariants de boucle en commentaire 
%dans vos algorithmes.\\ 
%
%Par exemple :
%
%
%\begin{lstlisting}
%x = 5			# x = u_0 (initialisation)
%for k in range(20): 	
%    # x = u_k (invariant en entrée de boucle)
%    x = 2 * x - k - 3	
%    # x = u_{k+1} (invariant en sortie de boucle)	
%# après la boucle for, k vaut 19, et x vaut donc u_{20}.
%\end{lstlisting}
%  


%\subsection{Un exemple : test de primalité} 
%
%  On veut tester si un entier \pyv{n} est premier:
%\begin{lstlisting}
%def est_premier(n):
%    """ Renvoie True si n est premier, False sinon
%        Préconditon : n est un entier."""
%    b = True
%    for d in range(2,n):
%        # b => n non divisible par 2, 3, ..., d-1.
%        if n % d == 0:
%            b = False
%    # b <=> n premier
%    return b
%\end{lstlisting}
%
%Remarque: les derniers tours de boucle sont inutiles dès que la
%variable \pyv{b} a été mise à \pyv{False}. Les éxécuter tout de même est une perte de temps. Il 
%existe plusieurs possibilités pour améliorer cela~:
%
%L'instruction \pyv{break} :
%\begin{pyverbatim}
%def est_premier(n):
%    """ Renvoie True si n est premier, False sinon
%        Préconditon : n est un entier."""
%    b = True
%    for d in range(2,n):
%        # b => n pas divisible par 2, 3, ..., d-1.
%        if n % d == 0:
%            b = False
%            break
%    # b <=> n est premier
%    return b
%\end{pyverbatim}
%
%L'utilisation d'un \pyv{return} en milieu de boucle, à favoriser en Python{} pour 
%une question de style et d'élégance~:
%
%\begin{pyverbatim}
%def est_premier(n):
%    """ Renvoie True si n est premier, False sinon
%        Préconditon : n est un entier."""
%    for d in range(2,n):
%        # n n'est pas divisible par 2, 3, ..., d-1.
%        if n % d == 0:
%            return False
%    return True
%\end{pyverbatim}

\section{Boucles indéfinies ou conditionnelles}

\subsection{Algorithme}

On peut aussi être amené à répéter un bloc d'instructions sans savoir combien de fois on devra le 
répéter. 

%\begin{ex}
Disposant d'une suite croissante, non majorée, on cherche à trouver le plus petit entier $p$ 
tel que la valeur au rang $p$ dépasse 10000.
%\end{ex}

Dans ce cas, on utilise la boucle \textbf{Tant que} qui permet de répéter le bloc d'instructions 
tant qu'une certaine condition est vérifiée.


\vspace{1cm}


\fbox{\begin{minipage}{15cm}
\begin{center}
\begin{tabular}{|l}
\textbf{Tant que} condition\\
\hspace{0.7cm}
\begin{tabular}{l}
\textbf{faire } bloc d'instructions \\
\end{tabular}\\
\textbf{Fin-du-Tant-que}
\end{tabular}
\end{center}

\begin{center}
signifie que 
\end{center}

\begin{description}
\item[Tant que] la condition est vérifiée (expression booléenne=\textit{True}) 

\item[Faire]  le bloc d'instructions.
\end{description}
\end{minipage}}

\subsection{Syntaxe en Python}


\begin{pyverbatim}
while condition :
    instructions
\end{pyverbatim}



\vspace{0.5cm}
%\begin{ex}
\label{ex-suite}
Rechercher le premier entier $n$ tel que la somme des entiers de 1 à n dépasse 11.

\begin{pyconsole}
n = 1
s = 1
while s < 11 :
    n = n + 1
    s = s + n
    
n
\end{pyconsole}

\begin{center}
REPONSE  : $n=5$  (dans ce cas $s=15$)
\end{center}

%\end{ex}


%\begin{ex}
\label{ex-syracuse} Conjecture de Syracuse :\\
On note $f : \N^{*} \to \N^{*}$ l'application vérifiant, pour tout $n$ pair
$f(n)=n/2$ et tout $n$ impair et $f(n)=3n+1$.

On conjecture que pour tout entier $n$, il existe $k$ tel que
$f^{k}(n)=1$.

Voici un algorithme calculant, pour tout $n$ donné, le plus petit
entier $k$ vérifiant $f^{k}(n) = 1$ :

%\begin{pyverbatim}
\begin{lstlisting}
def f(n):
    """Fonction de Syracuse.
    Précondition : n est un entier strictement positif"""
    if n % 2 == 0:
        return n // 2
    else:
        return 3 * n + 1
        
def syracuse(n):
    """Renvoie le premier entier k tel que  f\^k(n) = 0.
    Précondition : n est un entier strictement positif"""
    x = n
    k = 0
    while x != 1:
        x = f(x)
        k = k + 1
    return k
\end{lstlisting}
%\end{pyverbatim}

%\end{ex}


\begin{rem}
\begin{itemize}
\item Comme pour les boucles définies, nous sommes confrontés à un problème : comment démontrer un 
algorithme reposant sur une boucle indéfinie ? Pour cela, nous utilisons encore les invariants de 
boucle, afin de prouver qu'après la boucle, le résultat est bien celui voulu.
\item Pour montrer que la boucle \textbf{while} va réellement se finir on utilisera un \textbf{variant} de boucle. Cette notion sera également détaillée dans un chapitre ultérieur. Dans un premier temps nous retiendrons que généralement il faudra s'assurer que la condition contenu dans la boucle \textbf{while} sera généralement une suite d'entiers strictement croissante ou décroissante.
\end{itemize}
\end{rem}
%Mais avant même cela, il y a un point épineux : la boucle \textbf{while} va-t-elle vraiment se 
%finir ? Il faut démontrer ce que l'on appelle la \textbf{terminaison} de l'algorithme. C'est là que 
%réside le danger des boucles \textbf{while} : si elles sont mal écrites, la condition de la boucle 
%ne devient jamais fausse, et la boucle est infinie.\\
%Pour montrer la terminaison d'un algorithme, on utilise cette fois un \textbf{variant} de boucle. 
%Il consiste à mettre en avant une variable dont la valeur est modifiée au cours des tours de 
%boucle, de telle sorte que cette modification finisse par rendre fausse la condition de la boucle.\\
%Donnons ces invariant et variant pour l'algorithme de l'exemple \textbf{\ref{ex-suite}} (nous ne 
%pouvons malheureusement pas le faire pour l'exemple \textbf{\ref{ex-syracuse}}, puisque comme son 
%nom l'indique, la conjecture de Syracuse n'a jamais été demontrée) :\\


%\begin{pyconsole}
%n = 1
%s = 1           # s = somme des i de 1 à n
%while s < 11 :  # invariant : s = somme des i de 1 à n
%    n = n + 1
%    s = s + n   # variant : s, qui est entier et strictement croissant
%\end{pyconsole}

%\begin{exo}
%Démontrer cet algorithme.
%\end{exo}


%\subsection{Exercices}
%
%\begin{exo}
% Calculer $2^9$ à l'aide d'une boucle itérative.
%\end{exo}
%
%
%\begin{exo}
% \'Ecrire un algorithme affichant la table de multiplication de 9.
%\end{exo}
%
%\begin{exo}
% Calculer $16!$ à l'aide d'une boucle itérative.
%\end{exo}
%
%\begin{exo}
% Calculer
%  \begin{equation*}
%    \sum_{k=0}^{15} \frac{1}{k!}
%  \end{equation*}
%\end{exo}
%
%\begin{exo}
% Écrire une fonction calculant le nombre de chiffres d'un entier écrit en base 10.
%\end{exo}
%
%\begin{exo}
% On considère la suite $u$ définie par
%\begin{equation*}
%  \forall n\in \N^{*}\quad u_{n} = \sum_{k=1}^{n} \frac{1}{\sqrt{k}}
%\end{equation*}
%
%Quel est la plus petite valeur $n$ pour laquelle $u_{n}\geq 1000$?
%\end{exo}
%
%\begin{exo}
% Écrire une fonction trouvant le plus petit nombre premier supérieur ou
%égal à un entier donné.
%\end{exo}
%
%\begin{exo}
% Écrire une fonction calculant le nombre de diviseurs d'un entier $n$ donné.
%\end{exo}
%
%%\begin{exo}
% Calculer $p_{5}/q_{5}$ où $p$ et $q$ sont définies par:
%  \begin{align*}
%    p_{0} &= 1\\
%    q_{0} &= 1\\
%    \forall n\in \N \quad p_{n+1} &= p_{n}^{2} + 2q_{n}^{2}\\
%    \forall n\in \N \quad q_{n+1} &= 2p_{n}q_{n}
%  \end{align*}
%\end{exo}
%
%\begin{exo}
%  Définir la fonction $f$ qui à $x$ associe $\left\{\begin{array}{rcl}
							     2 & \mathrm{si} & x \in [-4,-2]\\
							    -x & \mathrm{si} & x \in [-2,0]\\
							      0 & \mathrm{si} & x \in [0,4]\\
							      \end{array} \right.$
%\end{exo}
%
%\begin{exo}
%  Ecrire une fonction calculant le produit des entiers impairs de 1 à $2n+1$.
%\end{exo}
%
%\begin{exo}
%  % \question
\begin{enumerate}[label = \emph{\alph*)}]
  \item \'Ecrire une fonction \pyv{neg(b)} qui renvoie la négation du booléen \pyv{b} sans utiliser \pyv{not}.
  \item \'Ecrire une fonction \pyv{ou(a,b)} qui renvoie le ou logique des booléen \pyv{a} et \pyv{b} sans utiliser \pyv{not}, \pyv{or} ni \pyv{and}.
  \item \'Ecrire une fonction \pyv{et(a,b)} qui renvoie le et logique des booléen \pyv{a} et \pyv{b} sans utiliser \pyv{not}, \pyv{or} ni \pyv{and}.
\end{enumerate}

%\end{exo}
%
%\begin{exo}
%  % \question 
Indenter de deux manières différentes la suite d'expressions suivante de manière à ce qu'à son exécution, le programme affiche soit la liste de tous les éléments de $L$ inférieurs ou égaux à $m$, soit juste le dernier.
\begin{pyverbatim}
from random import randrange
L = [randrange(100) for i in range(100)] # 100 valeurs entre 0 et 99.
m = 50
for x in L:
if x <= m:
p = x
print(p)
\end{pyverbatim}

%\end{exo}
%
%
%\begin{exo}
%  % \question 
Décrire ce que fait cette suite d'instructions. 
\begin{pyverbatim}
from random import randrange
# Un entier entre 0 et 999
n = randrange(1000) 
# n+1 valeurs entre 0 et 99.
L = [randrange(100) for i in range(n+1)] 
p = 0
for x in L:
    if x <= 10:
        p = p + x**2
\end{pyverbatim}
Et celle-ci ? 
\begin{pyverbatim}
from random import randrange
# Un entier entre 0 et 999
n = randrange(1000) 
# n+1 valeurs entre 0 et 99.
L = [randrange(100) for i in range(n+1)] 
p = 0
for i in range(n+1):
    if L[i] <= 10:
        p = p + i**2
\end{pyverbatim}

%\end{exo}
%
%
%\begin{exo}
%  % \question 
\'Ecrire une suite d'instructions permettant de calculer la somme des racines carrées des cinquante premiers entiers naturels non nuls. 

%\end{exo}
%
%\begin{exo}
%  % \question 
\'Ecrire la suite d'instructions suivantes dans un fichier, l'enregistrer puis l'exécuter (F5). \`A l'instant qui vous convient, presser Ctrl+C.
\begin{pyverbatim}
a = 1
while a>0:
    a = a+1
\end{pyverbatim}

%\end{exo}
%
%\begin{exo}
%  % \question 
Que fait la fonction suivante ? La corriger pour qu'elle coïncide avec le but annoncé. 

\begin{pyverbatim}
def sqrt_int(n):
    """Renvoie la partie entière de la racine carrée de n"""
    s = 0
    while s**2 <= n:
        s = s+1
        s = s-1
    return s
\end{pyverbatim}

%\end{exo}

