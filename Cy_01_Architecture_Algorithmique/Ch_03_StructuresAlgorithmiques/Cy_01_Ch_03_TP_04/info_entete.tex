% Déclaration des titres
% -------------------------------------


\graphicspath{{../../../style/png/}{images/}{../../../exos//}}
\lstinputpath{{../../../exos//}}

\def\discipline{Informatique}
\def\xxtete{Informatique}

\def\classe{\textsf{MPSI}}
\def\xxnumpartie{1}
\def\xxpartie{Architecture matérielle et initiation à l'algorithmique}
\def\xxdate{18 Octobre 2019}

\def\xxchapitre{3}
\def\xxnumchapitre{3}
\def\xxnomchapitre{Instructions conditionnelles et boucles}
\def\xxnumactivite{04}

\def\xxposongletx{2}
\def\xxposonglettext{1.45}
\def\xxposonglety{19}%16

\def\xxonglet{\textsf{Cycle 01}}
\def\xxauteur{\textsl{Émilien Durif \\ Xavier Pessoles}}


\def\xxpied{%
Cycle \xxnumpartie -- \xxpartie\\
Chapitre \xxnumchapitre -- \xxactivite -\xxnumactivite -- \xxnomchapitre%
}

\setcounter{secnumdepth}{5}
\chapterimage{Fond_ALG}
\def\xxfigures{}

\def\xxcompetences{%
\textsl{%
\textbf{Savoirs et compétences :}\\
\begin{itemize}[label=\ding{112},font=\color{ocre}]
\item AA.C4 : Comprendre un algorithme et expliquer ce qu'il fait
\item AA.C5 : Modifier un algorithme existant pour obtenir un résultat différent
\item AA.C6 : Concevoir un algorithme répondant à un problème précisément posé
\item AA.C7 : Expliquer le fonctionnement d'un algorithme
\item AA.C8 : Écrire des instructions conditionnelles avec alternatives, éventuellement imbriquées
\item AA.S8 : Instructions conditionnelles
\item AA.S9 : Instructions itératives
\end{itemize}
}}


%Infos sur les supports
\def\xxtitreexo{}
\def\xxsourceexo{\hspace{.2cm} \footnotesize{}}
%\def\xxtitreexo{Titre EXO}
%\def\xxsourceexo{\hspace{.2cm} \footnotesize{Source EXO}}


%---------------------------------------------------------------------------


