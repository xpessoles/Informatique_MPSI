
%\maketitle{}
\begin{enumerate}
\item  \textbf{Lisez attentivement  tout l'énoncé
    avant de commencer.}
\item Commencez la séance en créant un dossier au nom du TP dans le répertoire dédié à l'informatique de votre compte. 
\item Après la séance, vous devez rédiger un compte-rendu de TP et
l'envoyer au format électronique à votre enseignant.
\item Vous rendrez un compte-rendu sous forme d'un fichier d'extension \texttt{.py}, ainsi que des images au format PNG, en respectant exactement les spécifications données plus bas. 
\item Ce TP est à faire en binôme, vous ne rendrez donc qu'un  compte-rendu pour deux.
\item Ayez toujours un crayon et un papier sous la main. Quand vous réfléchissez à une question, utilisez les !
\item Vous devez être autonome. Ainsi, avant de poser une question à l'enseignant, merci de commencer par :
\begin{itemize}
  \item relire l'énoncé du TP (beaucoup de réponses se trouvent dedans) ;
  \item relire les passages du cours\footnote{Dans le cas fort 
improbable où vous ne vous en souviendriez pas.} relatifs à votre problème ;
  \item effectuer une recherche dans l'aide disponible sur votre ordinateur (ou sur internet) concernant votre question.
\end{itemize}
  Il est alors raisonnable d'appeler votre enseignant pour lui demander des explications ou une confirmation !
\end{enumerate}

Le but de ce TP est d'apprendre à représenter des fonctions avec Python, en prenant comme exemple l'approximation d'un signal par les séries de Fourier (un cadre déjà vu en physique).

\section*{Instructions de rendu}
Attention : suivez précisément ces instructions. 
Votre fichier portera un nom du type 
\begin{center}
  \texttt{tp05\_durif\_kleim.py},
\end{center}
 où les noms de vos enseignants sont à remplacer par ceux des membres du binôme. Le nom de ce 
fichier ne devra comporter ni espace, ni accent, ni apostrophe, ni majuscule.
Dans ce fichier, vous respecterez les consignes suivantes.
\begin{itemize}
  \item \'Ecrivez d'abord en commentaires (ligne débutant par \#), le titre du TP, les noms et prénoms des étudiants du groupe.
  \item Commencez chaque question par son numéro écrit en commentaires.
  \item Les questions demandant une réponse écrite seront rédigées en commentaires.
  \item Les questions demandant une réponse sous forme de fonction ou de script respecteront pointilleusement les noms de variables et de fonctions demandés.
\end{itemize}
Les figures demandées porteront toutes un nom du types \texttt{tp05\_durif\_kleim\_num.png}, où les noms de vos enseignants sont à remplacer par ceux des membres du binôme et où
\begin{itemize}
  \item \texttt{num} vaut \texttt{q04} pour la question~\ref{tp05:qu:sin2} ;
  \item \texttt{num} vaut \texttt{q05} pour la question~\ref{tp05:qu:transitoire} ;
  \item \texttt{num} vaut \texttt{q08} pour la question~\ref{tp05:qu:creneau} ;
  \item \texttt{num} vaut \texttt{q11} pour la question~\ref{tp05:qu:triangle} (question facultative).
\end{itemize}

\section{Tracé d'une fonction simple}

En utilisant \python{} on peut tracer de nombreux types de graphiques. 
Nous allons utiliser une bibliothèque regroupant de (très) nombreuses fonctions de tracé : \pyv{matplotlib}.
En fait, cette bibliothèque est bien trop vaste, nous n'utiliserons que sa sous-bibliothèque \pyv{matplotlib.pyplot}.

Commencez par ouvrir votre IDE, puis créez un script nommé \pyv{tp05_ex_sin.py}. Recopiez dedans le script suivant. 
\begin{pyverbatim}
import matplotlib.pyplot as plt
from numpy import sin

n = 20
x = [k*10/n for k in range(n)]
y = [sin(t) for t in x]

plt.clf()
plt.plot(x,y,label='sin(x)')
plt.xlabel('x')
plt.legend(loc=0)
plt.title('Tracé du sinus sur [0,10]')
plt.savefig('tp05_ex_sin.png')
\end{pyverbatim}
\'Exécutez ce script et vérifiez que la figure créée est en tout point semblable à celle présente sur le site de classe. 

\medskip{}

\question{} Quel est le type de \pyv{x} ?

\medskip{}

\question{} Comment \python{} représente-t-il graphiquement une fonction ? 

\medskip{}

\question{} Où le tracé s'arrête-t-il ? Pourquoi ? 

\medskip{}

On rappelle que l'on peut simplement créer une liste d'abscisses en utilisant la fonction \pyv{linspace} de la bibliothèque de calcul \pyv{numpy}. 

Chargez cette fonction dans l'interpréteur interactif par la commande 
\begin{pyverbatim}
from numpy import linspace
\end{pyverbatim}
puis consultez son manuel par la commande \pyv{help(linspace)}. 

\medskip{}

\question{}\label{tp05:qu:sin2} Écrire une fonction \pyv{ex_sin(nom_de_fichier)} permettant tracer de manière plus appropriée la courbe de l'exemple précédent et qui enregistre l'image produite dans le fichier \pyv{nom_de_fichier}. 
Vous produirez alors une image, que vous enverrez à votre enseignant. 

\emph{Indication :} Attention au type de \pyv{nom_de_fichier} ! Notamment, on supposera que l'extension du fichier est déjà présente dans \pyv{nom_de_fichier}.

\medskip{}

\question{}\label{tp05:qu:transitoire} \'Ecrire une fonction \pyv{transitoire(A,nom_de_fichier)} qui enregistre dans le fichier \pyv{nom_de_fichier} le graphe des fonctions 
\begin{equation*}
  t\mapsto A\p{1-\text{e}^{-\dfrac{t}{\tau}}}
\end{equation*}
sur $[0,10]$, pour chaque $\tau\in\left\{\dfrac{1}{2};1;2;4;8\right\}$. 
Vous produirez une image (vous choisirez $A$), que vous enverrez à votre enseignant. 

\emph{Indication :} Attention au type de \pyv{nom_de_fichier} !

\section{Synthèse de Fourier}

On s'intéresse maintenant à l'approximation d'un signal périodique par des fonctions trigonométriques. 
On considère sur $\R$ la fonction créneau, impaire et périodique de période 2, définie par $C(1) = 0$ et sur $\left]0,1\right[$ par 
\begin{equation*}
  C:t\mapsto 1.
\end{equation*}
Soit aussi la fonction triangle, définie sur $\R$, paire et périodique de période $2$, définie sur $[0,1]$ par  
\begin{equation*}
 T:t\mapsto 1-2t.
\end{equation*}
Ces deux fonctions sont représentées sur la figure~\ref{TP05:fig:creneau_triangle}.
% \begin{minipage}{0.5\textwidth}
\begin{figure}[!h]
  \begin{center}
    \begin{tikzpicture}
      \draw[->] (-2.5,0) -- (2.5,0);
      \draw[->] (0,-1.5) -- (0,1.5);
      \draw (-0.1,1) node[anchor=east] {$1$} -- (0.1,1);
      \draw (0.1,-1) node[anchor=west] {$-1$} -- (-0.1,-1);
      \draw (1,0) node[anchor = north] {$1$};
      \draw (-1,0) node[anchor = north] {$-1$};
      \draw[-(,blue,thick] (-2.5,-1) -- (-2,-1);
      \draw[)-(,blue,thick] (-2,1) -- (-1,1);
      \draw[)-(,blue,thick] (-1,-1) -- (0,-1);
      \draw[)-(,blue,thick] (0,1) -- (1,1);
      \draw[)-(,blue,thick] (1,-1) -- (2,-1);
      \draw[)-,blue,thick] (2,1) -- (2.5,1);
      \draw[blue,thick] (-2,0) node {\textbullet};
      \draw[blue,thick] (-1,0) node {\textbullet};
      \draw[blue,thick] (0,0) node {\textbullet};
      \draw[blue,thick] (1,0) node {\textbullet};
      \draw[blue,thick] (2,0) node {\textbullet};
      \draw[->] (4.5,0) -- (9.5,0) ;
      \draw[->] (7,-1.5) -- (7,1.5) ;
      \draw (6.9,1) node[anchor=east] {$1~$} -- (7.1,1);
      \draw (7.1,-1) node[anchor=west] {$-1$} -- (6.9,-1);
      \draw (8,0.1) node[anchor = south] {$1$} -- (8,-0.1);
      \draw (6,0.1) node[anchor = south] {$-1$} -- (6,-0.1);
      \draw[red,thick] (4.5,0) -- (5,1) -- (6,-1) -- (7,1) -- (8,-1) -- (9,1) -- (9.5,0);
    \end{tikzpicture}
    \caption{Signal créneau et signal triangulaire.}
    \label{TP05:fig:creneau_triangle}
  \end{center}
\end{figure}
% \end{minipage}


On peut montrer que, en tout réel $t$ où $C$ est continue, 
\begin{equation*}
  C(t) = \dfrac{4}{\pi}\sum_{p=0}^{+\infty} \dfrac{1}{2p+1} \sin((2p+1)\pi t).
\end{equation*}
De même, pour tout $t\in\R$, 
\begin{equation*}
  T(t) = \dfrac{8}{\pi^2}\sum_{p=0}^{+\infty} \dfrac{1}{(2p+1)^2} \cos((2p+1)\pi t).
\end{equation*}
Ce symbole $\displaystyle \sum_{p=0}^{+\infty}$ doit être vu comme une limite et sera défini dans le cours de mathématiques. 
On approche alors, pour tout $t\in\R$, $C(t)$ et $T(t)$ respectivement par les sommes 
\begin{equation*}
  C_n(t) = \dfrac{4}{\pi}\sum_{p=0}^{n} \dfrac{1}{2p+1} \sin((2p+1)\pi t)
\end{equation*}
et 
\begin{equation*}
   T_n(t) = \dfrac{8}{\pi^2}\sum_{p=0}^{n} \dfrac{1}{(2p+1)^2} \cos((2p+1)\pi t).
\end{equation*}
Les physiciens appellent le terme en $p=0$ le \emph{fondamental} et les autres termes les \emph{harmoniques}. 


\medskip{}

\question{} \'Ecrire une fonction \pyv{creneau(t)} renvoyant la valeur de $C(\texttt{t})$ (pour \pyv{t} réel). 

\emph{Indication :} on pourra considérer la parité de la partie entière de \pyv{t}.


\medskip{}

\question{} \'Ecrire une fonction \pyv{sp_creneau(n,t)} renvoyant la valeur de $C_{\texttt{n}}(\texttt{t})$ (pour \pyv{n} entier et \pyv{t} réel). 


\medskip{}

\question{}\label{tp05:qu:creneau} \'Ecrire une fonction \pyv{fourier_creneau(nom_de_fichier)} ne renvoyant rien 
et enregistrant dans \pyv{nom_de_fichier} le graphe sur l'intervalle $[0,4]$ de $C$, celui 
de son fondamental (\emph{i.e}, $C_0$), ainsi que ceux de $C_3$, $C_5$ et $C_{100}$.
Vous produirez une image, que vous enverrez à votre enseignant. 

\bigskip

\emph{Pour les plus rapides, voici des questions supplémentaires.}

\bigskip



\question{} \'Ecrire une fonction \pyv{triangle(t)} renvoyant la valeur de $T(\texttt{t})$ (pour 
\pyv{t} réel).


\medskip{}

\question{} \'Ecrire une fonction \pyv{sp_triangle(n,t)} renvoyant la valeur de $T_{\texttt{n}}(\texttt{t})$ (pour \pyv{n} entier et \pyv{t} réel). 


\medskip{}

\question{}\label{tp05:qu:triangle} \'Ecrire une fonction \pyv{fourier_triangle(nom_de_fichier)} ne renvoyant rien et enregistrant dans \pyv{nom_de_fichier} le graphe sur l'intervalle $[0,4]$ de $T$, celui de son fondamental (\emph{i.e}, $T_0$), ainsi que ceux de $T_3$, $T_5$ et $T_{100}$.
Vous produirez une image, que vous enverrez à votre enseignant. 
