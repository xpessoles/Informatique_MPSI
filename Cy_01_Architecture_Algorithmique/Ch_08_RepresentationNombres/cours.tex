\section{Base de numération}
\subsection{Rappel de CP : la base dix}
Chiffre : symbole utilisé pour représenter certains entiers.

Les chiffres «usuels» : $0$, $1$, $2$, $3$, $4$, $5$, $6$, $7$, $8$,
$9$.

%%\clearslide{}
Le nombre dix joue un rôle particulier.
\begin{itemize}
\item C'est le plus petit entier naturel non représentable uniquement par un chiffre.
\item Pour compter des objets en grand nombre, on les regroupe par paquets de dix.
\end{itemize}

Exemple : Pour compter $||||||||||||||||||||||||$, on obtient
\begin{equation*}
 ||||||||||\quad ||||||||||\quad |||| 
\end{equation*}
Deux paquets (deux dizaines), reste quatre unités.

Quand il y a trop de dizaines, on regroupe les dizaines par paquets de
dix (centaines), les centaines par paquets de dix (milliers), etc.

\subsection{Numération de position en base 10}


On décompose un entier en dizaines, centaines, milliers, etc. 
L'essentiel est alors qu'il y ait strictement moins de dix éléments dans chaque type de paquet. Ce nombre d'éléments peut être représenté par un chiffre.
On écrit alors tous les chiffres à la suite. À gauche, on place les \emph{chiffres de poids fort} (gros paquets). À droite, les \emph{chiffres de poids faible}.

Ainsi $2735$ représente deux milliers plus sept centaines plus trois
dizaines plus cinq unités.

De manière générale, avec $B=10$ et $n\in\N$,
$$\displaystyle\underline{a_{n}a_{n-1}\ldots a_{1}a_{0}}_{~B} =
\sum_{k=0}^{n}a_{k}B^{k},~ \textrm{ et }~\forall k \in \iif{0;n},~ a_k\in\ii{0;B}.$$ 

%\clearslide{}

\subsection{Pourquoi dix?}

Pourquoi regrouper par dix pas plutôt par deux? ou trois? ou six? ou
huit?

\begin{itemize}
\item Raison anthropomorphique (dix doigts) et poids de l'histoire.
\item À peu près aucune raison mathématique.
\end{itemize}

On peut choisir une autre base.

\begin{itemize}
\item Deux: Amérique du Sud et Océanie.
\item Cinq: Afrique, Romains et Maya (partiellement).
\item Six: Papouasie Nouvelle-Guinée.
\item Huit: certains dialectes amérindiens (Pame, Mexique; Yuki,
  Californie), proposition de Charles XII de Suède.
\item Douze: Népal, Europe.
\item Vingt: Bhoutan, Aztèques, Maya, Gaulois (?), Basques (?).
\item Soixante: Babyloniens, Indiens et Arabes (trigo)
\end{itemize}
(source: Wikipédia, article \textit{Numération})

\`A chaque fois, le principe est identique : on change juste $B$ dans l'écriture précédente.

%\clearslide{}
%\subsection{Un premier exemple : la base huit (octale)}
%
%En pratique, on ne l'utilisera pas ...
%
%On a besoin de huit symboles pour représenter les huit chiffres (représentant les nombres de zéro inclus à huit exclu). On prendra naturellement $0$, \ldots{}, $7$.
%Pour compter, on regroupe les unités par huitaine puis par
%  huitaines de huitaines, etc. On note les nombres sur le même principe que l'écriture décimale :
%$\oct{2735}$ représente cinq unités plus trois huitaines
%plus sept huitaines de huitaines, plus deux huitaines de huitaines de
%huitaines (soit $1501$, exprimé ici en base 10 !).
%
%De manière générale, avec $B=8$ et $n\in\N$,
%$$\displaystyle\underline{a_{n}a_{n-1}\ldots a_{1}a_{0}}_{~B} =
%\sum_{k=0}^{n}a_{k}B^{k},~ \textrm{ et }~\forall k \in \iif{0;n},~ a_k\in\ii{0;B}.$$ 
%
%Vous remarquerez la forte similitude avec la formule précédente ...
%
%Comptons en octal:
%
%\begin{minipage}[t]{0.18\linewidth}
%\begin{align*}
%\oct{0} &= 0\\
%\oct{1} &= 1\\
%\oct{2} &= 2\\
%\oct{3} &= 3\\
%\oct{4} &= 4\\
%\oct{5} &= 5\\
%\oct{6} &= 6\\
%\oct{7} &= 7\\
%\end{align*}
%\end{minipage}
%\begin{minipage}[t]{0.18\linewidth}
%\begin{align*}
%\oct{10} &= 8\\
%\oct{11} &= 9\\
%\oct{12} &= 10\\
%\oct{13} &= 11\\
%\oct{14} &= 12\\
%\oct{15} &= 13\\
%\oct{16} &= 14\\
%\oct{17} &= 15\\
%\end{align*}
%\end{minipage}
%\begin{minipage}[t]{0.18\linewidth}
%\begin{align*}  
%\oct{20} &= 16\\
%\oct{21} &= 17\\
%\oct{22} &= 18\\
%\oct{23} &= 19\\
%\oct{24} &= 20\\
%\oct{25} &= 21\\
%\oct{26} &= 22\\
%\oct{27} &= 23\\
%\end{align*}
%\end{minipage}
%\begin{minipage}[t]{0.18\linewidth}
%\begin{align*}
%\oct{30} &= 24\\
%\oct{40} &= 32\\
%\oct{50} &= 40\\
%\oct{60} &= 48\\
%\oct{70} &= 56\\
%\end{align*}
%\end{minipage}
%\begin{minipage}[t]{0.18\linewidth}
%  \begin{align*}
%\oct{100} &= 64\\
%\oct{200} &= 128\\
%\oct{1\,000} &= 512\\
%\oct{10\,000} &= 4096\\
%\oct{100\,000} &= 32768\\
%\end{align*}
%\end{minipage}
%
%%\clearslide{}
%\begin{minipage}[c]{.48\linewidth}
%\begin{center}
%  \begin{tabular}{rrrrrrrrr}
\toprule
\textbf{\texttt{+}}& \textbf{\texttt{0}}& \textbf{\texttt{1}}& \textbf{\texttt{2}}& \textbf{\texttt{3}}& \textbf{\texttt{4}}& \textbf{\texttt{5}}& \textbf{\texttt{6}}& \textbf{\texttt{7}}\\ \midrule
\textbf{\texttt{0}} & \texttt{0} & \texttt{1} & \texttt{2} & \texttt{3} & \texttt{4} & \texttt{5} & \texttt{6} & \texttt{7}\\
\textbf{\texttt{1}} & \texttt{1} & \texttt{2} & \texttt{3} & \texttt{4} & \texttt{5} & \texttt{6} & \texttt{7} & \texttt{10}\\
\textbf{\texttt{2}} & \texttt{2} & \texttt{3} & \texttt{4} & \texttt{5} & \texttt{6} & \texttt{7} & \texttt{10} & \texttt{11}\\
\textbf{\texttt{3}} & \texttt{3} & \texttt{4} & \texttt{5} & \texttt{6} & \texttt{7} & \texttt{10} & \texttt{11} & \texttt{12}\\
\textbf{\texttt{4}} & \texttt{4} & \texttt{5} & \texttt{6} & \texttt{7} & \texttt{10} & \texttt{11} & \texttt{12} & \texttt{13}\\
\textbf{\texttt{5}} & \texttt{5} & \texttt{6} & \texttt{7} & \texttt{10} & \texttt{11} & \texttt{12} & \texttt{13} & \texttt{14}\\
\textbf{\texttt{6}} & \texttt{6} & \texttt{7} & \texttt{10} & \texttt{11} & \texttt{12} & \texttt{13} & \texttt{14} & \texttt{15}\\
\textbf{\texttt{7}} & \texttt{7} & \texttt{10} & \texttt{11} & \texttt{12} & \texttt{13} & \texttt{14} & \texttt{15} & \texttt{16}\\
\bottomrule
\end{tabular}
\\
%  Table d'addition en octal
%\end{center}
%\end{minipage} \hfill
%\begin{minipage}[c]{.48\linewidth}
%\begin{center}
%  \begin{tabular}{rrrrrrrrr}
\toprule
\textbf{\texttt{*}}& \textbf{\texttt{0}}& \textbf{\texttt{1}}& \textbf{\texttt{2}}& \textbf{\texttt{3}}& \textbf{\texttt{4}}& \textbf{\texttt{5}}& \textbf{\texttt{6}}& \textbf{\texttt{7}}\\ \midrule
\textbf{\texttt{0}} & \texttt{0} & \texttt{0} & \texttt{0} & \texttt{0} & \texttt{0} & \texttt{0} & \texttt{0} & \texttt{0}\\
\textbf{\texttt{1}} & \texttt{0} & \texttt{1} & \texttt{2} & \texttt{3} & \texttt{4} & \texttt{5} & \texttt{6} & \texttt{7}\\
\textbf{\texttt{2}} & \texttt{0} & \texttt{2} & \texttt{4} & \texttt{6} & \texttt{10} & \texttt{12} & \texttt{14} & \texttt{16}\\
\textbf{\texttt{3}} & \texttt{0} & \texttt{3} & \texttt{6} & \texttt{11} & \texttt{14} & \texttt{17} & \texttt{22} & \texttt{25}\\
\textbf{\texttt{4}} & \texttt{0} & \texttt{4} & \texttt{10} & \texttt{14} & \texttt{20} & \texttt{24} & \texttt{30} & \texttt{34}\\
\textbf{\texttt{5}} & \texttt{0} & \texttt{5} & \texttt{12} & \texttt{17} & \texttt{24} & \texttt{31} & \texttt{36} & \texttt{43}\\
\textbf{\texttt{6}} & \texttt{0} & \texttt{6} & \texttt{14} & \texttt{22} & \texttt{30} & \texttt{36} & \texttt{44} & \texttt{52}\\
\textbf{\texttt{7}} & \texttt{0} & \texttt{7} & \texttt{16} & \texttt{25} & \texttt{34} & \texttt{43} & \texttt{52} & \texttt{61}\\
\bottomrule
\end{tabular}
\\
%  Table de multiplication en octal
%\end{center}
%\end{minipage}
%
%%\clearslide{}
%Comment faire une addition de nombres à plusieurs chiffres?
%
%Exactement comme en base 10, mais en utilisant la table d'addition de
%la base 8.
%
%Pour la multiplication: aussi.

\subsection{La base seize (hexadécimale)}

On reprend le même principe que précédemment, avec $B = 16$ (on forme des paquets de $16$ etc.). Mais cette fois on manque de chiffres pour représenter les nombres de zéro inclus à
seize exclu (il en manque six).

On rajoute de nouveaux \og chiffres \fg{}:
\begin{description}
\item[a] dix
\item[b] onze
\item[c] douze
\item[d] treize
\item[e] quatorze
\item[f] quinze
\end{description}
On peut alors se mettre à compter en hexadécimal !
%\clearslide{}
\vskip-1pt
\noindent\begin{minipage}[t]{0.18\linewidth}
\footnotesize{}
\begin{align*}
\hex{0} &= 0\\
\hex{1} &= 1\\
% \hex{2} &= 2\\
&\vdots\\
% \hex{4} &= 4\\
% \hex{5} &= 5\\
% \hex{6} &= 6\\
% \hex{7} &= 7\\
% \hex{8} &= 8\\
\hex{9} &= 9\\
\hex{a} &= 10\\
\hex{b} &= 11\\
\hex{c} &= 12\\
\hex{d} &= 13\\
\hex{e} &= 14\\
\hex{f} &= 15\\
\end{align*}
\end{minipage}
\begin{minipage}[t]{0.18\linewidth}
\footnotesize{}
\begin{align*}
\hex{10} &= 16\\
\hex{11} &= 17\\
&\vdots\\
\hex{19} &= 25\\
\hex{1a} &= 26\\
\hex{1b} &= 27\\
\hex{1c} &= 28\\
\hex{1d} &= 29\\
\hex{1e} &= 30\\
\hex{1f} &= 31\\
\end{align*}
\end{minipage}
\begin{minipage}[t]{0.18\linewidth}
\footnotesize{}
\begin{align*}
\hex{20} &= 32\\
\hex{21} &= 33\\
&\vdots\\
\hex{29} &= 41\\
\hex{2a} &= 42\\
\hex{2b} &= 43\\
\hex{2c} &= 44\\
\hex{2d} &= 45\\
\hex{2e} &= 46\\
\hex{2f} &= 47\\
\end{align*}
\end{minipage}
\begin{minipage}[t]{0.18\linewidth}
\footnotesize{}
\begin{align*}
\hex{30} &= 48\\
\hex{40} &= 64\\
&\vdots\\
\hex{90} &= 144\\
\hex{a0} &= 160\\
\hex{b0} &= 176\\
\hex{c0} &= 192\\
\hex{d0} &= 208\\
\hex{e0} &= 224\\
\hex{f0} &= 240\\
\end{align*}
\end{minipage}
\begin{minipage}[t]{0.18\linewidth}
\footnotesize{}
\begin{align*}
\hex{100} &= 256\\
\hex{200} &= 512\\
\hex{1\,000} &= 4096\\
\hex{10\,000} &= 65536\\
\end{align*}
\end{minipage}
%\clearslide{}

%
%\begin{center}
%  \tiny{}
%  \begin{tabular}{rrrrrrrrrrrrrrrrr}
\toprule
\textbf{\texttt{+}}& \textbf{\texttt{0}}& \textbf{\texttt{1}}& \textbf{\texttt{2}}& \textbf{\texttt{3}}& \textbf{\texttt{4}}& \textbf{\texttt{5}}& \textbf{\texttt{6}}& \textbf{\texttt{7}}& \textbf{\texttt{8}}& \textbf{\texttt{9}}& \textbf{\texttt{A}}& \textbf{\texttt{B}}& \textbf{\texttt{C}}& \textbf{\texttt{D}}& \textbf{\texttt{E}}& \textbf{\texttt{F}}\\ \midrule
\textbf{\texttt{0}} & \texttt{0} & \texttt{1} & \texttt{2} & \texttt{3} & \texttt{4} & \texttt{5} & \texttt{6} & \texttt{7} & \texttt{8} & \texttt{9} & \texttt{A} & \texttt{B} & \texttt{C} & \texttt{D} & \texttt{E} & \texttt{F}\\
\textbf{\texttt{1}} & \texttt{1} & \texttt{2} & \texttt{3} & \texttt{4} & \texttt{5} & \texttt{6} & \texttt{7} & \texttt{8} & \texttt{9} & \texttt{A} & \texttt{B} & \texttt{C} & \texttt{D} & \texttt{E} & \texttt{F} & \texttt{10}\\
\textbf{\texttt{2}} & \texttt{2} & \texttt{3} & \texttt{4} & \texttt{5} & \texttt{6} & \texttt{7} & \texttt{8} & \texttt{9} & \texttt{A} & \texttt{B} & \texttt{C} & \texttt{D} & \texttt{E} & \texttt{F} & \texttt{10} & \texttt{11}\\
\textbf{\texttt{3}} & \texttt{3} & \texttt{4} & \texttt{5} & \texttt{6} & \texttt{7} & \texttt{8} & \texttt{9} & \texttt{A} & \texttt{B} & \texttt{C} & \texttt{D} & \texttt{E} & \texttt{F} & \texttt{10} & \texttt{11} & \texttt{12}\\
\textbf{\texttt{4}} & \texttt{4} & \texttt{5} & \texttt{6} & \texttt{7} & \texttt{8} & \texttt{9} & \texttt{A} & \texttt{B} & \texttt{C} & \texttt{D} & \texttt{E} & \texttt{F} & \texttt{10} & \texttt{11} & \texttt{12} & \texttt{13}\\
\textbf{\texttt{5}} & \texttt{5} & \texttt{6} & \texttt{7} & \texttt{8} & \texttt{9} & \texttt{A} & \texttt{B} & \texttt{C} & \texttt{D} & \texttt{E} & \texttt{F} & \texttt{10} & \texttt{11} & \texttt{12} & \texttt{13} & \texttt{14}\\
\textbf{\texttt{6}} & \texttt{6} & \texttt{7} & \texttt{8} & \texttt{9} & \texttt{A} & \texttt{B} & \texttt{C} & \texttt{D} & \texttt{E} & \texttt{F} & \texttt{10} & \texttt{11} & \texttt{12} & \texttt{13} & \texttt{14} & \texttt{15}\\
\textbf{\texttt{7}} & \texttt{7} & \texttt{8} & \texttt{9} & \texttt{A} & \texttt{B} & \texttt{C} & \texttt{D} & \texttt{E} & \texttt{F} & \texttt{10} & \texttt{11} & \texttt{12} & \texttt{13} & \texttt{14} & \texttt{15} & \texttt{16}\\
\textbf{\texttt{8}} & \texttt{8} & \texttt{9} & \texttt{A} & \texttt{B} & \texttt{C} & \texttt{D} & \texttt{E} & \texttt{F} & \texttt{10} & \texttt{11} & \texttt{12} & \texttt{13} & \texttt{14} & \texttt{15} & \texttt{16} & \texttt{17}\\
\textbf{\texttt{9}} & \texttt{9} & \texttt{A} & \texttt{B} & \texttt{C} & \texttt{D} & \texttt{E} & \texttt{F} & \texttt{10} & \texttt{11} & \texttt{12} & \texttt{13} & \texttt{14} & \texttt{15} & \texttt{16} & \texttt{17} & \texttt{18}\\
\textbf{\texttt{A}} & \texttt{A} & \texttt{B} & \texttt{C} & \texttt{D} & \texttt{E} & \texttt{F} & \texttt{10} & \texttt{11} & \texttt{12} & \texttt{13} & \texttt{14} & \texttt{15} & \texttt{16} & \texttt{17} & \texttt{18} & \texttt{19}\\
\textbf{\texttt{B}} & \texttt{B} & \texttt{C} & \texttt{D} & \texttt{E} & \texttt{F} & \texttt{10} & \texttt{11} & \texttt{12} & \texttt{13} & \texttt{14} & \texttt{15} & \texttt{16} & \texttt{17} & \texttt{18} & \texttt{19} & \texttt{1A}\\
\textbf{\texttt{C}} & \texttt{C} & \texttt{D} & \texttt{E} & \texttt{F} & \texttt{10} & \texttt{11} & \texttt{12} & \texttt{13} & \texttt{14} & \texttt{15} & \texttt{16} & \texttt{17} & \texttt{18} & \texttt{19} & \texttt{1A} & \texttt{1B}\\
\textbf{\texttt{D}} & \texttt{D} & \texttt{E} & \texttt{F} & \texttt{10} & \texttt{11} & \texttt{12} & \texttt{13} & \texttt{14} & \texttt{15} & \texttt{16} & \texttt{17} & \texttt{18} & \texttt{19} & \texttt{1A} & \texttt{1B} & \texttt{1C}\\
\textbf{\texttt{E}} & \texttt{E} & \texttt{F} & \texttt{10} & \texttt{11} & \texttt{12} & \texttt{13} & \texttt{14} & \texttt{15} & \texttt{16} & \texttt{17} & \texttt{18} & \texttt{19} & \texttt{1A} & \texttt{1B} & \texttt{1C} & \texttt{1D}\\
\textbf{\texttt{F}} & \texttt{F} & \texttt{10} & \texttt{11} & \texttt{12} & \texttt{13} & \texttt{14} & \texttt{15} & \texttt{16} & \texttt{17} & \texttt{18} & \texttt{19} & \texttt{1A} & \texttt{1B} & \texttt{1C} & \texttt{1D} & \texttt{1E}\\
\bottomrule
\end{tabular}
\\
%  Table d'addition en hexadécimal
%\end{center}
%
%%\clearslide{}
%\begin{center}
%  \tiny{}
%  \begin{tabular}{rrrrrrrrrrrrrrrrr}
\toprule
\textbf{\texttt{*}}& \textbf{\texttt{0}}& \textbf{\texttt{1}}& \textbf{\texttt{2}}& \textbf{\texttt{3}}& \textbf{\texttt{4}}& \textbf{\texttt{5}}& \textbf{\texttt{6}}& \textbf{\texttt{7}}& \textbf{\texttt{8}}& \textbf{\texttt{9}}& \textbf{\texttt{A}}& \textbf{\texttt{B}}& \textbf{\texttt{C}}& \textbf{\texttt{D}}& \textbf{\texttt{E}}& \textbf{\texttt{F}}\\ \midrule
\textbf{\texttt{0}} & \texttt{0} & \texttt{0} & \texttt{0} & \texttt{0} & \texttt{0} & \texttt{0} & \texttt{0} & \texttt{0} & \texttt{0} & \texttt{0} & \texttt{0} & \texttt{0} & \texttt{0} & \texttt{0} & \texttt{0} & \texttt{0}\\
\textbf{\texttt{1}} & \texttt{0} & \texttt{1} & \texttt{2} & \texttt{3} & \texttt{4} & \texttt{5} & \texttt{6} & \texttt{7} & \texttt{8} & \texttt{9} & \texttt{A} & \texttt{B} & \texttt{C} & \texttt{D} & \texttt{E} & \texttt{F}\\
\textbf{\texttt{2}} & \texttt{0} & \texttt{2} & \texttt{4} & \texttt{6} & \texttt{8} & \texttt{A} & \texttt{C} & \texttt{E} & \texttt{10} & \texttt{12} & \texttt{14} & \texttt{16} & \texttt{18} & \texttt{1A} & \texttt{1C} & \texttt{1E}\\
\textbf{\texttt{3}} & \texttt{0} & \texttt{3} & \texttt{6} & \texttt{9} & \texttt{C} & \texttt{F} & \texttt{12} & \texttt{15} & \texttt{18} & \texttt{1B} & \texttt{1E} & \texttt{21} & \texttt{24} & \texttt{27} & \texttt{2A} & \texttt{2D}\\
\textbf{\texttt{4}} & \texttt{0} & \texttt{4} & \texttt{8} & \texttt{C} & \texttt{10} & \texttt{14} & \texttt{18} & \texttt{1C} & \texttt{20} & \texttt{24} & \texttt{28} & \texttt{2C} & \texttt{30} & \texttt{34} & \texttt{38} & \texttt{3C}\\
\textbf{\texttt{5}} & \texttt{0} & \texttt{5} & \texttt{A} & \texttt{F} & \texttt{14} & \texttt{19} & \texttt{1E} & \texttt{23} & \texttt{28} & \texttt{2D} & \texttt{32} & \texttt{37} & \texttt{3C} & \texttt{41} & \texttt{46} & \texttt{4B}\\
\textbf{\texttt{6}} & \texttt{0} & \texttt{6} & \texttt{C} & \texttt{12} & \texttt{18} & \texttt{1E} & \texttt{24} & \texttt{2A} & \texttt{30} & \texttt{36} & \texttt{3C} & \texttt{42} & \texttt{48} & \texttt{4E} & \texttt{54} & \texttt{5A}\\
\textbf{\texttt{7}} & \texttt{0} & \texttt{7} & \texttt{E} & \texttt{15} & \texttt{1C} & \texttt{23} & \texttt{2A} & \texttt{31} & \texttt{38} & \texttt{3F} & \texttt{46} & \texttt{4D} & \texttt{54} & \texttt{5B} & \texttt{62} & \texttt{69}\\
\textbf{\texttt{8}} & \texttt{0} & \texttt{8} & \texttt{10} & \texttt{18} & \texttt{20} & \texttt{28} & \texttt{30} & \texttt{38} & \texttt{40} & \texttt{48} & \texttt{50} & \texttt{58} & \texttt{60} & \texttt{68} & \texttt{70} & \texttt{78}\\
\textbf{\texttt{9}} & \texttt{0} & \texttt{9} & \texttt{12} & \texttt{1B} & \texttt{24} & \texttt{2D} & \texttt{36} & \texttt{3F} & \texttt{48} & \texttt{51} & \texttt{5A} & \texttt{63} & \texttt{6C} & \texttt{75} & \texttt{7E} & \texttt{87}\\
\textbf{\texttt{A}} & \texttt{0} & \texttt{A} & \texttt{14} & \texttt{1E} & \texttt{28} & \texttt{32} & \texttt{3C} & \texttt{46} & \texttt{50} & \texttt{5A} & \texttt{64} & \texttt{6E} & \texttt{78} & \texttt{82} & \texttt{8C} & \texttt{96}\\
\textbf{\texttt{B}} & \texttt{0} & \texttt{B} & \texttt{16} & \texttt{21} & \texttt{2C} & \texttt{37} & \texttt{42} & \texttt{4D} & \texttt{58} & \texttt{63} & \texttt{6E} & \texttt{79} & \texttt{84} & \texttt{8F} & \texttt{9A} & \texttt{A5}\\
\textbf{\texttt{C}} & \texttt{0} & \texttt{C} & \texttt{18} & \texttt{24} & \texttt{30} & \texttt{3C} & \texttt{48} & \texttt{54} & \texttt{60} & \texttt{6C} & \texttt{78} & \texttt{84} & \texttt{90} & \texttt{9C} & \texttt{A8} & \texttt{B4}\\
\textbf{\texttt{D}} & \texttt{0} & \texttt{D} & \texttt{1A} & \texttt{27} & \texttt{34} & \texttt{41} & \texttt{4E} & \texttt{5B} & \texttt{68} & \texttt{75} & \texttt{82} & \texttt{8F} & \texttt{9C} & \texttt{A9} & \texttt{B6} & \texttt{C3}\\
\textbf{\texttt{E}} & \texttt{0} & \texttt{E} & \texttt{1C} & \texttt{2A} & \texttt{38} & \texttt{46} & \texttt{54} & \texttt{62} & \texttt{70} & \texttt{7E} & \texttt{8C} & \texttt{9A} & \texttt{A8} & \texttt{B6} & \texttt{C4} & \texttt{D2}\\
\textbf{\texttt{F}} & \texttt{0} & \texttt{F} & \texttt{1E} & \texttt{2D} & \texttt{3C} & \texttt{4B} & \texttt{5A} & \texttt{69} & \texttt{78} & \texttt{87} & \texttt{96} & \texttt{A5} & \texttt{B4} & \texttt{C3} & \texttt{D2} & \texttt{E1}\\
\bottomrule
\end{tabular}
\\
%  Table de multiplication en hexadécimal
%\end{center}


%\clearslide{}
\subsection{Enfin : la base deux !}

C'est toujours le même principe, mais avec $B=2$ ; on représente les nombres \og par paquets de deux \fg{}. 
Pas de problème pour les chiffres, nous n'en avons besoin que de deux. Par convention : $0$ et $1$.

\begin{minipage}[t]{0.18\linewidth}
\begin{align*}
\bin{0} &= 0\\
\bin{1} &= 1\\
\end{align*}
\end{minipage}
\begin{minipage}[t]{0.18\linewidth}
\begin{align*}
\bin{10} &= 2\\
\bin{11} &= 3\\
\end{align*}
\end{minipage}
\begin{minipage}[t]{0.18\linewidth}
\begin{align*}  
\bin{100} &= 4\\
\bin{101} &= 5\\
\bin{110} &= 6\\
\bin{111} &= 7\\
\end{align*}
\end{minipage}
\begin{minipage}[t]{0.18\linewidth}
\begin{align*}
\bin{1000} &= 8\\
\bin{1\,0000} &= 16\\
\bin{10\,0000} &= 32\\
\bin{100\,0000} &= 64\\
\bin{1000\,0000} &= 128\\
\bin{1\,0000\,0000} &= 256\\
\end{align*}
\end{minipage}



  \begin{minipage}{0.5\textwidth}
  \begin{center}
    \begin{tabular}{rrr}
\toprule
\textbf{\texttt{+}}& \textbf{\texttt{0}}& \textbf{\texttt{1}}\\ \midrule
\textbf{\texttt{0}} & \texttt{0} & \texttt{1}\\
\textbf{\texttt{1}} & \texttt{1} & \texttt{10}\\
\bottomrule
\end{tabular}
\\
  Table d'addition en binaire
  \end{center}
  \end{minipage}
  \begin{minipage}{0.5\textwidth}
  \begin{center}
    \begin{tabular}{rrr}
\toprule
\textbf{\texttt{*}}& \textbf{\texttt{0}}& \textbf{\texttt{1}}\\ \midrule
\textbf{\texttt{0}} & \texttt{0} & \texttt{0}\\
\textbf{\texttt{1}} & \texttt{0} & \texttt{1}\\
\bottomrule
\end{tabular}
\\
  Table de multiplication en binaire
  \end{center}
  \end{minipage}

%\clearslide{}
\subsubsection{Intérêts du binaire}
Les nombres binaires sont facilement représentables par un dispositif
  mécanique/électrique/électronique/optique/électromagnétique etc.
De plus, les tables d'opérations très simples et sont facilement calculables par un
  dispositif mécanique/électrique/électronique etc.

C'est le système utilisé pour représenter les nombres en interne dans
un ordinateur.

%\clearslide{}
\subsubsection{Écriture d'un naturel $p$ en binaire}

Donnons d'abord un algorithme par divisions successives.
Si $p = \underline{a_n\dots a_0}_{~2}$, alors 
$$ p = \sum_{k=0}^n a_k 2^k = 2\sum_{k=1}^n a_k2^{k-1} + a_0.$$
Ainsi, $a_0$ est le reste de la division euclidienne de $p$ par $2$ et, si $p\neq a_0$, $\underline{a_n\dots a_1}_{~2}$ est le quotient de la division euclidienne de $p$ par $2$.

On peut donc écrire la fonction suivante.



\begin{lstlisting}
def conv_b2(p):
 """Convertit l'entier p en base 2 (renvoie une chaîne)"""
    x = p
    s = ""
    while x > 1 :
        s = str(x%2) + s
        x = x // 2
    return str(x)+s

print('0='+conv_b2(0)+' et 1='+conv_b2(1)+' et 42='+conv_b2(42))
\end{lstlisting}



Cela renvoie alors :
\begin{quote}
\begin{pycode}
def conv_b2(p):
    x = p
    s = ""
    while x > 1 :
        s = str(x%2) + s
        x = x // 2
    return str(x)+s

print('0='+conv_b2(0)+' et 1='+conv_b2(1)+' et 42='+conv_b2(42))
\end{pycode}
\end{quote}


Voici une autre idée : calculer $2^{k}$ pour $k=0, \ldots$ jusqu'à avoir $2^{k} >
  p$. Alors, $p$ s'écrit sur $k$ bits et le bit de poids fort est
  $1$. Le reste des bits est donné par la représentation en binaire de
  $p-2^{k-1}$.


\subsubsection{Calcul d'entier représenté en binaire}

On veut calculer l'entier $p$, représenté par une suite de bits
$\underline{a_{n}a_{n-1}\ldots a_{1}a_{0}}_{2}$, \emph{i.e.}
$\displaystyle\sum_{k=0}^{n}a_{k}2^{k}$.

On peut effectuer le calcul naïvement, en pensant bien à calculer les puissances de proche en proche. 

\begin{lstlisting}
def calc_b2_naif(s):
    """Renvoie l'entier p représente en binaire par s"""
    p = 0
    x = 1 ## 2**0
    for i in range(len(s)):
        p = p+int(s[len(s)-i-1])*x
        x = 2*x
    return p

print(0==calc_b2_naif("0"))
print(1==calc_b2_naif("1"))
print(42==calc_b2_naif("101010"))
\end{lstlisting}


Cela renvoie alors :
\begin{quote}
\begin{pycode}
def calc_b2_naif(s):
    p = 0
    x = 1 ## 2**0
    for i in range(len(s)):
        # Invariant : p = s[len(s)-i:]
        p = p+int(s[len(s)-i-1])*x
        x = 2*x
    return p

print(0==calc_b2_naif("0"))
print("\n")
print(1==calc_b2_naif("1"))
print("\n")
print(42==calc_b2_naif("101010"))
\end{pycode}
\end{quote}

On peut faire mieux en mettant en {\oe}uvre l'algorithme de Horner. Il suffit de remarquer que 
\begin{equation*}
  p = \sum_{k=0}^{n}a_{k}2^{k} = a_0 + 2\p{\sum_{k=1}^{n}a_{k}2^{k-1}} = a_0 + 2\p{a_1+2\p{a_2 + 2\p{\ldots + 2a_n}}}
\end{equation*}

\begin{lstlisting}
def calc_b2_horner(s):
    """Renvoie l'entier p représente en binaire par s"""
    p = int(s[0])
    for i in range(1,len(s)):
        p = int(s[i])+2*p
    return p
    
print(0==calc_b2_naif("0"))
print(1==calc_b2_naif("1"))
print(42==calc_b2_naif("101010"))
\end{lstlisting}


Cela renvoie alors :
\begin{quote}
\begin{pycode}
def calc_b2_horner(s):
    p = int(s[0])
    for i in range(1,len(s)):
        p = int(s[i])+2*p
    return p
    
print(0==calc_b2_naif("0"))
print("\n")
print(1==calc_b2_naif("1"))
print("\n")
print(42==calc_b2_naif("101010"))
\end{pycode}
\end{quote}

\begin{rem}
\begin{enumerate}
\item S'ils sont bien mis en œuvre, ces algorithmes de conversion demandent
un temps de calcul de l'ordre de $n$ opérations pour un nombre de $n$ chiffres
(binaires ou décimaux), soit de l'ordre de $\log p$ opérations.
\item Il existe des algorithmes plus efficaces. Meilleure complexité
  connue: complexité d'une multiplication de nombres de $n$ chiffres,
  soit $O(n\log n \log \log n)$ [Knuth].
\item Peu importe la base dans laquelle vous faites vos calculs, ces
  algorithmes permettent de convertir entre la base $2$ et votre base
  habituelle. 
\end{enumerate}
\end{rem}

\section{Représentation des entiers sur ordinateur}
\subsection{Cadre}
Sur un ordinateur récent :
\begin{itemize}
\item[\textbullet] on travaille sur des mots-machine de $64$ bits ($8$ octets) ;
\item[\textbullet] les opérations d'addition et de multiplication d'entiers internes au
  processeur se font sur $64$ bits.
\end{itemize}

De manière générale, on s'intéressera au fonctionnement sur des
ordinateurs travaillant sur des mots de $n$ bits ($n\geq 2$), mais pour les
exemples, on prendra systématiquement $n=16$.

\subsection{Somme d'entiers naturels}
\sloppy

Sur un processeur $n$ bits, un registre du processeur a $n$ bits et peut
  représenter tout entier (naturel) de $\ii{0,2^{n}}$.

 Lorsqu'on effectue l'addition de deux registres $r_{1}$ et
  $r_{2}$ pour stocker le résultat dans $r_{3}$, le registre fait $n$
  bits : s'il y a une retenue, elle est perdue.\footnote{En fait une
    trace en est généralement gardée dans un autre registre du
    processeur.}

\begin{exemple}
  Après addition de $\bin{1111\,0000\,1111\,0000}$ et
  $\bin{0011\,0011\,0011\,0011}$ sur $16$ bits, le registre résultat contient :
  $\bin{0010\,0100\,0010\,0011}$.
\end{exemple}
   
\subsection{Entiers relatifs}

On veut maintenant pouvoir travailler avec des entiers relatifs et notamment les additionner et les soustraire !

\subsubsection{Avec signe et valeur absolue}

Première possibilité de codage d'un entier relatif sur $n$ bits : on utilise $n-1$ bits pour la valeur absolue et
$1$ bit pour le signe.
\begin{exemple}

\begin{minipage}{0.45\textwidth}
Représentons les entiers relatifs sur 3 bits avec cette méthode :
\begin{itemize}
\item On utilise $3-1=2 bits$ pour la valeur absolue et 1 bit pour le signe.
\item On représenta alors les entiers de $-3$ à $+3$ avec deux fois la représentation de 0.
\item Avec cette représentation l'addition est compliquée à mettre en oeuvre et incohérente avec la méthode habituelle :  $2+(-1)=\bin{010}+\bin{101}=\bin{111}$.
\end{itemize} 
\end{minipage}
\begin{minipage}{0.45\textwidth}
\begin{tabular}{|p{0.1\textwidth}|p{0.13\textwidth}|p{0.28\textwidth}|p{0.28\textwidth}|}
\hline 
Signe & Valeur absolue & Représentation binaire & Représentation décimale \\ 
\hline 
$0$ & $00$ & $\bin{000}$ & $+0$ \\ 
$0$ & $01$ & $\bin{001}$ & $+1$ \\ 
$0$ & $10$ & $\bin{010}$ & $+2$ \\ 
$0$ & $11$ & $\bin{011}$ & $+3$ \\ 
$1$ & $00$ & $\bin{100}$ & $-0$ \\ 
$1$ & $01$ & $\bin{101}$ & $-1$ \\ 
$1$ & $10$ & $\bin{110}$ & $-2$ \\ 
$1$ & $11$ & $\bin{111}$ & $-3$ \\ 
\hline 
\end{tabular} 
\end{minipage}

%\begin{itemize}
%\item[\textbullet] On représente $-4$ par $\bin{1000\,0000\,0000\,0100}$.
%\item[\textbullet] On représente $4$ par $\bin{0000\,0000\,0000\,0100}$.
%\end{itemize}  
\end{exemple}

%Mais cela a deux inconvénients majeurs.
%\begin{enumerate}
%\item On a deux représentations pour zéro ($\bin{1000\,0000\,0000\,0000}$
%  et $\bin{0000\,0000\,0000\,0000}$).
%\item L'addition est compliquée à mettre en {\oe}uvre, notamment par rapport à celle définie par les entiers naturels.
%\end{enumerate}

Ainsi, cette représentation n'est quasiment jamais utilisée pour les nombres
\emph{entiers} d'un processeur.
%\clearslide{}
\subsubsection{Complément à deux}

On va utiliser l'idée suivante. Remarquons que l'addition d'entiers naturels sur le processeur n'est pas correcte mais l'est modulo $2^{n}$. De plus, pour tout $p\in \Z$, $p\mmod 2^{n} \in \ii{0,2^{n}}$. 
  Ainsi, $p\mmod 2^{n}$ est représentable sur $n$ bits.
  
C'est donc la \emph{représentation en complément à deux} qui est le plus souvent utilisée : un entier relatif $p$ est représenté sur $n$ bits comme l'entier naturel $p\mmod 2^n$. 

Ainsi sur $n$ bits on représentera tous les entiers relatifs compris dans l'intervalle  $\ii{-2^{n-1},2^{n-1}}$

\begin{rem}
  L'addition d'entiers relatifs (on dit aussi \emph{signés}) sera correcte modulo $2^{n}$ et utilisera les mêmes circuits que l'addition d'entiers naturels.
\end{rem}

\begin{exemple} ~\\

\begin{minipage}{0.4\textwidth}
sur 3 bits :
\begin{itemize}
\item  2 est codé par $2\mmod 2^3=2=\bin{001}$ ;
\item -2 est code par $-2\mmod 2^3=6=\bin{110}$ car $-2=-1\times 2^3+6$ 
\end{itemize}

\item Avec cette représentation l'addition est cohérente : $2+(-1)=\bin{010}+\bin{111}=\bin{001}=1$.

On trouve donc la table ci-contre : 
\end{minipage}
\begin{minipage}{0.4\textwidth}

\begin{tabular}{|p{0.4\textwidth}|p{0.4\textwidth}|}
\hline 
Entier relatif en décimal & entier relatif en binaire \\ 
\hline 
-4 & $\bin{100}$ \\ 
\hline 
-3 & $\bin{101}$ \\ 
\hline 
-2 & $\bin{110}$ \\ 
\hline 
-1 & $\bin{111}$ \\ 
\hline 
0 & $\bin{000}$ \\ 
\hline 
1 & $\bin{001}$ \\ 
\hline 
2 & $\bin{010}$ \\ 
\hline 
3 & $\bin{011}$ \\ 
\hline 
\end{tabular} 
\end{minipage}

\end{exemple}


%\begin{exemple}[Sur 16 bits]
%\begin{itemize}
%\item[\textbullet] $-5$ est codé par $-5 \mmod 2^{16} = 65531 =
%  \bin{1111\,1111\,1111\,1011}$.
%\item[\textbullet] $3$ est codé par $3 \mmod 2^{16} = 3 =
%  \bin{0000\,0000\,0000\,0011}$.
%\item[\textbullet] La somme obtenue par le processeur est
%  \begin{equation*}
%    \bin{1111\,1111\,1111\,1110} = 65534 = 2^{16}-2 = -2 \mmod 65536
%  \end{equation*}
%\end{itemize}
%
%\begin{itemize}
%\item[\textbullet] $-4$ est codé par $-4 \mmod 2^{16} = 65532 =
%  \bin{1111\,1111\,1111\,1100}$.
%\item[\textbullet] $6$ est codé par $6 \mmod 2^{16} = 6 =
%  \bin{0000\,0000\,0000\,0110}$.
%\item[\textbullet] La somme obtenue par le processeur est
%  \begin{equation*}
%   \bin{0000\,0000\,0000\,0010} = 2 = 2 \mmod 65536
%  \end{equation*}
%\end{itemize}
%\end{exemple}

\subsubsection{Soustraction d'entiers relatifs}

Elle peut se faire relativement facilement (voir annexe).



\subsection{Dans les langages de programmation}

Dans de nombreux langages (C, Java, \ldots):
\begin{equation*}
  \text{Entiers du langage} = \text{Entiers sur $n$ bits}
\end{equation*}
Dans ces langages, sur une machine $64$ bits,
\texttt{4 * 2**62} vaut \texttt{0}.
%\clearslide{}

Dans d'autres langages, les entiers ne sont pas les entiers machines. Plusieurs
  représentation sont possibles. Parmis celles classiques : on utilise un tableau dont les
  éléments sont des octets/mots machines/chiffres dans une base $B$  (avec $B$ puissance de $2$ ou $10$).
 Des fonctions internes  au langage prennent alors soin d'effectuer les
  opérations correctement (en utilisant le fait que le processeur sait
  calculer sur $n$ bits). Python est dans ce cas.

\section{Représentation des réels}

L'essentiel à savoir :
\begin{itemize}
\item[\textbullet] Le principe de la représentation des nombres \emph{normalisés}.
\item[\textbullet] Les origines des problèmes de précision.
\item[\textbullet] Les conséquences de ces problèmes.
\end{itemize}
\subsection{Généralités}
Mathématiquement, il y a de nombreuses façons de voir les réels.

Une façon particulière : c'est la donnée d'un entier relatif, donnant la partie entière, et d'une
suite (infinie) de chiffres, donnant la partie fractionnaire.

Peut-on représenter une suite de chiffres infinies ? Oui, par un algorithme.

Peut-on représenter toutes les suites de chiffres infinies ? Non (cela découle des travaux de Cantor et de  Turing).

%\clearslide{}
Pour des besoins de calcul scientifique, nous n'avons pas besoin de représenter tous les réels. On travaille avec des approximations des réels : ici, les nombres décimaux.
\begin{rem}
  Qui dit approximation dit \emph{erreur}.
\end{rem}
\begin{defi}
Soit $(a,x)\in\R^2$. On distingue deux notions d'erreurs dans l'approximation de $x$ par $a$.
\begin{description}
\item[Erreur absolue :] $|x-a|$ 
\item[Erreur relative :]  $\displaystyle\frac{|x-a|}{|x|}$ (non définie si $x=0$)
\end{description}
\end{defi}
%\clearslide{}
On a aussi besoin d'avoir une représentation des nombres de taille réduite :
\begin{itemize}
\item[\textbullet] pour prendre une place réduite (en mémoire, sur disque, sur le réseau) ;
\item[\textbullet] pour calculer vite.
\end{itemize}

\subsection{Virgule fixe}

On représente tous les nombres décimaux avec un nombre $n$ fixé de chiffres après la virgule.

\begin{description}
  \item[Avantage :] on comprend bien comment ça marche. 
  \item[Inconvénient:]
\begin{itemize}
\item[\textbullet] On a parfois besoin de beaucoup de chiffres après la virgule 
(masse de l'électron: $9\times{}10^{-31}$kg, $h\approx 6\times10^{-34}$J.s).
\item[\textbullet] Garder $30$ chiffres après la virgule est parfois inutile pour manipuler
  un grand nombre (durée de vie moyenne de l'électron: $10^{34}$s).
\end{itemize}
\end{description}


Dans cette représentation, l'erreur absolue est au plus $10^{-n}$. Mais l'important est souvent l'erreur \emph{relative}.

\subsection{Virgule flottante}
On utilise plutôt l'idée de la notation scientifique des nombres. Un nombre est représenté sous la forme $s\times m \times 10^{e}$, avec $(s,m,e)$ définis comme suit. 
\begin{itemize} 
\item[\textbullet] $s\in\{-1;+1\}$ est le \emph{signe}.
\item[\textbullet] $m\in [1,10[$ est un nombre décimal, avec $n$ chiffres après la virgule ($n$ fixé). C'est la \emph{mantisse}.
\item[\textbullet] $e$: entier (relatif) appartenant à une plage de valeurs fixée. C'est \emph{l'exposant}.
\end{itemize}
%\clearslide{}
\begin{exemple}
  Sur une  calculatrice HP48SX (d'après tests personnels) :
\begin{itemize}
\item[\textbullet] la mantisse $m$ a $11$ chiffres après la virgule,
\item[\textbullet] l'exposant $e\in \ii{-499,500}$.
\end{itemize}
Cela permet de représenter :
\begin{itemize}
\item[\textbullet] de très grands nombres: jusqu'à $9,999\,999\,999\,99\times 10^{499}$ ;
\item[\textbullet] de très petits (en valeur absolue): jusqu'à $10^{-499}$ ;
\item[\textbullet] et leurs opposés: $-9,999\,999\,999\,99\times 10^{499}$ et $-10^{-499}$ ;
\item[\textbullet] avec une erreur relative inférieure à $10^{-11}$.
\end{itemize}
avec seulement $12$ chiffres décimaux, un signe et trois chiffres pour
l'exposant. 
\end{exemple}

\subsection{Virgule flottante en binaire}

La notation scientifique présentée plus haut utilise la base $10$. 
C'est souvent cohérent, mais pas toujours en informatique où l'on préférera utiliser la base $2$.
On a alors l'équivalent de la notion de nombre décimal, dans la base $2$.

\begin{defi}
 Un nombre (ou fraction) décimal est un nombre de la forme $\displaystyle\frac{n}{10^k}$, avec $n\in \Z$ et $k\in \N$.
\end{defi}

\begin{defi} 
  Un nombre (ou fraction) dyadique est un nombre de la forme $\displaystyle\frac{n}{2^k}$, avec $n\in
    \Z$ et $k\in \N$.
\end{defi}
%\clearslide{}
\begin{exemple}
En décimal le nombre $12345 / 10^{3}$ s'écrit $12,345$.
\end{exemple}
\begin{exemple}
En binaire, le nombre $\dfrac{\bin{10101011}}{\bin{10}^{\bin{101}}}$ s'écrit
$\bin{101,01011}$. Il vaut $\dfrac{171}{2^5}= 5,34375$.

Autre façon de calculer:
\begin{equation*}
  \bin{101,01011} =2^2+ 0\times 2^1+2^0 + \dfrac{0}{2} + \dfrac{1}{2^2} + \dfrac{0}{2^3} + \dfrac{1}{2^4} + \dfrac{1}{2^5}= 5 + \frac{1}{2^{2}} + \frac{1}{2^{4}}
  +\frac{1}{2^{5}}
\end{equation*}
\end{exemple}
%\clearslide{}

Un nombre sera donc représenté en virgule flottante en base $2$ sous la forme $s\times m \times 2^{e}$, avec $(s,m,e)$ comme suit.
\begin{itemize}
\item[\textbullet] $s\in\{-1;+1\}$ est le \emph{signe}.
\item[\textbullet] $m\in [1,2[$ est un nombre dyadique, avec $n$ chiffres après la virgule ($n$ fixé). C'est la \emph{mantisse}.
\item[\textbullet] $e$: entier (relatif) appartenant à une plage de valeurs fixée. C'est \emph{l'exposant}.
\end{itemize}

On commet au plus une erreur relative de $2^{-n}$ en représentant un réel ainsi. 


\subsection{Norme IEEE 754}

La norme IEEE 754 est utilisée dans tous les ordinateurs pour les nombres à
  virgule flottante. Elle existe en plusieurs versions (simple précision, double précision, double
  précision étendue). On ne parlera ici que de la double précision (la plus répandue).

%\clearslide{}

Les nombres réels seront donc représentés en virgule flottante avec double précision. Chaque nombre est  représenté sur $64$ bits, utilisés comme suit.
\begin{itemize}
\item[\textbullet] $1$ bit pour le signe ($0$ pour $+$, $1$ pour $-$).
\item[\textbullet] $11$ bits pour l'exposant décalé $e$ (exposant plus $1023$).
\item[\textbullet] $52$ bits pour les $52$ chiffres après la virgule de la mantisse
  (inutile de garder le premier bit de $m$: c'est $1$).
\end{itemize}

%\clearslide{}
On interprète donc la suit de bits $se_{10}\ldots e_{0}m_{1}\ldots m_{52}$ comme le nombre $x$ défini comme suit. 

Notons $\displaystyle e  = \underline{e_{10}\ldots e_{0}} = \sum_{k=0}^{10} e_k2^k$. 
\begin{itemize}
  \item Si $e\in\ii{1,2047}$, $x$ est le  nombre \emph{normalisé}:
\begin{equation*}
  x = s \times \underline{1,m_{1}\ldots m_{52}}\times
  2^{\p{-1023+\underline{e_{10}\ldots  e_{0}}}}
  = 
  s \times \p{1+\sum_{k=1}^{52} \frac{m_{k}}{2^{k}}}
  \times 2^{\p{-1023 + \sum_{k=0}^{10} e_{k}2^{k}}}
\end{equation*}

\item Si $e = 0$ et $m_{1}=\dots=m_{52}=0$ : $x=0$ (deux
  versions : $+0$ et $-0$).

\item  Si $e = 0$ et $m_{1},\ldots,m_{52}$ non tous nuls, $x$ est le nombre
\emph{dénormalisé}:
\begin{equation*}
  x = s \times \underline{0,m_{1}\ldots m_{52}}\times
  2^{-1022}
  = 
  s \times \p{\sum_{k=1}^{52} \frac{m_{k}}{2^{k}}}
  \times 2^{-1022}
\end{equation*}

\item  Si $e= 2047$ et $m_{1}=\ldots= m_{52}=0$ : $x = s\infty$ ($+\infty$
ou $-\infty$).

\item  Si $e = 2047$ et $m_{1},\ldots, m_{52}$ non tous nuls: $x=NaN$.
\end{itemize}
%\clearslide{}
\begin{rem}
  On ne rentrera pas dans le détail des signification de $+\infty$, $-\infty$ et de $NaN$.
\end{rem}


Les nombres normalisés permettent de représenter de façon précise les
réels de
$[-M,-m]\cup [m, M]$ 
avec
\begin{align*}
  m &\approx2^{-1022}\approx 2\times 10^{-308}\\
\text{et }  M &\approx 2^{1024}\approx 1,8 \times 10^{308}
\end{align*}
Les nombres dénormalisés ne respectent pas la convention de la notation scientifique standard, mais permettent de représenter des nombres plus petits que les nombres normalisés ne peuvent. 
%\clearslide{}


%\clearslide{}

\subsection{En Python}

Avec Python, on peut accéder à la représentation d'un nombre flottant par la méthode \pyv{.hex()}. Attention, le nombre est écrit en hexadécimal. 

\begin{exemple}
  Avec \pyv{5.5}. 
\begin{pyconsole}
5.5.hex()
\end{pyconsole}
En effet, on a 
\begin{equation*}
  5,5 = \dfrac{11}{8} \times 4 = \left(1 + \dfrac{1}{4} + \dfrac{1}{8} \right) \times 2^2.
\end{equation*}
En binaire, on écrit $1,375 = 1+\dfrac{1}{4} + \dfrac{1}{8}$ comme 
\begin{equation*}
  1,\,\underbrace{0110}\,\underbrace{0000}\,\underbrace{0000}\,\underbrace{0000}\,\underbrace{0000}\,\underbrace{0000}\,\underbrace{0000}\,\underbrace{0000}\,\underbrace{0000}\,\underbrace{0000}\,\underbrace{0000}\,\underbrace{0000}\,\underbrace{0000}.
\end{equation*}
Le regroupement indiqué par les accolades donne l'écriture hexadécimale 
\begin{equation*}
  1,6000000000000.
\end{equation*}
\end{exemple}


\subsection{Problèmes de précision}

On rencontre différents types de problèmes de précision.
\begin{enumerate}
\item Les problèmes liés aux arrondis des calculs.
\item Les problèmes liés au passage à la représentation binaire.
\end{enumerate}
%\clearslide{}

\subsubsection{Problèmes liés aux arrondis}
Supposons que l'on veuille effectuer des calculs avec des chiffres décimaux n'ayant que deux chiffres après la virgule.
\begin{exemple}
Pour la multiplication:
$1,23\times 1,56 = 1,9188$
\end{exemple}
\begin{exemple}
Pour l'addition:
$1,23\times 10^{3} + 4,56\times 10^{0} = 1,23456\times 10^{3}$
\end{exemple}
Pour garder deux chiffres après la virgule, on arrondit le résultat et l'on introduit donc une erreur d'approximation.

Ce problème se pose en décimal, comme en binaire !

%\clearslide{}
\subsubsection{Problèmes liés au passage à la représentation binaire}

\textbf{Attention:}
Les représentations binaires et décimales partagent les \emph{mêmes} problèmes d'arrondis. 
Cependant, on crée des erreurs d'arrondis lors du \emph{passage} d'une représentation à l'autre.
\begin{exemple}
En Python, on rentrera dans la console des nombres en écriture décimale mais le calcul interne se fera en binaire. Cela donne la chose suivante.
\begin{pyconsole}
0.1+0.2 == 0.3
0.1+0.2
0.1+0.2-0.3
\end{pyconsole}
\end{exemple}


\begin{center}
\textbf{Que se passe-t-il ???}
\end{center}

%\clearslide{}
\subsubsection{Origine de ce problème}

\begin{theoremeT}
Soit $p/q$ un nombre rationnel écrit sous forme irréductible, c'est-à-dire avec $p$
et $q$ entiers, premiers entre eux et $q>0$. Alors :
\begin{enumerate}
\item $p/q$ est un nombre décimal si et seulement
  si $q$ est de la forme $2^{\alpha}5^{\beta}$ où $(\alpha,\beta)\in
  \N\times \N$ ;
\item $p/q$ est un nombre dyadique si et seulement
  si $q$ est de la forme $2^{\alpha}$ où $\alpha\in \N$.
\end{enumerate}
\end{theoremeT}

%\clearslide{}
Ainsi, $\frac{1}{10}$ n'est pas un nombre dyadique.  En écriture décimale, 
$$1/10 = 0,1$$ 
alors qu'en écriture binaire, 
$$1/10 = \bin{0,00011001100110011001100110011...}.$$

Le flottant\footnote{On comprendra : représentation normalisée à virgule flottante en double précision.} (arrondi par défaut) $x$ représentant $\frac{1}{10}$ est donc
\begin{equation*}
 \bin{1,1001100110011001100110011001100110011001100110011001}\times 2^{-4}.
\end{equation*}
L'approximation dyadique au plus près de $0,1$ vaut donc
\begin{equation*}
 \bin{1,1001100110011001100110011001100110011001100110011010}\times 2^{-4},
\end{equation*}
qui est la représentation exacte de
\begin{equation*}
 0,1000000000000000055511151231257827021181583404541015625 
\end{equation*}
De même, le flottant $y$ (arrondi au plus proche) représentant
$\frac{2}{10}$ est
\begin{equation*}
 \bin{1,1001100110011001100110011001100110011001100110011010}\times 2^{-3} .
\end{equation*}

Ainsi, si on effectue le calcul de $x+y$, on obtient :
\small
\begin{align*}
&~~\bin{0},  \bin{11001100110011001100110011001100110011001100110011010}.
2^{-3}\\
+&~~\bin{1}, \bin{1001100110011001100110011001100110011001100110011010}.2^{-3}\\
=&\bin{10},\bin{01100110011001100110011001100110011001100110011001110}.
2^{-3}
\end{align*}
\normalsize
Arrondi au flottant le plus proche, cela donne :
\begin{equation*}
\bin{1},  \bin{0011001100110011001100110011001100110011001100110100}.2^{-2},
\end{equation*}
soit
\begin{equation*}
0,3000000000000000444089209850062616169452667236328125.
\end{equation*}

Cela explique bien ce que donne Python.
\begin{pyconsole}
0.1+0.2
\end{pyconsole}
Quand on effectue le calcul \pyv{0.1 + 0.2 - 0.3}, on crée de nouvelles erreurs d'arrondi. Ces dernières sont «négligeables» devant $0,1$, mais pas négligable $4\times 10^{-17}$ !
D'où le résultat final de l'ordre de $6\times 10^{-17}$ : 
\begin{pyconsole}
0.1+0.2-0.3
\end{pyconsole}


\subsection{Erreurs d'arrondis: conséquences}

\fbox{\begin{minipage}{\textwidth} \begin{center} \textsc{Un test de la forme}
\pyv{x == 0} ou \pyv{x == y} \textsc{pour des flottants n'a aucun sens !} \end{center}\end{minipage}}

\vspace{0.5cm}

 La seule possibilité parfois raisonnable est  le \og test de petitesse\fg.
\begin{exemple}
  On peut tester \pyv{abs(x)< epsilon} avec \pyv{epsilon = 1e-6}.
\end{exemple}

 La question qui se pose alors est : quelle valeur de \pyv{epsilon} choisir? Il n'y a pas de réponse universelle,  cela dépend du problème étudié \ldots{}


De même, on se méfiera des tests du type \pyv{x<y} ou \pyv{x<=y}.

%\clearslide{}

\begin{exemple}
  
\end{exemple}
 Construisons une équation du second degré.

\begin{pyconsole}
r1 = 1 + 1.2e-16
r1
r2 = 1
a, b, c = 1, -(r1+r2), r1 * r2
\end{pyconsole}
Normalement \pyv{r1} et \pyv{r2} sont les deux racines réelles distinctes de
$\text{\texttt{a}}X^2+\text{\texttt{b}}X+\text{\texttt{c}}$, qui a donc un discriminant strictement positif.
%\clearslide{}
Vérifions cela.
\begin{pyconsole}
Delta = b**2 - 4*a*c
Delta
\end{pyconsole}
Oups...
\begin{pyconsole}
a*r1**2 + b*r1 + c
a*r2**2 + b*r2 + c
\end{pyconsole}

On peut aussi trouver des cas où \pyv{Delta} est nul avec le polynôme
qui s'annule au moins sur deux flottants, dont l'un n'est pas supposé
être une racine...

%\section{Exercices}
%
%%\begin{exo}
%%    En mathématiques, la méthode de la dichotomie à partir d'un segment $[a_0,b_0]$ donne deux suites infinies de réels $(a_n)$ et $(b_n)$ vérifiant $\forall n \in \N$, $a_n<b_n$.

\question{Est-ce le cas en informatique ? }

\question{Justifier la réponse et l'illustrer par un script.}

%%\end{exo}
%

\section{Annexe : représentation détaillé des entiers}

\subsection*{Conversion d'un entier en base 2}
\sloppy
On s'intéresse à la démonstration de l'algorithme de converion par divisions successives.
Si $p = \underline{a_n\dots a_0}_{~2}$, alors 
$$ p = \sum_{k=0}^n a_k 2^k = 2\sum_{k=1}^n a_k2^{k-1} + a_0.$$
Ainsi, $a_0$ est le reste de la division euclidienne de $p$ par $2$ et, si $p\neq a_0$, $\underline{a_n\dots a_1}_{~2}$ est le quotient de la division euclidienne de $p$ par $2$.

On peut donc écrire la fonction suivante.

\begin{lstlisting}
def conv_b2(p):
    """Convertit l'entier p en base 2 (renvoie une chaine)"""
    x = p # On copie p
    s = ""
    i=0
    while x > 1 : 
        s = str(x%2) + s
        x = x // 2
        i=i+1
    return str(x)+s

print('0='+conv_b2(0)+' et 1='+conv_b2(1)+' et 42='+conv_b2(42))
\end{lstlisting}


\begin{pycode}
def conv_b2(p):
    x = p # On copie p
    s = ""
    i=0
    while x > 1 : 
        s = str(x%2) + s
        x = x // 2
        i=i+1
    return str(x)+s

print('0='+conv_b2(0)+' et 1='+conv_b2(1)+' et 42='+conv_b2(42))
\end{pycode}
\begin{itemize}
\item \textbf{Invariant : } $p=x+s$, avec : 
\begin{itemize}
\item \textbf{s : } $s=\underline{a_{i-1}\dots a_1a_0}_2$
\item \textbf{x : } $x=\displaystyle{\sum_{k=i}^n a_k2^{k-i}}$
\end{itemize}
\item \textbf{Initialisation : $i=0$}
\begin{itemize}
\item \textbf{s : } $s$ est vide
\item \textbf{x : } $x=\displaystyle{\sum_{k=i=0}^n a_k2^{k-0}=p}$
\end{itemize}
\item On suppose l'hypothèse vrai au rang $i$, montrons qu'elle est vrai au rang $i+1$ : 
\item $x=\displaystyle{\sum_{k=i}^n a_k2^{k-i}=a_i+\sum_{k=i+1}^n a_k2^{k-i}=a_i+2\cdot \sum_{k=i+1}^n a_k2^{k-(i+1)}}$
\begin{itemize}
\item $x\%2\rightarrow a_i$
\item $x//2\rightarrow \displaystyle{\sum_{k=i+1}^n a_k2^{k-(i+1)}}$
\end{itemize}
\item \textbf{Terminaison : }
à la sortie de la boucle $x\leq 1$ ce qui donne :
\begin{align*}
x=\displaystyle{\sum_{k=i+1}^n a_k2^{k-(i+1)}}\leq 1
\end{align*}
obtenue pour $i=n$ donc à la fin de la boucle $x=a_n$ et $s=\underline{a_{n-1}\dots a_1a_0}_2$, il faut donc bien ajouter $a_n$ à $s$.
\end{itemize}

\subsection*{Somme de naturels}
\sloppy

Dans un processeur $n$ bits:

\begin{enumerate}
\item Un registre du processeur a $n$ bits et peut
  représenter tout entier de $\ii{0,2^{n}}$.
\item Lorsqu'on effectue l'addition de deux registres $r_{1}$ et
  $r_{2}$ pour stocker le résultat dans $r_{3}$, le registre fait $n$
  bits: s'il y a une retenue, elle est perdue.\footnote{En fait une
    trace en est généralement gardée dans un autre registre du
    processeur.}

  Exemple: après addition de $\bin{1111\,0000\,1111\,0000}$ et
  $\bin{0011\,0011\,0011\,0011}$ sur $16$ bits, registre résultat:
  $\bin{0010\,0100\,0010\,0011}$.
\end{enumerate}

Troncature d'un entier $p$ à ses $n$ bits de poids faibles: valeur du
reste de la division de $p$ par $2^{n}$ (noté $p \mmod 2^{n}$).

Définitions:
\begin{enumerate}
\item Soit $p\in\N$. $p$ \emph{représentable comme
    entier non signé sur $n$ bits} si $p\in\ii{0,2^{n}}$.
\item \emph{Représentation de $p$ comme entier non signé sur $n$
    bits}:
  suite des $n$ chiffres de son écriture en binaire.
\item Abus de notation: on identifie $\ii{0,2^{n}}$ et les
  représentations sur $n$ bits.
\item Soit $(p,q)\in \ii{0,2^{n}}^{2}$.
  \emph{somme (non signée) de $p$ et $q$ sur $n$ bits}: $(p+q) \mmod
  2^{n}$, notée $p+_{n}q$ (notation non canonique).
\end{enumerate}

Remarques pour $(p,q)\in \ii{0,2^{n}}^{2}$:
  \begin{enumerate}
  \item $p+_{m}q \equiv p+q\ [2^{n}]$.
  \item $p+_{n}q = p+q$ si $p+q< 2^{n}$.
  \item $p+_{n}q = p+q - 2^{n}$ si $p+q \geq 2^{n}$.
  \end{enumerate}
\subsection*{Entiers relatifs}

\subsubsection*{Avec signe et valeur absolue}
Première possibilité: $n-1$ bits pour la valeur absolue et
$1$ bit pour le signe, par ex.:
\begin{itemize}
\item on représente $-4$ par $\bin{1000\,0000\,0000\,0100}$;
\item on représente $4$ par $\bin{0000\,0000\,0000\,0100}$.
\end{itemize}

Inconvénients:
\begin{enumerate}
\item Deux représentations pour zéro ($\bin{1000\,0000\,0000\,0000}$
  et $\bin{0000\,0000\,0000\,0000}$).
\item Plus compliqué à additionner que les entiers
  naturels.
\end{enumerate}

Représentation quasiment jamais utilisée pour les nombres
\emph{entiers} d'un processeur.
%\clearslide{}
\subsubsection*{Complément à deux}
Idée géniale:
\begin{enumerate}
\item L'addition d'entiers naturels sur le processeur n'est pas
  correcte mais l'est modulo $2^{n}$;
\item pour tout $p\in \Z$, $p\mmod 2^{n} \in \ii{0,2^{n}}$;
\item donc $p\mmod 2^{n}$ représentable sur $n$ bits;
\item l'addition de ces entiers relatifs sera correcte modulo $2^{n}$.
\end{enumerate}
%\clearslide{}

Exemples:
\begin{enumerate}
\item $-5$ est codé par $-5 \mmod 2^{16} = 65531 =
  \bin{1111\,1111\,1111\,1011}$.
\item $3$ est codé par $3 \mmod 2^{16} = 3 =
  \bin{0000\,0000\,0000\,0011}$.
\item La somme obtenue par le processeur est
  \begin{equation*}
    \bin{1111\,1111\,1111\,1110} = 65534 = 2^{16}-2 = -2 \mmod 65536
  \end{equation*}
\item $-4$ est codé par $-4 \mmod 2^{16} = 65532 =
  \bin{1111\,1111\,1111\,1100}$.
\item $6$ est codé par $6 \mmod 2^{16} = 6 =
  \bin{0000\,0000\,0000\,0110}$.
\item La somme obtenue par le processeur est
  \begin{equation*}
   \bin{0000\,0000\,0000\,0010} = 2 = 2 \mmod 65536
  \end{equation*}
\end{enumerate}

Définitions:
\begin{enumerate}
\item Soit $p\in \ii{0,2^{n}}$. \emph{Complément à deux de $p$ sur $n$
    bits}: $(-p) \mmod 2^{n}$, noté $c_{n}(p)$ (non canonique).
\item Soit $p\in\Z$. $p$ est \emph{représentable comme
    entier signé sur $n$ bits} si $p\in\ii{-2^{n-1},2^{n-1}}$.
\item Soit $p\in\ii{-2^{n-1},2^{n-1}}$. \emph{Représentation en
    complément à deux de $p$}: $p\mmod 2^{n}$, notée $r_{n}(p)$ (non
  canonique).
\end{enumerate}
%\clearslide{}

Remarques:
\begin{enumerate}
\item $c_{n}$ involution\footnote{Bijection
    d'un ensemble sur lui-même qui est sa propre bijection
    réciproque.}
  de $\ii{0,2^{n}}$.
\item Soit $p\in \ii{0,2^{n}}$. $c_{n}(p)=2^{n}-p$ si $p\neq 0$, $0$
  si $p=0$.
\item $r_{n}$  bijection de $\ii{-2^{n-1},2^{n-1}}$ sur
  $\ii{0,2^{n}}$.
\item $\forall p\in \ii{-2^{n-1}+1,2^{n-1}}\quad r_{n}(-p) = c_{n}(r_{n}(p))$.
\item Pour tout $(p,q)\in\ii{-2^{n-1},2^{n-1}}^{2}$ on a
  $r_{n}^{-1}\p{r_{n}(p)+_{n}r_{n}(q)} \equiv p+q \ [2^{n}]$.
\item On a l'égalité si $p+q\in \ii{-2^{n-1},2^{n-1}}$ (en
  particulier si $p$ et $q$ de signes différents).
\item Le bit de poids fort de $r_{n}(p)$
  vaut $1$ si et seulement si $p<0$.
\end{enumerate}

Intérêt du complément à deux: Pour calculer une addition, on utilise
exactement les mêmes circuits électroniques que pour des entiers non
signés!

\subsubsection*{Calcul de la représentation en complément à deux}

\begin{description}
\item[Définition] \emph{Complément à un sur $n$ bits de
  $\underline{a_{n-1}\ldots a_{0}}_{2}$}:
  $\underline{b_{n-1}\ldots b_{0}}_{2}$ où $b_{k}=1-a_{k}$ pour
  $k\in\ii{0,n}$. On note $c'_{n}(p)$ cette valeur.
\item[Proposition] Pour tout $p\in\ii{0,2^{n}}$,
  \begin{equation*}
   p + c'_{n}(p) = \sum_{k\in\ii{0,n}}2^{k}= 2^{n}-1 
  \end{equation*}
\item[Conséquence] Pour tout $p$ entier non signé sur $n$ bits,
  \begin{equation*}
    c_{n}(p) = c'_{n}(p) +_{n} 1 
  \end{equation*}
\item[Conséquence (bis)] Pour tout $p\in\ii{-2^{n-1}+1,2^{n-1}}$
  \begin{equation*}
    r_{n}(-p) = c'_{n}(r_{n}(p)) +_{n} 1
  \end{equation*}
\end{description}
%\clearslide{}

Exemple (sur $16$ bits):
\begin{enumerate}
\item Représentation  de $5$ en tant qu'entier non signé?
\item Complément à un de $5$?
\item Complément à deux de $5$?
\item Représentation de $-5$ sur $16$ bits?
\item Calcul de l'opposé de l'entier signé $\underline{1111\,1111\,1111\,1000}$?
\item Que vaut l'entier signé $\underline{1111\,1111\,1111\,1000}$?
\end{enumerate}

\subsubsection*{Soustraction}

Soit $(p,q)\in\ii{2^{n-1}-1,2^{n-1}}$.
\begin{description}
\item[Définition] \emph{Différence sur $n$ bits de $p$ et $q$}: $(p-
  q)\mmod 2^{n}$, notée $p-_{n}q$ (non canonique).
\item[Proposition] $p -_{n} q = p +_{n} c_{n}(q)$.
\end{description}

\begin{enumerate}
\item Complément à deux facile à calculer.
\item Addition/Soustraction: utilisation du même circuit!
\end{enumerate}


\subsection*{Octal et hexadécimal en informatique}

\begin{enumerate}
\item Représenter des données par une succession de chiffres binaires
  est bien adapté à l'utilisation de l'électronique pour construire des
  ordinateurs.
\item C'est également bien adapté au calcul sur les entiers.
\end{enumerate}
%\clearslide{}

Exemple de représentation par du binaire:

\begin{itemize}
\item Une adresse IPv6 est une suite de $128$ bits. Exemple d'adresse:
  \begin{equation*}
    \begin{array}{r}
  0010 0000 0000 0001 % 2001
  0000 0110 0111 1100\\ % 67c
  0000 0010 1110 1000 % 2e8
  0000 0000 0010 0010\\ % 22
  0000 0000 0000 0000 %
  0000 0000 0000 0000\\ % ::
  1100 0001 0000 0000 % c100
  0000 0110 1000 1011 % 68b
  %2001:67c:2e8:22::c100:68b
    \end{array}
  \end{equation*}
\item Une adresse IPv4 est une suite de $32$ bits. Exemple:
  \begin{equation*}
  % 129.175.15.11
  1000 0001 % 129
  1010 1111 % 175
  0000 1111 % 15
  0000 1011 % 11
  \end{equation*}
\end{itemize}

Ce n'est pas très pratique à manipuler\ldots{}

Comment trouver une représentation plus simple à manipuler?
%\clearslide{}

Deux possibilités:

\begin{enumerate}
\item Considérer ces suites de bits comme des entiers binaires et
  convertir en décimal (difficile à faire de tête)
\item Grouper ces suites de bits, par exemple par octet, remplacer
  chaque octet par sa représentation décimale.
\end{enumerate}

Adresses IPv4: 2\up{e} solution. Sur l'adresse précédente:
$129.175.15.11$

2\up{e}  solution: pas  pratique si  la  suite de  bits représente  un
entier (difficile ensuite de calculer sur la représentation).

%\clearslide{}
Autre solution:
\begin{enumerate}
\item Grouper les chiffres binaires par $4$ (en partant de la droite)
\item Remplacer chaque groupement par le chiffre hexadécimal
  correspondant.
\end{enumerate}

Avantages:
\begin{enumerate}
\item Facile à faire à la main et même de tête.
\item A du sens mathématiquement: le nombre hexadécimal obtenu a la
  même valeur que le nombre binaire de départ.
\end{enumerate}
%\clearslide{}

Solution prise pour les adresses IPv6.

Exemples:
\begin{itemize}
\item traduire en hexadécimal le nombre
$\bin{11 0101 0010 1010 1110}$
\item traduire en binaire le nombre
$\hex{90f5e56712}$
\end{itemize}

Pour l'octal: même principe mais en groupant par $3$.
%\clearslide{}

\begin{enumerate}
\item Binaire, octal et hexadécimal sont très utilisés en informatique.
\item En Python, notations autorisées:
  \begin{lstlisting}
>>> 0b101010
42
>>> 0o52
42
>>> 0x2a
42
  \end{lstlisting}
\item Écriture d'un entier \texttt{n} en décimal, binaire, octal ou
  hexadécimal: \texttt{str(n)}, \texttt{bin(n)},\texttt{oct(n)}, \texttt{hex(n)}.
\end{enumerate}
%\clearslide{}

Danger, ancienne notation encore autorisée en Python 2 mais à
  ne plus utiliser:
  \begin{lstlisting}
>>> 052
42
  \end{lstlisting}

En python 3:
  \begin{lstlisting}
>>> 052
  File "<stdin>", line 1
    052
      ^
SyntaxError: invalid token
  \end{lstlisting}
  
