% Déclaration des titres
% -------------------------------------


\graphicspath{{../../style/png/}{images/}{../../exos//}}
\lstinputpath{{../../exos//}}

\def\discipline{Informatique}
\def\xxtete{Informatique}

\def\classe{\textsf{MPSI}}
\def\xxnumpartie{1}
\def\xxpartie{Architecture matérielle et initiation à l'algorithmique}
\def\xxdate{11 Décembre 2019}

\def\xxchapitre{08}
\def\xxnumchapitre{08}
\def\xxnomchapitre{Représentation des nombres}
\def\xxnumactivite{08}

\def\xxposongletx{2}
\def\xxposonglettext{1.45}
\def\xxposonglety{19}%16

\def\xxonglet{\textsf{Cycle 01}}
\def\xxauteur{\textsl{Émilien Durif \\ Sylvaine Kleim \\ Xavier Pessoles }}


\def\xxpied{%
Cycle \xxnumpartie -- \xxpartie\\
Chapitre \xxnumchapitre -- \xxactivite -\xxnumactivite -- \xxnomchapitre%
}

\setcounter{secnumdepth}{5}
\chapterimage{Fond_ALG}
\def\xxfigures{}

\def\xxcompetences{%
\textsl{%
\textbf{Savoirs et compétences :}\\
\begin{itemize}[label=\ding{112},font=\color{ocre}]
\item AA.C2 : Appréhender les limitations intrinsèques à la manipulation informatique des nombres
\item AA.C3 : Initier un sens critique au sujet de la qualité et de la précision des résultats de calculs numériques sur ordinateur
\item AA.S4 : Principe de la représentation des nombres entiers en mémoire
\item AA.S5 : Principe de la représentation des nombres réels en mémoire
\end{itemize}
}}


%Infos sur les supports
\def\xxtitreexo{Représentation des nombres}
\def\xxsourceexo{\hspace{.2cm} \footnotesize{
}}
%\def\xxtitreexo{Titre EXO}
%\def\xxsourceexo{\hspace{.2cm} \footnotesize{Source EXO}}


%---------------------------------------------------------------------------


