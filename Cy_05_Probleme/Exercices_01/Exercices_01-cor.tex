%%%% Paramétrage du cours %%%%
\def\xxactivite{\ifprof TP -- Corrigé  \else  TP \fi}
\def\xxauteur{\textsl{Émilien Durif -- Xavier Pessoles}}

%\fichefalse
%\proftrue
%\tdfalse
%\courstrue

% Déclaration des titres
% -------------------------------------

\def\discipline{Informatique}
\def\xxtete{Informatique}

\def\classe{\textsf{MPSI}}
\def\xxnumpartie{3}
\def\xxpartie{Simulation numérique}
\def\xxdate{11 Mars 2020}

\def\xxnumchapitre{3}
\def\xxnomchapitre{Problèmes stationnaires}

\def\xxposongletx{2}
\def\xxposonglettext{1.45}
\def\xxposonglety{19}%16

\def\xxonglet{\textsf{Cycle 01}}
\def\xxauteur{\textsl{Émilien Durif \\ Xavier Pessoles}}


\def\xxpied{%
\xxnumpartie -- \xxpartie\\
\xxnumchapitre -- \xxactivite -- \xxnomchapitre%
}

\setcounter{secnumdepth}{5}
\chapterimage{Fond_SIMU}
\def\xxfigures{}

\def\xxcompetences{%
\textsl{%
\textbf{Savoirs et compétences :}\\
\begin{itemize}[label=\ding{112},font=\color{ocre}]
\item SN.C1; SN.C2; SN.C3; SN.C4; SN.C5; SN.S1; SN.S2
\item ....
\end{itemize}
}}


%---------------------------------------------------------------------------





\def\xxfigures{
%\includegraphics[width=.6\linewidth]{fig_00}
}%figues de la page de garde


\iflivret
\input{../../style/pagegarde_info}
\else
\input{../../style/pagegarde_info}
\fi
\setlength{\columnseprule}{.1pt}

\pagestyle{fancy}
\thispagestyle{plain}

\ifprof
\vspace{4.5cm}
\else
\vspace{4.5cm}
\fi

\def\columnseprulecolor{\color{ocre}}
\setlength{\columnseprule}{0.4pt} 

%%%%%%%%%%%%%%%%%%%%%%%

\setcounter{exo}{0}
\vspace{1cm}

\subsection*{Exercice 1 -- Arithmétique -- Corrigé}

\begin{corrige}
$\quad$

\begin{lstlisting}[language=Python]
# Question 1
n = 1234
q = n//10
r =  n%q

# r contient le nombre d'unités de n
\end{lstlisting}
\end{corrige}

\begin{corrige}
$\quad$
\begin{lstlisting}[language=Python] 
# Question 2
s=0
while n!=0:
    q=n//10
    r = n%10
    #print(r)
    s=s+r**3
    n=q
\end{lstlisting}%\end{python}
\end{corrige}

\begin{corrige}
$\quad$
\begin{lstlisting}[language=Python] 
# Question 3
def somcube(n):
    """
    Entrées :
     * n, int : nombre
    Sortie : 
     * s, int : somme des cubes du chiffre n
    """
    s=0
    while n!=0:
        q=n//10
        r = n%10
        s=s+r**3
        n=q
    return s
\end{lstlisting}%\end{python}
\end{corrige}

\begin{corrige}
$\quad$
\begin{lstlisting}[language=Python] 
# Question 4
res = []
for i in range (10001):
    if i == somcube(i):
        res.append(i)
\end{lstlisting}%\end{python}
\end{corrige}

\begin{corrige}
$\quad$
\begin{lstlisting}[language=Python] 
# Question 5
def somcube2(n):
    """
    Entrées :
     * n, int : nombre
    Sortie : 
     * s, int : somme des cubes du chiffre n
    """
    nombre=str(n)
    s=0
    for chiffre in nombre :
        s = s+int(chiffre)**3
    return s
    
print(somcube2(1234))
\end{lstlisting}%\end{python}
\end{corrige}

%\vfill

%%\columnbreak

%\addcontentsline{toc}{subsection}{Exercice 2 -- Intégration -- Corrigé}



\subsection*{Exercice 2 -- Intégration -- Corrigé}

\begin{corrige}
$\quad$
\begin{lstlisting}[language=Python] 
# Question 1
# ==========
# Le répertoire courant est Exercice_02.
# Le sous-répertoire data contient le
# fichier ex_02.txt.

# On ouvre le fichier en lecture)
fid = open("data\ex_02.txt")

# On charge le fichier dans une liste.
# Chaque élément de la liste correspond à 
# chaque ligne sous forme de chaîne de caractère.
file = fid.readlines()
# On ferme le fichier
fid.close()

LX=[]
LY=[]
for ligne in file :
    ligne = ligne.split(';')
    LX.append(float(ligne[0]))
    LY.append(float(ligne[1]))
\end{lstlisting}%\end{python}
\end{corrige}

\begin{corrige}
$\quad$
\begin{lstlisting}[language=Python] 
# Question 2
# ==========
# Ne pas oublier de charger préalablement 
# import matplotlib.pyplot as plt

plt.plot(LX,LY)
plt.show()
\end{lstlisting}%\end{python}
\end{corrige}
\begin{center}
\includegraphics[width=6cm] {images/exo_02}
\end{center}


\begin{corrige}
$\quad$
\begin{lstlisting}[language=Python] 
# Question 3
# ==========
def trapeze(x,y):
    res = 0 
    for i in range(1,len(LX)):
        res = res+(LX[i]-LX[i-1])*0.5*(LY[i]+LY[i-1])
    return res
print(trapeze(LX,LY))

>>> 75.13635
\end{lstlisting}%\end{python}
\end{corrige}

\begin{corrige}
$\quad$
\begin{lstlisting}[language=Python] 
# Question 4
# ==========
from scipy.integrate import trapz
# Attention à l'ordre des arguments dans 
# la fonction trapz : les_y puis les_x
# Après l'import, help(trapz) permet d'avoir 
# de l'aide sur la fonction.
print(trapz(LY,LX))

>>> 75.13635
\end{lstlisting}%\end{python}
\end{corrige}

%\addcontentsline{toc}{subsection}{Exercice 3 -- Graphe -- Corrigé}
\subsection*{Exercice 3 -- Graphe -- Corrigé}

\begin{corrige}
$\quad$
\begin{lstlisting}[language=Python] 
# Question 1
# ==========
# Matrices avec des listes
M=[[0,9,3,-1,7],
      [9,0,1,8,-1],
      [3,1,0,4,2],
      [-1,8,4,0,-1],
      [7,-1,2,-1,0]]
\end{lstlisting}%\end{python}
\end{corrige}
%\newpage

\begin{corrige}
$\quad$
\begin{lstlisting}[language=Python] 
# Question 2 & 3
# ==============
def voisins(M,i):
    """
    Entrées : 
      * M(lst) : graphe
      * i : noeud considéré
    Sortie :
      * v(lst) : liste des voisins
    """
    v = []
    # On cherche les voisins sur une ligne 
    # (on pourrait le faire sur une colonne)
    for j in range(len(M[i])):
        if M[i][j]>0:
            v.append(j)
    return v
    
# print(voisins(M,0))
\end{lstlisting}%\end{python}
\end{corrige}


\begin{corrige}
$\quad$
\begin{lstlisting}[language=Python] 
# Question 4
# ==========
def degre(M,i):
    """
    Entrées : 
      * M(lst) : graphe
      * i : noeud considéré
    Sortie :
      * (int) : nomnbre de voisins
    """
    return len(voisins(M,i))
\end{lstlisting}%\end{python}
\end{corrige}

\begin{corrige}
$\quad$
\begin{lstlisting}[language=Python] 
# Question 5
# ==========
def longueur(M,chemin):
    l = 0
    for i in range(len(chemin)-1):
        if M[chemin[i]][chemin[i+1]]<0:
            return -1
        else :
            l=l+M[chemin[i]][chemin[i+1]]
    return l
    
chemin = [1,2,3,1,4]
print(longueur(M,chemin))        
chemin = [0,4,2,1,0]
print(longueur(M,chemin))
\end{lstlisting}%\end{python}
\end{corrige}

\vfill

%\columnbreak
%\addcontentsline{toc}{subsection}{Exercice 4 -- Corrigé}
\subsection*{Exercice 4 -- Corrigé}
\begin{corrige}
$\quad$
\begin{lstlisting}[language=Python] 
# Question 1 
# ==========
def nombreZeros(t,i):
    if t[i]==1:
        return 0
    else : 
        res = 1
        j=i+1
        while j<len(t) and t[j]==0:
            res = res+1
            j=j+1
    return res
# t1=[0,1,1,1,0,0,0,1,0,1,1,0,0,0,0]
# print(nombreZeros(t1,4))
# print(nombreZeros(t1,1))
# print(nombreZeros(t1,8))
\end{lstlisting}%\end{python}
\end{corrige}

\begin{corrige}
$\quad$
\begin{lstlisting}[language=Python] 
# Question 2
# ==========
def nombreZerosMax(t):
    max=nombreZeros(t,0)
    for i in range(1,len(t)):
        tmp = nombreZeros(t,i)
        if tmp>max:
            max = tmp
    return max
print(nombreZerosMax(t1))
\end{lstlisting}%\end{python}
\end{corrige}

\begin{corrige}
La complexité est quadratique ($\mathcal{O}^2$)  du fait de la boucle \texttt{for} et de la boucle \texttt{while} imbriquée.

Pour diminuer la complexité, il est possible de parcourir une seule fois la liste. On lit alors les termes un à un. Quand on détecte un zéro, on compte alors le nombre de zéros consécutifs et on poursuit jusqu'à la fin...

\end{corrige}

