%%%% Paramétrage du cours %%%%
\def\xxactivite{\ifprof TP -- Corrigé  \else  TP \fi}
\def\xxauteur{\textsl{Xavier Pessoles}}

%\fichefalse
%\proftrue
%\tdfalse
%\courstrue

% Déclaration des titres
% -------------------------------------

\def\discipline{Informatique}
\def\xxtete{Informatique}

\def\classe{\textsf{MPSI}}
\def\xxnumpartie{3}
\def\xxpartie{Simulation numérique}
\def\xxdate{11 Mars 2020}

\def\xxnumchapitre{3}
\def\xxnomchapitre{Problèmes stationnaires}

\def\xxposongletx{2}
\def\xxposonglettext{1.45}
\def\xxposonglety{19}%16

\def\xxonglet{\textsf{Cycle 01}}
\def\xxauteur{\textsl{Émilien Durif \\ Xavier Pessoles}}


\def\xxpied{%
\xxnumpartie -- \xxpartie\\
\xxnumchapitre -- \xxactivite -- \xxnomchapitre%
}

\setcounter{secnumdepth}{5}
\chapterimage{Fond_SIMU}
\def\xxfigures{}

\def\xxcompetences{%
\textsl{%
\textbf{Savoirs et compétences :}\\
\begin{itemize}[label=\ding{112},font=\color{ocre}]
\item SN.C1; SN.C2; SN.C3; SN.C4; SN.C5; SN.S1; SN.S2
\item ....
\end{itemize}
}}


%---------------------------------------------------------------------------





\def\xxfigures{
%\includegraphics[width=.6\linewidth]{fig_00}
}%figues de la page de garde


\iflivret
\input{../../style/pagegarde_info}
\else
\input{../../style/pagegarde_info}
\fi
\setlength{\columnseprule}{.1pt}

\pagestyle{fancy}
\thispagestyle{plain}

\ifprof
\vspace{4.5cm}
\else
\vspace{4.5cm}
\fi

\def\columnseprulecolor{\color{ocre}}
\setlength{\columnseprule}{0.4pt} 

%%%%%%%%%%%%%%%%%%%%%%%

\setcounter{exo}{0}
\vspace{1.5cm}

\begin{multicols}{2}
\subsection*{Exercice 1 -- Arithmétique}
\begin{enumerate}
\item Soit l’entier $n = 1234$. Quel est le quotient, noté $q$, dans la division euclidienne de $n$ par $10$ ? Quel est
le reste ? Que se passe-t-il si on recommence la division par 10 à partir de $q$ ?
\item Écrire la suite d’instructions calculant la somme des cubes des chiffres de l’entier 1234.
\item Écrire une fonction \texttt{somcube}, d’argument \texttt{n}, renvoyant la somme des cubes des chiffres du nombre
entier \texttt{n}.
\item Trouver tous les nombres entiers inférieurs à 1000 égaux à la somme des cubes de leurs chiffres.
\item En modifiant les instructions de la fonction \texttt{somcube}, écrire une fonction \texttt{somcube2} qui convertit
l’entier \texttt{n} en une chaîne de caractères permettant ainsi la récupération de ses chiffres sous forme de
caractères. Cette nouvelle fonction renvoie toujours la somme des cubes des chiffres de l’entier \texttt{n}.
\end{enumerate}


%\addcontentsline{toc}{subsection}{Exercice 2 -- Intégration}
\subsection*{Exercice 2 -- Intégration}
On cherche à calculer une valeur approchée de l’intégrale d’une fonction donnée par des points dont les coordonnées sont situées dans un fichier.
\begin{enumerate}
\item Le fichier \texttt{ex\_01.txt}, contient des lignes écrites selon le modèle suivant :

\begin{center}

\texttt{0.0;1.00988282142}

\texttt{0.1;1.07221264497}

\end{center}

Chaque ligne contient deux valeurs flottantes séparées par un point-virgule, représentant respective-
ment l’abscisse et l’ordonnée d’un point. Les points sont ordonnés par abscisses croissantes.
Ouvrir le fichier en lecture, le lire et construire la liste \texttt{LX} des abscisses et la liste \texttt{LY} des ordonnées contenues dans ce fichier.
\item Représenter les points sur une figure.
\item Les points précédents sont situés sur la courbe représentative d’une fonction \texttt{f}. On souhaite déterminer une valeur approchée de l’intégrale \texttt{I} de cette fonction sur le segment où elle est définie. Écrire une fonction \texttt{trapeze}, d’arguments deux listes \texttt{y} et \texttt{x} de même longueur \texttt{n}, renvoyant :

$$
\sum\limits_{i=1}^{n-1} \left(x_i - x_{i-1} \right) \dfrac{y_i +y_{i-1}}{2}.
$$

\texttt{trapeze(LY,LX)} renvoie donc une valeur approchée de l’intégrale $I$ par la méthode des trapèzes.
\item En utilisant la méthode d’intégration numérique \texttt{trapz} de la sous-bibliothèque \texttt{scipy.integrate} du langage Python ou la méthode \texttt{inttrap} du logiciel Scilab, retrouver la valeur approchée de l'intégrale $I$.
\end{enumerate}

%\addcontentsline{toc}{subsection}{Exercice 3 -- Graphe}
\subsection*{Exercice 3}
On considère le graphe \texttt{G} suivant, où le nombre situé sur l'arête joignant deux sommets est leur distance, supposée entière :
\begin{center}
\includegraphics[width=.75\linewidth]{images/exo_3}
\end{center}

\begin{enumerate}
\item Construire la matrice $\left( M_{ij}\right)_{0\leq i,j\leq 4}$, matrice de distances du graphe \texttt{G}, définie par :

<< pour tous les indices $i$, $j$, $M_{ij}$ représente la distance entre les sommets $i$ et $j$,
ou encore la longueur de l'arête reliant les sommets $i$ et $j$ >>.

On convient que, lorsque les sommets ne sont pas reliés, cette distance vaut -1. La distance du
sommet $i$ à lui-même est, bien sûr, égale à 0.
\item Écrire une suite d'instructions permettant de dresser à partir de la matrice \texttt{M} la liste des voisins du sommet 4.
\item Écrire une fonction \texttt{voisins}, d'argument un sommet $i$, renvoyant la liste des voisins du sommet $i$.
\item Écrire une fonction \texttt{degre}, d'argument un sommet $i$, renvoyant le nombre des voisins du sommet $i$, c'est-à-dire le nombre d’arêtes issues de $i$.
\item Écrire une fonction \texttt{longueur}, d’argument une liste \texttt{L} de sommets de \texttt{G}, renvoyant la longueur du trajet d'écrit par cette liste \texttt{L}, c’est-à-dire la somme des longueurs des arêtes empruntées. Si le trajet n'est pas possible, la fonction renverra $-1$.
\end{enumerate}


%\addcontentsline{toc}{subsection}{Exercice 4 -- Gestion de liste}
\subsection*{Exercice 4 -- Gestion de liste}
Soit un entier naturel $n$ non nul et une liste \texttt{t} de longueur $n$ dont les termes valent 0 ou 1. Le but de cet exercice est de trouver le nombre maximal de 0 contigus dans \texttt{t} (c’est-à-dire figurant dans des cases consécutives). Par exemple, le nombre maximal de zéros contigus de la liste \texttt{t1} suivante vaut 4 :
\begin{center}
\begin{tabular}{|c|c|c|c|c|c|c|c|c|}
\hline 
\texttt{i} & 0 & 1 & 2 & 3 & 4 & 5 & 6 & 7 \\
\hline
\texttt{t1[i]} & 0 & 1 & 1 & 1 & 0 & 0 & 0 & 1 \\
\hline
\hline
\texttt{i} & 8 & 9 & 10 & 11 & 12 & 13 & 14 \\
\cline{0-7} 
\texttt{t1[i]} & 0 & 1 & 1 & 0 & 0 & 0 & 0 \\
\cline{0-7}  
\end{tabular}
\end{center}

\begin{enumerate}
\item Écrire une fonction \texttt{nombreZeros(t,i)}, prenant en paramètres une liste \texttt{t}, de longueur n, et un indice i compris entre 0 et $n-1$, et renvoyant :
$$
\left\{
\begin{array}{l}
0, \text{ si } t[i]=1 \\
\text{le  nombre de zéros consécutifs dans t} \\ \text{à partir de t[i] inclus, si t[i] = 0}.
\end{array}
\right.
$$
Par exemple, les appels \texttt{nombreZeros(t1,4)}, \texttt{nombreZeros(t1,1)} et \texttt{nombreZeros(t1,8)} renvoient respectivement les valeurs 3, 0 et 1.
\item Comment obtenir le nombre maximal de zéros contigus d’une liste \texttt{t} connaissant la liste des \texttt{nombreZeros(t,i)} pour $0\leq i \leq n-1$ ?
En déduire une fonction \texttt{nombreZerosMax(t)}, de paramètre \texttt{t}, renvoyant le nombre maximal de 0 contigus d’une liste \texttt{t} non vide. On utilisera la fonction \texttt{nombreZeros}.
\item Quelle est la complexité de la fonction \texttt{nombreZerosMax(t)} construite à la question précédente ?
\item Trouver un moyen simple, toujours en utilisant la fonction \texttt{nombreZeros}, d’obtenir un algorithme plus performant.
\end{enumerate}

\end{multicols}



