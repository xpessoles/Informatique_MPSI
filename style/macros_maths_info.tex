% macros utilisées pour les maths et l'info
% ce fichier n'est pas concerné par la mise en page.
\DeclareMathOperator*{\mygrando}{O}
\DeclareMathOperator{\return}{\mathrm{return}}
\DeclareMathOperator{\card}{\mathrm{card}}
\newcommand{\grando}{\mygrando\limits}
\newcommand{\unif}{\mathcal{U}}
\newcommand{\N}{\mathbb{N}}
\newcommand{\Z}{\mathbb{Z}}
\newcommand{\R}{\mathbb{R}}
\newcommand{\C}{\mathbb{C}}
\newcommand{\K}{\mathbb{K}}
%
\newcommand{\cA}{\mathscr{A}}
\newcommand{\cM}{\mathscr{M}}
\newcommand{\cL}{\mathscr{L}}
\newcommand{\cS}{\mathscr{S}}
%
\newcommand{\e}{\mathrm{e}}
%
\newcommand{\eps}{\varepsilon}
%
\newcommand{\z}[1]{\Z_{#1}}
\newcommand{\ztimes}[1]{\Z_{#1}^{\times}}
\newcommand{\ii}[1]{[\![#1[\![}
\newcommand{\iif}[1]{[\![#1]\!]}
\newcommand{\llbr}{\ensuremath{\llbracket}}
\newcommand{\rrbr}{\ensuremath{\rrbracket}}
\newcommand{\p}[1]{\left(#1\right)}
\newcommand{\ens}[1]{\left\{ #1 \right\}}
\newcommand{\croch}[1]{\left[ #1 \right]}
%\newcommand{\of}[1]{\lstinline{#1}}
% \newcommand{\py}[2]{%
%   \begin{tabular}{|l}
%     \lstinline+>>>+\textbf{\of{#1}}\\
%     \of{#2}
%   \end{tabular}\par{}
% }
\newcommand{\floor}[1]{\left\lfloor#1\right\rfloor}
\newcommand{\ceil}[1]{\left\lceil#1\right\rceil}
\newcommand{\abs}[1]{\left|#1\right|}
% \fct{A}{B}{x}{e} :
% A  --> B
% x |--> e

% Fonctions et systèmes
\newcommand{\fct}[5][t]{%
  \begin{array}[#1]{rcl}
    #2 & \rightarrow & #3\\
    #4 & \mapsto     & #5\\
  \end{array}
}
\newcommand{\fonction}[5]{#1 : \left\{\begin{array}{rcl} #2& \longrightarrow &#3 \\ #4 &\longmapsto & #5\end{array}\right.}
\newenvironment{systeme}{\left\{ \begin{array}{rcl}}{\end{array}\right.}

% Matrices
\newcommand{\mat}[1]{
  \begin{pmatrix}
    #1
  \end{pmatrix}
}
\newcommand{\inv}{\ensuremath{^{-1}}}
\newcommand{\bpm}{\begin{pmatrix}}
\newcommand{\epm}{\end{pmatrix}}
% Ponctuation et mots pour les équations

\newcommand{\pt}{\enspace .}
\newcommand{\virg}{\enspace ,}
\newcommand{\ptvirg}{\enspace ;}
\newcommand{\ptint}{\enspace ?}
% \newcommand{\et}{~\textrm{ et }~}
\newcommand{\soit}{~\textrm{ soit }~}
% \newcommand{\ou}{~\textrm{ ou }~}
%\newcommand{\si}{~\textrm{ si }~} %% Incompatible avec siunitx
\newcommand{\sinon}{~\textrm{ sinon }~}
\newcommand{\avec}{~\textrm{ avec }~}
\newcommand{\donc}{~\textrm{ donc }~}
\newcommand{\tq}{\text{ tq }~}

\newcommand{\norm}[1]{\ensuremath{\left\Vert {#1}\right\Vert}}
\newcommand{\Ker}{\mathop{\mathrm{Ker}}\nolimits}

%\newcommand{\python}{\texttt{Python}}

% bases de données
\newcommand{\relat}[1]{\textsc{#1}}
\newcommand{\attr}[1]{\emph{#1}}
\newcommand{\prim}[1]{\uline{#1}}
\newcommand{\foreign}[1]{\#\textsl{#1}}

% Théorèmes & co

\usepackage{amsthm}

\theoremstyle{definition}

%\newtheorem{thm}{Théorème}[subsection]
%\newtheorem{defi}[thm]{Définition}
%\newtheorem{prop}[thm]{Proposition}               
%\newtheorem{rem}[thm]{Remarque}
%\newtheorem{rems}[thm]{Remarques}
%\newtheorem{notation}[thm]{Notation}
%\newtheorem{ex}[thm]{Exemple}
%\newtheorem{exs}[thm]{Exemples}
%\newtheorem{exo}[thm]{Exercice}
%\newtheorem{exos}[thm]{Exercices}

%\newtheorem{algorithm}[thm]{Algorithm}
% \renewcommand*\proofname{Proof}
% \makeatletter% correct qed adjustment
% \renewenvironment{proof}[1][\proofname]{\par
%   \pushQED{\qed}%
%   \normalfont\topsep2\p@\@plus2\p@\relax
%   \trivlist
%   \item[\hskip\labelsep
% 	  \sffamily\bfseries #1]\mbox{}\hfill\\*\ignorespaces
% }{%
%   \popQED\endtrivlist\@endpefalse
% }
% \makeatother
% note environment
\newenvironment{note}[1]{% 
  \labeling{#1}
		\item[#1]\ignorespaces
}{%
  \endlabeling
}

% Cours bd
\newenvironment{mld}
{\par\begin{minipage}{\linewidth}\begin{tabular}{rp{0.7\linewidth}}}
{\end{tabular}\end{minipage}\par}


% Binaire, octal, hexa
\newcommand{\hex}[1]{\underline{\text{\texttt{#1}}}_{16}}
\newcommand{\oct}[1]{\underline{\text{\texttt{#1}}}_{8}}
\newcommand{\bin}[1]{\underline{\text{\texttt{#1}}}_{2}}
\DeclareMathOperator{\mmod}{\texttt{\%}}

% Bases de données

\newcommand{\att}{\ensuremath{\mathbf{att}}}
\newcommand{\dom}{\ensuremath{\mathbf{dom}}}
\newcommand{\sort}{\ensuremath{\mathbf{sort}}}
\newcommand{\relname}{\ensuremath{\mathbf{relname}}}
\newcommand{\var}{\ensuremath{\mathbf{var}}}
\newcommand{\FILM}{\ensuremath{\mathtt{FILM}}}
\newcommand{\JOUE}{\ensuremath{\mathtt{JOUE}}}
\newcommand{\PERSONNE}{\ensuremath{\mathtt{PERSONNE}}}
\newcommand{\PERSONNAGE}{\ensuremath{\mathtt{PERSONNAGE}}}

\newcommand{\ttid}{\ensuremath{\mathtt{id}}}
\newcommand{\tttitre}{\ensuremath{\mathtt{titre}}}
\newcommand{\ttdate}{\ensuremath{\mathtt{date}}}
\newcommand{\ttidr}{\ensuremath{\mathtt{idrealisateur}}}
\newcommand{\ttdatenais}{\ensuremath{\mathtt{datenaissance}}}
\newcommand{\ttnom}{\ensuremath{\mathtt{nom}}}
\newcommand{\ttprenom}{\ensuremath{\mathtt{prenom}}}
\newcommand{\ttidacteur}{\ensuremath{\mathtt{idacteur}}}
\newcommand{\ttidfilm}{\ensuremath{\mathtt{idfilm}}}
\newcommand{\ttidpersonnage}{\ensuremath{\mathtt{idpersonnage}}}

\newcommand{\fv}{\mathrm{libre}}
\newcommand{\sem}[1]{[\![ #1 ]\!]}


%%%% Pour moins d'erreurs
\renewcommand{\it}{\normalfont \itshape}
\renewcommand{\bf}{\normalfont \bfseries}
\renewcommand{\rm}{\normalfont \rmfamily}
\renewcommand{\tt}{\normalfont \ttfamily}

%%%% Compteur pour se rappeler des numéros de questions

\newcounter{saveenum}