% Déclaration des titres
% -------------------------------------


\graphicspath{{../../style/png/}{images/}{../../exos//}}
\lstinputpath{{../../exos//}}

\def\discipline{Informatique}
\def\xxtete{Informatique}

\def\classe{\textsf{MPSI}}
\def\xxnumpartie{4}
\def\xxpartie{Bases de données}
\def\xxdate{20 Mai 2020}

\def\xxchapitre{04}
\def\xxnumchapitre{04}
\def\xxnomchapitre{Algèbre relationnelle}
\def\xxnumactivite{04}

\def\xxposongletx{2}
\def\xxposonglettext{1.45}
\def\xxposonglety{19}%16

\def\xxonglet{\textsf{Cycle 01}}
\def\xxauteur{\textsl{Émilien Durif \\ Sylvaine Kleim \\ Xavier Pessoles }}


\def\xxpied{%
Cycle \xxnumpartie -- \xxpartie\\
Chapitre \xxnumchapitre -- \xxactivite -\xxnumactivite -- \xxnomchapitre%
}

\setcounter{secnumdepth}{5}
\chapterimage{Fond_DB}
\def\xxfigures{}

\def\xxcompetences{%
\textsl{%
%\vspace{-.5cm}
\textbf{Savoirs et compétences :}\\
\vspace{-.1cm}
\begin{itemize}[label=\ding{112},font=\color{ocre}]
\item BDD.C4 : Traduire dans le langage de l'algèbre relationnelle des requêtes écrites en langage courant
\item BDD.C5 : Concevoir une base constituée de plusieurs tables, et utiliser les jointures symétriques pour effectuer des requêtes croisées
\item BDD.S2 : Opérateurs usuels sur les ensembles dans un contexte de bases de données : union, intersection, différence.
\item BDD.S3 : Opérateurs spécifiques de l'algèbre relationnelle : projection, sélection (ou restriction), renommage, jointure, produit et division cartésiennes ; fonctions d'agrégation : min, max, somme, moyenne, comptage.
\end{itemize}
}}


%Infos sur les supports
\def\xxtitreexo{Algèbre relationnelle}
\def\xxsourceexo{\hspace{.2cm} \footnotesize{
}}
%\def\xxtitreexo{Titre EXO}
%\def\xxsourceexo{\hspace{.2cm} \footnotesize{Source EXO}}


%---------------------------------------------------------------------------


