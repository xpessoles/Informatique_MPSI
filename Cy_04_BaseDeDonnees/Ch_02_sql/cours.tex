\section{Le modèle logique (MLD)}

On a représenté des données par des tables. Voici par exemple celle des films.
\begin{center}
\begin{tabular}{ll}
\toprule
  titre & date\\
\midrule
  Gran Torino & 2008\\
  The good, the Bad and the Ugly& 1966\\
  Study in Pink & 2010\\
  Schindler's List& 1993\\
  Dr Strangelove&1964\\
  Invictus& 2009\\
\bottomrule
\end{tabular}
\end{center}

Au départ, on a identifié les films par leur titre. On dit que le
champ titre de la table «films» est une \emph{clé primaire} pour
signifier que :
\begin{itemize}
\item tout film a un titre;
\item pour un titre donné, il y au plus un film ayant ce titre;
\item on fera référence à tout film par son titre.
\end{itemize}
Une clé primaire peut être constituée de plusieurs champs. 
%\clearslide{}
Exemple
pour la table «personnes» sans identifiants :  le couple (nom, prénom). On dit qu'il
s'agit  d'une clé primaire \emph{composée}.

\begin{center}
\begin{tabular}{lll}
\toprule
nom & prénom & datenaissance\\
\midrule
Kubrick&Stanley&1928\\
Spielberg & Steven& 1946\\
Eastwood & Clint& 1930\\
Cumberbatch & Benedict & 1976\\
Freeman & Martin& 1971\\
Leone & Sergio & 1929 \\
McGuigan & Paul & 1963\\
Sellers&Peter&1925\\
\bottomrule
\end{tabular}
\end{center}

%\clearslide{}

On a vu qu'il est préférable d'utiliser des identifiants pour faire référence à une personne. La clef primaire est alors naturellement cet identifiant. 

\begin{center}
\begin{tabular}{llll}
\toprule
id & nom & prénom & datenaissance\\
\midrule
1&Kubrick&Stanley&1928\\
2&Spielberg & Steven& 1946\\
3&Eastwood & Clint& 1930\\
4&Cumberbatch & Benedict & 1976\\
5&Freeman & Martin& 1971\\
6&Leone & Sergio & 1929 \\
7&McGuigan & Paul & 1963\\
8&Sellers&Peter&1925\\
\bottomrule
\end{tabular}
\end{center}
%\clearslide{}

Dans la table «joue» (sans identifiants), on fait référence à ces clés.
On dit que les champs (nom, prénom) de la table «joue» sont une \emph{clé
étrangère} car ils font référence à une clé primaire de la table
«personnes».

\begin{center}
  \begin{tabular}{llll}
    \toprule
    nom & prenom & titre & nom (de personnage)\\
    \midrule
    Eastwood & Clint & {\tiny The good, the Bad and the Ugly}& Blondie\\
    Eastwood & Clint & Gran Torino& Walt Kowalski\\
    Cumberbatch&Benedict&Study in Pink& Sherlock Holmes\\
    Freeman & Martin&Study in Pink&Dr John Watson\\
    Selers & Peters&Dr Strangelove&Dr Strangelove\\
    Selers & Peters&Dr Strangelove&{\tiny Group Capt. Lionel Mandrake}\\
    Selers & Peters&Dr Strangelove&{\small President Merkin Muffley}\\
\bottomrule
  \end{tabular}
\end{center}

%\clearslide{}

%\begin{exo}
  Voici la table de la relation «joue», avec les identifiants.
\begin{center}
  \begin{tabular}{lll}
    \toprule
    idacteur&idfilm&idpersonnage \\
    \midrule
    3&2&2\\
    3&1&1\\
    4&3&3\\
    5&3&4\\
    8&5&5\\
    8&5&6\\
    8&5&7\\
    \bottomrule
  \end{tabular}
\end{center}
Que peut-on considérer comme clef primaire ? Y a-t-il une ou des clef étrangères ?
%\end{exo}



\section{Vers le modèle physique (MPD)}

Pour implanter  cette base de  données on peut choisir  n'importe quel
langage de programmation.

\textbf{Mais} tous les langages ne sont pas aussi adaptés à la
modélisation et l'interrogation des données.


\subsection{SQL}

Nous allons étudier le langage \textbf{SQL}\footnote{Prononcer comme le mot anglais
  \textit{sequel}.}, pour : \emph{Standard Query Language}.


Notamment
\begin{itemize}
\item C'est un langage de définition de données (LDD ou DDL), pour définir le
  format des données.
\item C'est un langage de manipulation de données (LMD ou DML) pour ajouter ou
  supprimer des données.
\item Il possède une instruction SELECT pour effectuer des requêtes (recherches) sur
  les données.
\end{itemize}

C'est \textbf{le} standard actuel (décliné en différentes versions),
implanté dans les systèmes de gestion de bases de données.

%\clearslide{}
\subsection{Définition de tables}

Nous allons utiliser SQLite, un outil très dépouillé pour la
manipulation d'une base de données.

En SQlite, une base de données est stockée dans un fichier (et tout
fichier contient une unique base de données).

%\clearslide{}
\subsubsection{Lancement de SQLite}

En ligne de commande, taper
\begin{Verbatim}[commandchars=\\\{\}]
sqlite3 \emph{f}
\end{Verbatim}
pour lancer l'interface (textuelle) de sqlite3.

Cette  commande:
\begin{itemize}
\item  ouvre  la base  de  données contenue  dans le  fichier
\emph{f};
\item crée cette base si ce fichier n'existe pas.
\end{itemize}

%\clearslide{}
\subsubsection{Table «personne»}

On va créer une table «personne» avec 4 champs:
\begin{itemize}
\item \verb+id+
\item \verb+nom+
\item \verb+prenom+
\item \verb+datenaissance+
\end{itemize}

Il reste à préciser le \emph{type} des données que contiendront ces
champs.

\subsubsection{Types de données en SQL}
SQL permet de nombreux types de données, dont :
\begin{description}
\item[VARCHAR($n$)] (une chaîne de caractères d'au plus $n$ caractères) ;
\item[INT ou INTEGER] (entier) ;
\item[REAL] (nombre flottant) ;
\item[DECIMAL($p$, $s$)] (décimal sur $p$ chiffres, dont $s$ après la virgule) ;
\item[DATE] (une date, par exemple : 2016-02-29)
\item[TIME] (une heure de la journée, par exemple 15:43:07, précision en
  général $10^{-7} s$) ;
\item[DATETIME] (couple des deux précédents, par exemple 2016-02-29 15:43:07).
\end{description}


\subsection{Définition des tables}

La définition de tables est l'objet de la partie de SQL appelée DDL (\emph{Data Definition
Language}). Les deux opérations les plus importantes sont :
\begin{description}
\item[CREATE] pour créer une table ;
\item[DROP] pour effacer une table («DROP TABLE \emph{nomtable};»).
\end{description}

On les utilise de la manière suivante.



\lstset{language=SQL,morekeywords={REFERENCES}}
\lstinputlisting{images/create-personne.sql}

On fera attention à 
\begin{itemize}
\item ne pas oublier de terminer les commandes par des «;»
\item noter la déclaration de la clé primaire.
\end{itemize}

%\clearslide{}

On crée de même la table «personnage»:
\lstinputlisting{images/create-personnage.sql}



%\clearslide{}
et la table «joue»
\lstinputlisting{images/create-joue.sql}

Notez bien les déclarations de clés étrangères.
Enfin, on crée la table \og film\fg .
\lstinputlisting{images/create-film.sql}
%\clearslide{}
\subsection{Manipulation des tables }

La manipulation de tables est l'objet de la partie de SQL appelée DML (\emph{Data Manipulation
Language}). Voici quelques actions possibles.
\subsubsection{Insertions de valeurs}

On peut insérer une donnée:
\begin{lstlisting}
INSERT INTO PERSONNE (nom, prenom, datenaissance)
  VALUES('Kubrick', 'Stanley', date('1928-07-26'));
\end{lstlisting}

%\clearslide{}
On remarquera que l'on n'a pas donné de valeur pour la variable \texttt{id}.

\begin{itemize}
\item Pour une clé primaire entière, la base de données donne une
  valeur si on ne la fournit pas.
\item On peut indiquer lors de la création de
  la table une valeur par défaut pour les champs manquants, sinon la
  valeur \texttt{NULL} est mise par défaut, sauf si on l'a interdit
  lors de la création de la table.
\end{itemize}

%\clearslide{}
\begin{exemple}
La commande
\begin{lstlisting}
INSERT INTO PERSONNE (prenom, nom)
  VALUES ('Clint', 'Eastwood');
\end{lstlisting}
est accepté (pas de date de naissance renseignée), alors que 
\begin{lstlisting}
INSERT INTO PERSONNE (nom) VALUES ('Spielberg');
\end{lstlisting}
déclenche l'erreur suivante. 
\begin{Verbatim}
Error: PERSONNE.prenom may not be NULL
\end{Verbatim}
\end{exemple}



\begin{rem}
  Il n'y a pas besoin de mettre les champs dans le même ordre que dans la
déclaration de la table.
\end{rem}
 
%\clearslide{}


%\clearslide{}
\subsubsection{Modification de valeurs}
On peut modifier une valeur déjà renseignée. 
\begin{lstlisting}
UPDATE PERSONNE
SET datenaissance = date('1930-05-31')
WHERE prenom = 'Clint' AND nom = 'Eastwood';
\end{lstlisting}

\subsubsection{Suppression de valeurs}
On peut aussi supprimer des entrées de la base de données. 
\begin{lstlisting}
DELETE FROM PERSONNE WHERE prenom='Clint';
\end{lstlisting}
%\clearslide{}

\subsubsection{Vérification des contraintes de clés primaires}
Lorsque l'on ajoute une valeur dans une nouvelle base et que l'on veut spécifier la clef primaire, cette dernière ne doit bien évidemment pas avoir déjà été affectée. 
\begin{lstlisting}
INSERT INTO PERSONNE (id, prenom, nom)
  VALUES (3, 'Steven', 'Spielberg');
INSERT INTO PERSONNE (id, prenom, nom)
  VALUES (3, 'Benedict', 'Cumberbatch');
\end{lstlisting}
La première insertion acceptée, la deuxième donne:
\begin{Verbatim}
Error: PRIMARY KEY must be unique
\end{Verbatim}

%\clearslide{}
\subsection{Requêtes}

Les requêtes SQL sont introduites par le mot-clé \texttt{SELECT}.

Une requête renvoie une liste de $n$-uplets, que l'on peut identifier à une table.  De très nombreuses variantes de cette requête sont possibles, mais elles ont toutes en commun la structure suivante. 
\begin{lstlisting}
  SELECT col 
  FROM t ;
\end{lstlisting}
qui renvoie les colonnes désignées par \texttt{col} de la table \texttt{t}. On remarquera que \texttt{t} peut être le résultat d'une  autre requête, c'est-à-dire que l'on pourra avoir des constructions du type
\begin{lstlisting}
  SELECT col 
  FROM (SELECT colbis 
        FROM [...]
  );
\end{lstlisting}
Enfin, on pourra renommer les colonnes (on dit qu'on leur donne un \emph{alias}) par la commande \texttt{AS} : 
\begin{lstlisting}
  SELECT col AS alias 
  FROM t ;
\end{lstlisting}


\subsubsection{Lister tous les éléments d'une table}
On dispose d'un symbole \emph{joker} \texttt{*}, qui désigne toutes les colonnes d'une table. 
\lstinputlisting{images/star-from-personne.sql}
% columns/nom/.style={string type},
% columns/prenom/.style={string type},
% columns/datenaissance/.style={string type},
\pgfplotstableset{col sep=tab,
  every head row/.style={before row=\toprule,after row=\midrule},
  every last row/.style={after row=\bottomrule},
  string type,
  column type=l,
}
\begin{center}
\pgfplotstabletypeset{images/star-from-personne.csv}
\end{center}
%\clearslide{}
\subsubsection{Lister certaines colonnes d'une table}

\lstinputlisting{images/nom-prenom-personne.sql}
\begin{center}
\pgfplotstabletypeset{images/nom-prenom-personne.csv}
\end{center}

\subsubsection{Lister certaines lignes d'une table.}
Si on sait que Clint Eastwood est le réalisateur no 3:
\lstinputlisting{images/films-eastwood.sql}
\begin{center}
\pgfplotstabletypeset{images/films-eastwood.csv}
\end{center}

Et sinon, on utilise une requête \emph{croisée}, sur les tables \og film \fg\ et \og personne\fg\ :
\lstinputlisting{images/films-eastwood2.sql}
\begin{center}
\pgfplotstabletypeset{images/films-eastwood2.csv}
\end{center}


% \lstinputlisting{star-from-personne-film.sql}
% \begin{center}
% \pgfplotstabletypeset{star-from-personne-film.csv}
% \end{center}

\subsubsection{Calcul de jointure.}

Dans l'exemple précédent, on construit le \emph{produit cartésien} des tables \texttt{FILM} et \texttt{PERSONNE}, écrit comme la table \texttt{FILM, PERSONNE}. 
Cela revient, pour chaque ligne de la table \texttt{FILM}, de lui concaténer chaque ligne de la table \texttt{PERSONNE}. 
Cela crée une table ayant autant de lignes que le produit des nombre de lignes des tables \texttt{FILM} et \texttt{PERSONNE}.

L'instruction précédente ne conserve dans cette table produit que les lignes pour lesquelles les identifiants du réalisateur du film et celui de la personne coïncident. 
On a donc concaténé à chaque ligne de la table \texttt{FILM} la ligne de la table \texttt{PERSONNE} du réalisateur de ce film. 
On dit que l'on a réalisé une \emph{jointure} de ces deux tables, selon le critère \texttt{idrealisateur = PERSONNE.id}. 

On peut reformuler la requête précédente en utilisant l'instruction spécifique \texttt{JOIN ON}.
\lstinputlisting{images/films-eastwood-join.sql}
\begin{center}
\pgfplotstabletypeset{images/films-eastwood-join.csv}
\end{center}

\begin{rem}
  On peut mettre tout type de clause booléenne à droite du \textbf{\texttt{ON}}, même si seul l'exemple présenté est au programme. 
  
  On peut aussi mettre plusieurs instructions \texttt{JOIN ON}, qui partent toutes d'une même table centrale. 
\end{rem}


\subsubsection{Ensembles ou listes ?}
Commençons par un exemple. 
\lstinputlisting{images/acteurs-non-distincts.sql}
\begin{center}
\pgfplotstabletypeset{images/acteurs-non-distincts.csv}
\end{center}

\texttt{SELECT} ne renvoie pas un ensemble de $n$-uplets mais en fait
une liste de $n$-uplets: il peut y avoir des doublons. On peut les
supprimer:

\lstinputlisting{images/acteurs.sql}
\begin{center}
\pgfplotstabletypeset{images/acteurs.csv}
\end{center}

\subsubsection{Tri des résultats.}
On peut trier la liste renvoyée (par ordre alphabétique sur les chaînes de caractères, temporel sur les dates).

On peut trier par ordre croissant avec le mot clef \texttt{ASC}.
\lstinputlisting{images/acteurs-tries.sql}
\begin{center}
\pgfplotstabletypeset{images/acteurs-tries.csv}
\end{center}

On peut trier par ordre décroissant avec le mot clef \texttt{DESC}.
\lstinputlisting{images/personnes-triees.sql}
\begin{center}
%\pgfplotstabletypeset{images/personnes-triees.csv} % FICHIER A TROUVER
\end{center}


\subsubsection{Requêtes agrégats.}
On peut compter le nombre de lignes d'une table par la fonction  \texttt{COUNT}.
\'Etant donné une colonne, on peut facilement calculer des moyennes (fonction \texttt{AVG}), obtenir le maximum (\texttt{MAX}) et le minimum (\texttt{MIN}), ou bien une somme (\texttt{SUM}). 
\lstinputlisting{images/nb-films.sql}
\begin{center}
\pgfplotstabletypeset{images/nb-films.csv}
\end{center}

%\clearslide{}
\lstinputlisting{images/date-prem-film.sql}
\begin{center}
\pgfplotstabletypeset{images/date-prem-film.csv}
\end{center}

%\clearslide{}

On peut calculer des cumuls. Par exemple, le nombre de films que les
différents réalisateurs ont tournés:

\lstinputlisting{images/realisateurs-nb.sql}
\begin{center}
\pgfplotstabletypeset{images/realisateurs-nb.csv}
\end{center}

%\clearslide{}
Pour compter dans combien de
films chaque acteur a joué:

\lstinputlisting{images/acteurs-nb-roles.sql}
\begin{center}
\pgfplotstabletypeset{images/acteurs-nb-roles.csv}
\end{center}

%\clearslide{}
Aïe, ce n'est clairement pas ce qu'on voulait (Peter Sellers n'a joué
que
dans un film de notre base de données). Pourquoi?
%\clearslide{}

Façon possible de corriger (compliquée)
\lstinputlisting{images/acteurs-nb-films.sql}
\begin{center}
\pgfplotstabletypeset{images/acteurs-nb-films.csv}
\end{center}
%\clearslide{}
La condition \texttt{WHERE} porte sur les données qu'on va ensuite
agréger. On peut aussi mettre des conditions sur les agrégats à
retenir:

\lstinputlisting{images/acteurs-plusieurs-roles.sql}
\begin{center}
\pgfplotstabletypeset{images/acteurs-plusieurs-roles.csv}
\end{center}

\subsubsection{Opérations sur les tables.}

On peut faire l'union de deux tables avec la commande \texttt{UNION}, l'intersection de deux tables avec la commande \texttt{INTERSECT} et la différence (au sens ensembliste) de deux tables avec la commande \texttt{EXCEPT}. 
Tout ceci permet d'écrire plus simplement des requêtes, qui demanderaient des écritures lourdes à base de \texttt{WHERE}, \texttt{AND}, \texttt{OR} \textit{etc.}

Par exemple, de manière quelque peu factice, on peut vouloir obtenir  la liste des personnes nées après 1950 ou dont l'identifiant est inférieur ou égal à 4. 

\lstinputlisting{images/ex-union.sql}
\begin{center}
\pgfplotstabletypeset{images/ex-union.csv}
\end{center}

De même, on peut obtenir la liste des personnes nées après 1950 et dont l'identifiant est inférieur ou égal à 4. 

\lstinputlisting{images/ex-intersect.sql}
\begin{center}
\pgfplotstabletypeset{images/ex-intersect.csv}
\end{center}

Enfin, on peut obtenir la liste des personnes nées après 1950, exceptées celles dont l'identifiant est inférieur ou égal à 4. 

\lstinputlisting{images/ex-except.sql}
\begin{center}
\pgfplotstabletypeset{images/ex-except.csv}
\end{center}

\subsection{Annexes}

Cette partie ne sera pas traitée en cours.

\subsubsection{Calcul du lieu d'un maximum, d'un minimum.}
C'est une question assez fréquente. On peut par exemple chercher à déterminer le ou les acteurs les plus âgés (\emph{i.e.} qui possède la date de naissance minimale. 

Une première requête imbriquée naïve. 

\lstinputlisting{images/lieu-max-age1.sql}
\begin{center}
\pgfplotstabletypeset{images/lieu-max-age1.csv}
\end{center}

Sous la condition qu'il n'y a qu'un enregistrement réalisant ce minimum, on pourra aussi utiliser l'idée suivante : il suffit de trier la table par dates de naissances, puis de conserver uniquement le premier enregistrement (fonction \texttt{LIMIT}, on pourra aussi utiliser  \texttt{OFFSET k} pour passer les \texttt{k} premières lignes).
\lstinputlisting{images/lieu-max-age2.sql}
\begin{center}
FICHIER MANQUANT %\pgfplotstabletypeset{images/lieu-max-age2.csv}
\end{center}

\subsubsection{Compter des enregistrements différents.}

On peut essayer de compter le nombre d'enregistrements ayant des valeurs distinctes. Par exemple : combien d'acteurs ont joué dans les films de notre base ? 

Commençons part une requête imbriquée naïve. 
\lstinputlisting{images/nb-acteurs1.sql}
\begin{center}
FICHIER MANQUANT %\pgfplotstabletypeset{images/nb-acteurs1.csv}
\end{center}

On peut aussi écrire cela comme suit. 
\lstinputlisting{images/nb-acteurs2.sql}
\begin{center}
\pgfplotstabletypeset{images/nb-acteurs2.csv}
\end{center}
\subsubsection{Opérations sur les nombres et les chaînes de caractères.}

On peut réaliser des opérations arithmétiques sur les nombres : \texttt{+} pour l'addition, \texttt{-} pour la soustraction \emph{etc.}    

Par exemple, pour l'addition.
\lstinputlisting{images/addition.sql}
\begin{center}
FICHIER MANQUANT %\pgfplotstabletypeset{images/addition.csv}
\end{center}

Attention, comme en Python, les opérations sur les entiers donnent des résultats entiers. 
\lstinputlisting{images/division1.sql}
\begin{center}
FICHIER MANQUANT %\pgfplotstabletypeset{images/division1.csv}
\end{center}

Pour effectuer des divisions flottantes (par exemple, pour calculer ici un pourcentage), on pourra utiliser une fonction de conversion ou bien ruser avec la division entière.
\lstinputlisting{images/division2.sql}
\begin{center}
FICHIER MANQUANT %\pgfplotstabletypeset{images/division2.csv}
\end{center}

On pourra mettre des chaînes de caractères en minuscules ou en majuscules.
\lstinputlisting{images/upper-lower.sql}
\begin{center}
FICHIER MANQUANT %\pgfplotstabletypeset{images/upper-lower.csv}
\end{center}

Cette liste n'est bien entendu pas exhaustive. 

\section{Exercices}

Donner des requêtes pour:

\begin{itemize}
\item trouver le nombre d'acteurs qui ont joué au moins deux rôles ;
\item trouver les acteurs qui sont aussi des réalisateurs ;
\item trouver les acteurs nés le 1er janvier 1930 ou après ;
\item compter les acteurs nés avant le 1er janvier 1940 ;
\item trouver les acteurs qui ont joué au moins dans un film où jouait
  Martin Freeman ;
\item trouver les réalisateurs qui ont dirigé Clint Eastwood.
\end{itemize}
