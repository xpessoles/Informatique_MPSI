% Déclaration des titres
% -------------------------------------


\graphicspath{{../../style/png/}{images/}{../../exos//}}
\lstinputpath{{../../exos//}}

\def\discipline{Informatique}
\def\xxtete{Informatique}

\def\classe{\textsf{MPSI}}
\def\xxnumpartie{4}
\def\xxpartie{Bases de données}
\def\xxdate{13 Mai 2020}

\def\xxchapitre{03}
\def\xxnumchapitre{03}
\def\xxnomchapitre{Architecture trois tiers}
\def\xxnumactivite{03}

\def\xxposongletx{2}
\def\xxposonglettext{1.45}
\def\xxposonglety{19}%16

\def\xxonglet{\textsf{Cycle 01}}
\def\xxauteur{\textsl{Émilien Durif \\ Sylvaine Kleim \\ Xavier Pessoles }}


\def\xxpied{%
Cycle \xxnumpartie -- \xxpartie\\
Chapitre \xxnumchapitre -- \xxactivite -\xxnumactivite -- \xxnomchapitre%
}

\setcounter{secnumdepth}{5}
\chapterimage{Fond_DB}
\def\xxfigures{}

\def\xxcompetences{%
\textsl{%
\textbf{Savoirs et compétences :}\\
\begin{itemize}[label=\ding{112},font=\color{ocre}]
\item BDD.C3 : Distinguer les rôles respectifs des machines client, serveur, et éventuellement serveur de données
\item BDD.S4 : Concept de client-serveur. Brève extension au cas de l'architecture trois-tiers.
\end{itemize}
}}


%Infos sur les supports
\def\xxtitreexo{Architecture trois tiers}
\def\xxsourceexo{\hspace{.2cm} \footnotesize{
}}
%\def\xxtitreexo{Titre EXO}
%\def\xxsourceexo{\hspace{.2cm} \footnotesize{Source EXO}}


%---------------------------------------------------------------------------


