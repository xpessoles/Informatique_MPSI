% Déclaration des titres
% -------------------------------------


\graphicspath{{../../style/png/}{images/}{../../exos/algo/ALG-017/}{../../exos/python_bases/PYB-517/}{../../exos/algo/ALG-020/}{../../exos/algo/ALG-021/}{../../exos/algo/ALG-022/}}
%\lstinputpath{}
\def\discipline{Informatique}
\def\xxtete{Informatique}

\def\classe{\textsf{MPSI}}
\def\xxnumpartie{2}
\def\xxpartie{Analyse des algorithmes}
\def\xxdate{8 Janvier 2020}

\def\xxchapitre{01}
\def\xxnumchapitre{01}
\def\xxnomchapitre{Invariance et variance}
\def\xxnumactivite{01}

\def\xxposongletx{2}
\def\xxposonglettext{1.45}
\def\xxposonglety{19}%16

\def\xxonglet{\textsf{Cycle 01}}
\def\xxauteur{\textsl{Émilien Durif -- Sylvaine Kleim \\ Xavier Pessoles }}


\def\xxpied{%
Cycle \xxnumpartie -- \xxpartie\\
Chapitre \xxnumchapitre -- \xxactivite -\xxnumactivite -- \xxnomchapitre%
}

\setcounter{secnumdepth}{5}
\chapterimage{Fond_ANA}
\def\xxfigures{}

\def\xxcompetences{%
\textsl{%
%\vspace{-.5cm}
\textbf{Savoirs et compétences :}\\
\vspace{-.1cm}
\begin{itemize}[label=\ding{112},font=\color{ocre}]
\item AN.C1 : Justifier qu'une itération (ou boucle) produit l'effet attendu au moyen d'un invariant
\item AN.C2 : Démontrer qu'une boucle se termine effectivement
\end{itemize}
}}


%Infos sur les supports
\def\xxtitreexo{Invariance et variance}
\def\xxsourceexo{\hspace{.2cm} \footnotesize{\textbf{Sources : }

exercice 3 : Recherche d'invariant
exercice 4 : Recherche d'invariant
exercice 5 : Recherche d'invariant
}}
%\def\xxtitreexo{Titre EXO}
%\def\xxsourceexo{\hspace{.2cm} \footnotesize{Source EXO}}


%---------------------------------------------------------------------------


