\begin{multicols}{2}
\begin{pyverbatim}
def moyenne(t):
    """Calcule la moyenne de t
       Précondition : t est un tableau de 
       nombres non vide"""
    s = 0 
    for x in t:
        # Invariant : 
        # s == somme des éléments de t 
        # avant x
        s = s + x 
    return s/len(t)
\end{pyverbatim}

%\begin{lstlisting}
\begin{pyverbatim}
def variance(t):
    """Renvoie la variance de t
       Précondition : t est un tableau de 
       nombres non vide"""
    sc = 0
    for x in t:
        # Invariant : sc == somme des 
        # carrés des éléments de 
        # t avant x
        sc = sc + x**2 
    return sc/len(t) - moyenne(t)**2
\end{pyverbatim}
%\end{lstlisting}


\begin{pyverbatim}
def moyenne(t):
    """Calcule la moyenne de t
       Précondition : t est un tableau de 
       nombres non vide"""
    n = len(t) # Longueur de t
    s = 0 
    for i in range(n):
        # Invariant : s == sum(t[0:i])
        s = s + t[i]
    return s/n 
\end{pyverbatim}



\begin{pyverbatim}
def maxi(t):
    """Renvoie le plus grand élément de t.
       Précondition : t est un tableau 
       non vide"""
    m = t[0]
    for x in t:
        # Invariant : m est le plus grand 
        # élément trouvé jusqu'ici
        if x > m:
            m = x # On a trouvé plus grand, 
                  # on met à jour m
    return m
\end{pyverbatim}


\begin{pyverbatim}
def maxi(t):
    """Renvoie le plus grand élément de t.
       Précondition : t est un tableau 
       non vide"""
    m = t[0] # Initialisation par le 
             # premier élément
    for i in range(1, len(t)):
        # Invariant : m == max(t[0:i])
        if t[i] > m:
            m = t[i] # On a trouvé plus
                     # grand, on met à 
                     # jour m
    return m
\end{pyverbatim}

\vfill\null
\columnbreak

\begin{pyverbatim}
def indicemaxi(t):
    """Renvoie l'indice du plus grand 
        élément de t.
       Précondition : t est un tableau 
       non vide"""
    im = 0 # Indice du maximum, 
           # initialisation par 
           # le premier élément
    for i in range(1, len(t)):
        # Invariant : im est indice d'un 
        # plus grand élément de t[0:i]
        if t[i] > t[im]:
            im = i # On a trouvé plus grand, 
                   # on met à jour im
    return im
    
\end{pyverbatim}



\begin{pyverbatim}
def indicemaxi(t):
    """Renvoie l'indice du plus grand élément 
       de t.
       Précondition : t est un tableau 
       non vide"""
    im = 0 # Indice du maximum, 
            # initialisation par le 
            # premier élément
    for i, x in enumerate(t):
        # Invariant : im est indice 
        # d'un plus grand élément de t[0:i]
        if x > t[im]:
            im = i # On a trouvé plus grand, 
                    # on met à jour im
    return im
\end{pyverbatim}



\begin{pyverbatim}
def appartient(e, t):
    """Renvoie un booléen disant si e 
    appartient à t
       Précondition : t est un tableau"""
    for x in t:
        # Invariant : e n'est pas positionné 
        # dans t avant x
        if e == x:
            return True # On a trouvé e, 
                      # on s'arrête
    return False
 
\end{pyverbatim}

\begin{pyverbatim}
def ind_appartient(e,t):
    """Renvoie l'indice de la première 
       occurrence de e dans t,
       None si e n'est pas dans t
       Précondition : t est un tableau"""
    for i in len(t):
        # Invariant : e n'est pas dans t[0:i]
        if t[i] == e:
            return i # On a trouvé e 
                   # à l'indice i
    return None
\end{pyverbatim}

\end{multicols}