% Déclaration des titres
% -------------------------------------


\graphicspath{{../../../style/png/}{images/}{../../../exos//}}

\def\discipline{Informatique}
\def\xxtete{Informatique}

\def\classe{\textsf{MPSI}}
\def\xxnumpartie{2}
\def\xxpartie{Analyse des algorithmes}
\def\xxdate{24 Janvier 2020}

\def\xxchapitre{2}
\def\xxnumchapitre{2}
\def\xxnomchapitre{Complexité}
\def\xxnumactivite{09}

\def\xxposongletx{2}
\def\xxposonglettext{1.45}
\def\xxposonglety{19}%16

\def\xxonglet{\textsf{Cycle 01}}
\def\xxauteur{\textsl{Émilien Durif \\ Xavier Pessoles}}


\def\xxpied{%
Cycle \xxnumpartie -- \xxpartie\\
Chapitre \xxnumchapitre -- \xxactivite -\xxnumactivite -- \xxnomchapitre%
}

\setcounter{secnumdepth}{5}
\chapterimage{Fond_ANA}
\def\xxfigures{}

\def\xxcompetences{%
\textsl{%
\textbf{Savoirs et compétences :}\\
\begin{itemize}[label=\ding{112},font=\color{ocre}]
\item  
\item  
\item  
\item AN.C3 S’interroger sur l’efficacité algorithmique temporelle d’un algorithme
\item AN.S1 Recherche dans une liste, recherche du maximum dans une liste de nombres, calcul de la moyenne et de la variance.
\item AN.S2 Recherche par dichotomie. 
\item AN.S4 Recherche d’un mot dans une chaîne de caractères. 
\item  
\item  
\end{itemize}
}}


%Infos sur les supports
\def\xxtitreexo{}
\def\xxsourceexo{\hspace{.2cm} \footnotesize{}}
%\def\xxtitreexo{Titre EXO}
%\def\xxsourceexo{\hspace{.2cm} \footnotesize{Source EXO}}


%---------------------------------------------------------------------------


