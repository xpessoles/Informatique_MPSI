Le but est d'obtenir un encadrement de
\quad $\displaystyle I=4\int_0^1\frac{\text{d}x}{1+x^2}$

\begin{minipage}[c]{.6\linewidth}
\question{}
Compléter cet algorithme et le coder en python afin d'obtenir une valeur approchée de $I$ par la méthode des rectangles à gauche en utilisant les champs suivants : \fbox{x+h}
\fbox{(b-a)/n}
\fbox{h}
\fbox{somme+f(x)}
\fbox{f(a)}.

\question{}
Modifier cet algorithme pour que la méthode soit celle des rectangles à droite.

\question{}
Modifier cet algorithme afin d'afficher les résultats des deux méthodes.

\question{}
Augmenter le nombre de subdivisions.

\question{}
Justifier que la méthode des rectangles à droite donne un minorant de $I$ et que la méthode des rectangles à gauche donne un majorant. 
\end{minipage} \hfill
\begin{minipage}[c]{.35\linewidth}
\begin{py}
\begin{python}
a=0
b=1
n=100
h=
x=a
somme=
for k in range(1,n) :
    x=
    somme=
print(somme* 
\end{python}
\end{py}
\end{minipage}