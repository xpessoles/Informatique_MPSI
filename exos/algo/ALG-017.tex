%\subsection{Un exemple : test de primalité} 


On veut tester si un entier \pyv{n} est premier:


\begin{lstlisting}
def est_premier(n):
    """ Renvoie True si n est premier, False sinon
        Préconditon : n est un entier."""
    b = True
    for d in range(2,n):
        # b => n non divisible par 2, 3, ..., d-1.
        if n % d == 0:
            b = False
    # b <=> n premier
    return b
\end{lstlisting}

Remarque: les derniers tours de boucle sont inutiles dès que la
variable \pyv{b} a été mise à \pyv{False}. Les éxécuter tout de même est une perte de temps. Il 
existe plusieurs possibilités pour améliorer cela~:

L'instruction \pyv{break} :
\begin{pyverbatim}
def est_premier(n):
    """ Renvoie True si n est premier, False sinon
        Préconditon : n est un entier."""
    b = True
    for d in range(2,n):
        # b => n pas divisible par 2, 3, ..., d-1.
        if n % d == 0:
            b = False
            break
    # b <=> n est premier
    return b
\end{pyverbatim}

L'utilisation d'un \pyv{return} en milieu de boucle, à favoriser en Python{} pour 
une question de style et d'élégance~:

\begin{pyverbatim}
def est_premier(n):
    """ Renvoie True si n est premier, False sinon
        Préconditon : n est un entier."""
    for d in range(2,n):
        # n n'est pas divisible par 2, 3, ..., d-1.
        if n % d == 0:
            return False
    return True
\end{pyverbatim}