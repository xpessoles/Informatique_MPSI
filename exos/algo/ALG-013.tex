On donne la \pyv{fonctionMystere(n)} définie comme suit.
\begin{xxpy}~\\
\vspace{-1cm}
\begin{pyverbatim}
def fonctionMystere(n) :
    if n==0 or n==1:
         return 1
    else :
        res = 1
    for i in range (2,n+1) :
        res = res * i
    return res
\end{pyverbatim}
\end{xxpy}


\question{Si $n=5$ quelles sont les valeurs que va prendre la variable \pyv{i} ? }

\question{Si $n=4$ donner les valeurs successives que vont prendre les variables \pyv{i} et \pyv{res}  lorsqu'on exécute l'algorithme. }

\question{Quel est le nom mathématique usuel donné à la fonction \pyv{fonctionMystere} ?}