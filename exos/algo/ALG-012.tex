\begin{minipage}[c]{.6\linewidth}
La société Sharp commercialise des caisses automatiques utilisées par exemple dans des boulangeries. Le client glisse directement les billets ou les pièces dans la machine qui se charge de rendre automatiquement la monnaie. 
\begin{obj}
Afin de satisfaire les clients, on cherche à déterminer un algorithme qui va permettre de rendre le moins de monnaie possible. 
\end{obj}
\end{minipage}\hfill
\begin{minipage}[c]{.37\linewidth}
\begin{center}
\includegraphics[width=.9\textwidth]{sharp.png}
\end{center}
\end{minipage}

%\setcounter{subparagraph}{0}

La machine dispose de billets de 20€, 10€ et 5€ ainsi que des pièces de 2€, 1€, 50, 20, 10, 5, 2 et 1 centimes. 

On se propose donc de concevoir un algorithme qui demande à l'utilisateur du programme la somme totale à payer ainsi que le montant donné par l'acheteur. L'algorithme doit alors afficher quels sont les billets et les pièces à rendre par le vendeur. 


\question{}
\textit{Pour un montant d'achat donné et pour une somme donnée par le client, proposer un algorithme en pseudo code permettant de rendre le minimum de monnaie au client. Cet algorithme devra détailler la somme à rendre (nombre de pièces et nombre de billets).}

\question{}
\textit{Quelle structure de donnée est-il possible d'utiliser pour gérer les valeurs des billets ou des pièces ?}

\question{}
\textit{Quel type de variable est il préférable d'utiliser pour décompter l'argent ? Pourquoi ?}

\question{}
\textit{Implémenter cet algorithme dans Python.}

\question{}
\textit{Vérifier sur plusieurs cas que l'algorithme fonctionne.}