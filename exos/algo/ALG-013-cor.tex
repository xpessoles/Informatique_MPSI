\begin{minipage}{0.5\textwidth}
\question{Si $n=5$ quelles sont les valeurs que va prendre la variable \pyv{i} ? }

\begin{pycode}
def fonctionMystere(n) :
    if n==0 or n==1:
         return 1
    else :
        res = 1
    for i in range (2,n+1) :
        res = res * i
    return res

print('fonctionMystere(5) renvoie '+str(fonctionMystere(5))+'\n')
\end{pycode}
\end{minipage}
\begin{minipage}{0.5\textwidth}
\question{Si $n=4$ donner les valeurs successives que vont prendre les variables \pyv{i} et \pyv{res}  lorsqu'on exécute l'algorithme. }

\begin{pycode}
def fonctionMystere(n) :
    if n==0 or n==1:
         return 1
    else :
        res = 1
    for i in range (2,n+1) :
        res = res * i
        print('i='+str(i)+' res='+str(res)+'\n')
    return res

fonctionMystere(4)
\end{pycode}
\end{minipage}




\question{Quel est le nom mathématique usuel donné à la fonction \pyv{fonctionMystere} ?}

\pyv{fonctionMystere(n)} retourne la factorielle de $n$.