Il existe de nombreuses traditions étranges et amusantes sur Algoréa, la grande course de 
grenouilles annuelle en fait partie. Il faut savoir que les grenouilles algoréennes sont beaucoup 
plus intelligentes que les grenouilles terrestres et peuvent très bien être dressées pour participer 
à des courses. Chaque candidat a ainsi entraîné sa grenouille durement toute l'année pour ce grand 
événement.\\
La course se déroule en tours et, à chaque tour, une question est posée aux dresseurs. Le premier 
qui trouve la réponse gagne le droit d'ordonner à sa grenouille de faire un bond. Dans les règles de 
la course de grenouilles algoréennes, il est stipulé que c'est la grenouille qui restera le plus 
longtemps en tête qui remportera la victoire. Comme cette propriété est un peu difficile à vérifier, 
le jury demande votre aide.\\
Ce que doit faire votre programme :\\

\noindent \texttt{nbg} numérotées de \texttt{1} à \texttt{nbg} sont placées sur une ligne 
de départ. À chaque tour, on vous indique le numéro de la seule grenouille qui va sauter lors de ce 
tour, et la distance qu'elle va parcourir en direction de la ligne d'arrivée.\\
Écrivez un programme qui détermine laquelle des grenouilles a été strictement en tête de la course 
à la fin du plus grand nombre de tours.\\

\noindent ENTRÉE : deux entiers \texttt{nbg} et \texttt{nbt} et un tableau \texttt{t}.\\
\texttt{nbg} est le nombre de grenouilles participantes.\\
\texttt{nbt} est le nombre de tours de la course.\\
\texttt{t} est un tableau ayant \texttt{nbt} éléments, et tel que chaque élément est un couple : 
(numéro de la grenouille qui saute lors de ce tour, longueur de son saut).\\
\noindent SORTIE : vous devez renvoyer un entier : le numéro de la grenouille qui a 
été strictement en tête à la fin du plus grand nombre de tours. En cas d'égalité entre plusieurs 
grenouilles, choisissez celle dont le numéro est le plus petit.\\

\noindent EXEMPLE :\\
entrée : (4, 6, [ [2,2], [1,2], [3,3], [4,1], [2,2], [3,1] ] )\\
sortie : 2.