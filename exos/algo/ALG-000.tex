Votre robot dispose de nombreux récepteurs et enregistre tous les signaux qui l'entourent. Cependant 
vous avez remarqué que certains de ces signaux sont très bruités. Vous décidez donc d'écrire un 
programme qui atténue le bruit de ces signaux, en effectuant ce que l'on appelle un lissage.\\
Une opération de lissage d'une séquence de mesures (des nombres décimaux) consiste à remplacer 
chaque mesure sauf la première et la dernière, par la moyenne des deux valeurs qui l'entourent.\\

Par exemple, si l'on part de la séquence de mesures suivantes :

$$1\hspace{.4cm} 3\hspace{.4cm} 4\hspace{.4cm} 5$$

On obtient après un lissage :

$$1\hspace{.4cm} 2.5\hspace{.4cm} 4\hspace{.4cm} 5$$

Le premier et dernier nombre sont inchangés. Le deuxième nombre est remplacé par la moyenne du 1er 
et du 3e, soit $(1+4)/2 = 2.5$, et le troisième est remplacé par $(3+5)/2 = 4$.\\

On peut ensuite repartir de cette nouvelle séquence, et refaire un nouveau lissage, puis un autre 
sur le résultat, etc.\\
Votre programme doit calculer le nombre minimum de lissages successifs nécessaires pour s'assurer 
que la valeur absolue de la différence entre deux valeurs successives de la séquence finale obtenue 
ne dépasse jamais une valeur donnée, \texttt{diffMax}.\\

On vous garantit qu'il est toujours possible d'obtenir la propriété voulue en moins de 5000 lissages 
successifs.\\

\noindent ENTRÉE : Un tableau \texttt{t} contenant les mesures, qui sont des flottants, et un 
flottant \texttt{diffmax}.\\
SORTIE : vous devez retourner un entier sur la sortie : le nombre minimal de lissages nécessaire.\\

\noindent EXEMPLE :\\
\noindent entrée : \texttt{[1.292, 1.343, 3.322, 4.789, -0.782, 7.313, 4.212], 1.120}\\
sortie : \texttt{13}
