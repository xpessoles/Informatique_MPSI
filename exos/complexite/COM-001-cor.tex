\question\
Voici un algorithme répondant à la question. 
\begin{Verbatim}[gobble=0,numbers=left]
def add(M,N):
    """M+N, représentation classique"""
    n = len(M)
    S = [[0]*n for i in range(n)] # M+N
    for i in range(n):
        # Invariant : S[:i] = (M+N)[:i]
        for j in range(n):
            # Invariant : S[:i+1][:j] = (M+N)[:i+1][:j]
            S[i][j] = M[i][j] + N[i][j]
            # Invariant : S[:i+1][:j+1] = (M+N)[:i+1][:j+1]
        # Invariant : S[:i+1] = (M+N)[:i+1]
    return S
\end{Verbatim}
Montrons que \texttt{add(M,N)} renvoie bien le bon résultat. On a supposé que \texttt{M} et \texttt{N} sont de même dimension. 
\begin{description}
  \item[Ligne 4] \texttt{S} représente bien une matrice carrée d'ordre \texttt{n}. On a $S[:0] = []$ et $S[:0][:0] = []$, donc l'invariant de boucle de la ligne 4 est bien vérifié en début de boucle.
  \item[Ligne 6] On suppose que, pour tout $0\leq a < i$ et $0\leq b < j$, $S[a][b] = M[a][b] + N[a][b]$. Notamment, $S[i][:0] = []$, donc l'invariant de boucle de la ligne 6 est bien vérifié en début de boucle. 
  \item[Ligne 8] On suppose que, pour tout $0\leq b < j$, $S[i][b] = M[i][b] + N[i][b]$. 
  \item[Ligne 9] Directement, $S[i][j] = M[i][j] + N[i][j]$.
  \item[Ligne 10] Ainsi, $S[:i+1][:j+1] = (M+N)[:i+1][:j+1]$.
  \item[Ligne 11] En fin de boucle, $j = n-1$, donc $S[:i+1] = (M+N)[:i+1]$.
  \item[Ligne 12] En fin de boucle, $i = n-1$, donc $S = M+N$. L'algorithme renvoie bien le bon résultat.
\end{description}

\question\
Voici un algorithme répondant à la question. 
\begin{Verbatim}[gobble=0,numbers=left]
def lexico(X,Y):
    """(a,b) < (c,d) pour l'ordre lexicographique
        avec X = (a,b,x), Y = (a,b,y)"""
    a,b,_ = X
    c,d,_ = Y
    return (a < c) or (a==c and b < d) 

def add_creuse(M,N):
    """M+N, représentation creuse"""
    m,n = len(M), len(N)
    iM, iN = 0,0 # Indices des prochains éléments de M, N à ajouter
    S = [] # M+N
    while (iM < m) or (iN < n):
        # Invariant : M[:iM] et N[:iN] ont été additionnés
        # Variant : m+n-iM-iN
        if (iM == m) or (iN < n and lexico(N[iN],M[iM])):
            S.append(N[iN])
            iN += 1
        elif (iN == n) or lexico(M[iM],N[iN]):
            S.append(M[iM])
            iM += 1
        else:
            if M[iM][2]+N[iN][2] != 0:
                S.append((M[iM][0],M[iM][1],M[iM][2]+N[iN][2]))
            iM +=1
            iN +=1
    return S
\end{Verbatim}
Montrons que \texttt{add\_creuse(M,N)} renvoie bien le bon résultat. 
\begin{description}
  \item[Fonction lexico] \pyv{lexico((i,j,x),(a,b,y))} renvoie un booléen indiquant si \pyv{(i,j)} est strictement plus petit que \pyv{(a,b)} dans l'ordre lexicographique.
  \item[Lignes 11-12] \pyv{M[:iM]} et \pyv{N[:iN]} sont vides donc \pyv{S} (vide) contient bien la somme des éléments de \pyv{M[:iM]} et \pyv{N[:iN]}.
  \item[Ligne 14] Supposons que tous les éléments de \pyv{M[:iM]} et \pyv{N[:iN]} ont été ajoutés (leur somme se trouve dans \pyv{S}). 
    Supposons de plus que tous les éléments de \pyv{M[:iM]} et \pyv{N[:iN]} sont strictement plus petits que tous les éléments de \pyv{M[iM:]} et \pyv{N[iN:]}.
    On a trois cas possibles.
    \begin{itemize}
      \item Le prochain élément à ajouter est \pyv{N[iN]}, à une case contenant un zéro de $M$.
      \item Le prochain élément à ajouter est \pyv{M[iM]}, à une case contenant un zéro de $N$.
      \item Le prochain élément à ajouter correspond au cas où l'on ajoute deux cases non nulle de $M$ et de $N$, de mêmes indices : $M[iM]$ et $N[iN]$. 
    \end{itemize}
  \item[Lignes 15-17] On est dans le premier cas, soit $iM = m$, dans ce cas $M[iM:]$ est vide, soit $N[iN]$ est strictement plus petit que $M[iM]$ lexicographiquement. 
    Dans les deux cas, le prochain élément à ajouter est $N[iN]$ (avec un zéro de $M$) et l'invariant de la ligne 14 est vérifié en $(iM,iN+1)$.
  \item[Ligne 18] On a donc $iM < m$ et, si $iN < n$, on a $M[iM] \leq N[iN]$ lexicographiquement. 
  \item[Lignes 18-20] On est dans le deuxième cas, soit $iN = n$, dans ce cas $N[iN:]$ est vide, soit $M[iM]$ est strictement plus petit que $N[iN]$ lexicographiquement. 
    Dans les deux cas, le prochain élément à ajouter est $M[iM]$ (avec un zéro de $N$) et l'invariant de la ligne 14 est vérifié en $(iM+1,iN)$.
  \item[Ligne 21] On a donc $iM < m$ et $iN < n$, avec $M[iM][:2]  = N[iN][:2]$. On est donc dans le troisième cas, on additionne deux cases non nulles de $M$ et de $N$ de mêmes indices.
  \item[Lignes 22-23] Si la somme des cases est non nulle, on la rajoute à $S$, sinon cette case  \og disparaît \fg\ . 
  \item[Lignes 24-25] L'invariant de la ligne 14 est vérifié en ($iM+1$,$iN+1$).
  \item[Ligne 26] \`A la sortie de la boucle \pyv{while}, on a donc $m = iM$ et $n = iN$, donc $M$ et $N$ ont bien été ajoutés dans $S$. Le résultat donné est donc correct. 
  \item[Variant] Dans tous les cas, la quantité $m+n-iM-iN$ a diminué d'au moins 1 (et d'au plus deux). De plus, $iM < m$ ou $iN < n$ si et seulement si $m+n-iM-iN > 0$, donc la boucle \pyv{while} termine. 
\end{description}

\question\ 
On considère le modèle de complexité usuel. 

Comptons les opérations effectuées, on donne pour chacune une domination asymptotique du nombre d'opérations élémentaires effectuées, en fonction de $n$. 
\begin{itemize}
  \item Un calcul de taille de tableau par la fonction \pyv{len} : $O(1)$. 
  \item Création d'un tableau de taille $n$ : $n$ fois le temps de création de chaque élément de $S$. Chaque élément de $S$ est créé comme un tableau de taille $n$ et se fait en temps $O(n)$. 
    Ainsi, $S$ est créé en temps $O(n)$. 
  \item Une création d'objet \pyv{range} : $O(1)$. 
  \item Dans chaque tour de la boucle indicée par \pyv{i} ($n$ tours en tout) : 
    \begin{itemize}
      \item Une création d'objet \pyv{range} : $O(1)$. 
      \item Dans chaque tour de la boucle indicée par \pyv{j} ($n$ tours en tout) : deux lectures dans un tableau, une addition de nombres, une écriture dans un tableau : $O(1)$
    \end{itemize}
\end{itemize}
Ainsi, le nombre d'opérations élémentaires est en $O(n^2)$. Remarquons que l'on ne pouvait pas faire mieux (asymptotiquement) : chaque case de $M$ doit être parcourue et $M$ possède $n^2$ cases. 

Notons $A_n$ le nombre d'opérations élémentaires effectuées pour calculer \pyv{add(M,N)}. Il existe donc une constante $K$ telle que, pour tout $n\in\N$, 
\begin{equation*}
  n^2 \leq A_n \leq K\times n^2.
\end{equation*}

\question\ 
On considère le modèle de complexité usuel. 

Comptons les opérations effectuées, on donne pour chacune une domination asymptotique du nombre d'opérations élémentaires effectuées, en fonction de $p = \max(m,n)$.
\begin{itemize}
  \item Remarquons qu'un appel de \pyv{lexico} se fait en temps $O(1)$. 
  \item On a deux calculs de longueurs par la fonction \pyv{len} et cinq affectations : $O(1)$.
  \item Chaque tour dans la boucle \pyv{while} se fait en $O(1)$ : au plus $5$ comparaisons, $2$ appels à \pyv{lexico}, $12$ accès à des tableaux, $1$ ajout à une liste et deux incrémentations. 
  \item Comme, à chaque tour de boucle, $m+n-iM-iN$ est réduit de $1$ ou deux, on a au plus $m+n\leq 2p$ tours de boucle. 
\end{itemize}
Ainsi, le nombre d'opérations élémentaires est en $O(p^2)$. Remarquons que l'on ne pouvait pas faire mieux (asymptotiquement) : chaque case non vide de $M$ (et de $N$) doit être parcourue et $M$ (ou $N$) possède $p$ cases. 

Notons $C_p$ le nombre d'opérations élémentaires effectuées pour calculer \pyv{add_creuse(M,N)}. Il existe donc une constante $K'$ telle que, pour tout $p\in\N$, 
\begin{equation*}
  p \leq C_p \leq K'\times p.
\end{equation*}

\question\ 
On discute ici de complexités asymptotiques, or nous avons deux variables sur lesquelles jouer : $n$ et $p$. Supposons que $p$ varie en fonction de $n$ (on le note $p(n)$). 
Notamment, $\dfrac{p(n)}{n^2}$ est la proportion de cases non vides de $M$ et de $N$.

Avec la constante $K'$ trouvée précédemment,
\begin{equation*}
  \forall n \in \N, \dfrac{C_{p(n)}}{A_n} \leq K' \times \dfrac{p(n)}{n^2}.
\end{equation*}
Ainsi, dès que $\dfrac{p(n)}{n^2} \leq \dfrac{1}{K'}$, on a $C_{p(n)} \leq A_n$. 
Il existe donc une proportion de cases non vides en deçà de laquelle il est toujours préférable d'utiliser la représentation creuse des matrices.

De même, il existe une proportion de cases non vides au dessus de laquelle il est toujours préférable d'utiliser la représentation classique des matrices. 

Notre analyse ne nous permet pas d'être plus précis. 