Beaucoup de problèmes technologiques actuels mettent en jeu des données de grandes dimensions et font souvent intervenir de grandes matrices.

Beaucoup de ces matrices sont \emph{creuses}, c'est-à-dire qu'elles ne contiennent que peu de valeurs non nulles. Nous allons développer une méthode de codage de telles matrices et étudier leur intérêt. 

Dans tout ce problème, on s'interdira d'utiliser les opérations entre vecteurs et matrices \texttt{numpy} (méthode \texttt{.dot()} par exemple). 

Dans tout ce problème, on utilisera le type \texttt{array} de \texttt{numpy} pour représenter des vecteurs. 
On pourra supposer que la ligne suivante a été écrite : 
\begin{pyverbatim}
from numpy import array, zeros
\end{pyverbatim}

Dans tout ce problème, on veillera à l'optimalité des fonctions écrites en termes de complexité temporelle asymptotique. Les réponses clairement sous-optimales seront pénalisées. 

\bigskip{}

Soit une matrice réelle $A \in \mathcal{M}_{n,p}(\R)$ de dimension $n \times p$ et possédant $s$ coefficients non nuls. Ce coefficient $s$ est appelé \emph{niveau de remplissage} de la matrice $A$. 
Comme toujours en \python{}, on numérotera les lignes et les colonnes en partant de $0$. 

Une telle matrice se code usuellement comme un tableau de nombres de dimension $n\times p$. 

Le codage CSR (Compress Sparse Raw) code la matrice $A$ par trois listes $V$, $L$ et $C$ définies comme suit.

%Le codage CSR (Compress Sparse Raw) code la matrice $A$ par trois listes $u$, $v$ et $w$ définies comme suit.
\begin{itemize}
    \item La liste $V$ contient toutes les valeurs des coefficients non nuls de $A$, en les listant ligne par ligne (de la première à la dernière ligne puis de la première à la dernière colonne). Ainsi, $V$ est de longueur $s$. 
    \item La liste $L$ est de longueur $n+1$, $L_0$ vaut toujours $0$ et pour chaque $1\leq i \leq n$, $L_i$ vaut le nombre de coefficients non nuls dans les $i$ premières lignes de $A$ (\emph{i.e.} de la ligne d'indice $0$ à celle d'indice $i-1$, inclu). 
    \item La liste $C$ contient les numéros de colonne de chaque coefficients non nuls de $A$, en les listant ligne par ligne (de la première à la dernière ligne puis de la première à la dernière colonne). Ainsi, $C$ est de longueur $s$.
\end{itemize}
Par exemple, avec
\begin{equation*}
    A = \begin{pmatrix} 0 & -1 & 0 & 0 \\ 2 & 0 & 4 & 0 \\ 0 & 0 & 0 & -1  \end{pmatrix},
\end{equation*}
on a $n = 3$, $p = 4$, $s = 4$ et sa représentation CSR est donnée par les trois listes 
\begin{align*}
    V &= [-1,2,4,-1], \\
    L &= [0,1,3,4], \\
    C &= [1,0,2,3].
\end{align*}
On remarquera qu'ajouter une colonne nulle à droite de $A$ ne change pas sa représentation CSR. 

\bigskip{}

\question{} Déterminer la représentation CSR de la matrice 
\begin{equation*}
    A = \begin{pmatrix} 1 & 0 & 3 \\ 0 & 0 & 0 \\ 0 & -2 & 0 \\ 0 & 0 & 0 \end{pmatrix}.
\end{equation*}
\question{} Donner une matrice dont la représentation CSR est 
\begin{align*}
    V &= [2,-3,1,1], \\
    L &= [0,0,2,4], \\
    C &= [0,3,2,3].
\end{align*}
\question{} Soit $(V,L,C)$ la représentation CSR de $A$, soit $0 \leq i \leq n-1$. Combien y a-t-il de coefficients non nuls sur la ligne \no$i$ de $A$ ? Donner deux expressions \python{} (en fonction de $V$, $L$, $C$ et $i$) permettant d'obtenir respectivement la liste des valeurs de ces coefficients et la liste des indices colonnes de ces coefficients. 

\bigskip{}

On rappelle la formule du produit matriciel : pour une matrice $A \in \mathcal{M}_{n,p}(\R)$ de coefficients $a_{i,j}$ et un vecteur $X \in \mathcal{M}_{p,1}(\R)$ de coefficients $x_j$, si $0 \leq i \leq n-1$, alors la $i\ieme$ coordonnée de $AX$ est 
\begin{equation*}
    \sum_{j=0}^{p-1} a_{i,j} x_j.
\end{equation*}

\bigskip{}

\question{} Écrire une fonction \texttt{coeff\_prod(V,L,C,X,i)} renvoyant la $i\ieme$ coordonnée du produit $AX$, où le triplet $(V,L,C)$ est la représentation CSR de $A$. 
On supposera que les dimension de $A$ et $X$ sont compatibles pour effectuer le produit $AX$.

Donner la complexité asymptotique de cette fonction, en fonction de $n$, $p$ et $\ell_i$, où $\ell_i$ est le nombre d'éléments non nuls sur la $i\ieme$ ligne de $A$. 

\question{} Écrire une fonction \texttt{prod(V,L,C,X)} renvoyant le produit $AX$, où le triplet $(V,L,C)$ est la représentation CSR de $A$.  
On supposera que les dimension de $A$ et $X$ sont compatibles pour effectuer le produit $AX$.

Donner la complexité asymptotique de cette fonction, en fonction de $n$, $p$ et $s$. 

\question{} Écrire une fonction \texttt{prod\_naif(A,X)} renvoyant le produit $AX$, où $A$ est codée usuellement comme un tableau à double dimension.  

Donner la complexité asymptotique de cette fonction, en fonction de $n$, et $p$.

\question{} En les comparant, discuter des deux complexités précédentes, notamment en fonction du niveau de remplissage $s$. 

\bigskip{}

On prend maintenant pour exemple celui de la dérivation discrète, typique en traitement du signal. Pour un vecteur de longueur $n$
\begin{equation*}
    X = \begin{pmatrix} x_0 \\ \vdots \\ x_{n-1} \end{pmatrix}
\end{equation*}
on définit la dérivée discrète de $X$ comme le vecteur $Y$ de longueur $n$ définit par : 
\begin{equation*}
    y_0 = x_1-x_0,~y_{n-1} = x_{n-1} - x_{n-2}~\textrm{et}~\forall i \in \ii{1,n-1},~ y_i = \dfrac{1}{2}(x_{i+1}-x_{i-1}).
\end{equation*}

\question{} Déterminer une matrice $A$ vérifiant $Y = AX$. 

\question{} Écrire une fonction \texttt{CSR\_A(n)} renvoyant les trois vecteurs \texttt{V}, \texttt{L} et \texttt{C} codant $A$ au format CSR.

\emph{Indication : on pourra écrire une fonction pour la création de chaque vecteur}
