  On considère la suite $u$ à valeurs dans $\ii{0;64~007}$ définie par 
  \begin{equation*}
    u_0 = 42,\quad \forall n \in\N,~ u_{n+1} = 15~091 u_n ~[64~007], 
  \end{equation*}
  ainsi que, pour tout $n\in\N$, $S_n = \displaystyle\sum_{k=0}^n u_k$.
  
  On propose l'algorithme suivant pour calculer les valeurs de $S$. 
\begin{pyverbatim}
def u(n):
    """u_n, n : entier naturel"""
    v = 42
    # Inv : v = u_0
    for k in range(n):
        # Inv : v = u_k
        v = 15091 * v % 64007
        # Inv : v = 15091*u_k % 64007 = u_k+1
    # Inv : au dernier tour, k = n-1, donc v = u_n
    return v
    
def S(n):
    """u_n, n : entier naturel"""
    s = u(0)
    # Inv : s = S_0
    for k in range(n):
        # Inv : s = S_k
        s = s + u(k+1)
        # Inv : v = S_k+u_k+1 = S_k+1
    # Inv : au dernier tour, k = n-1, donc s = S_n
    return s
\end{pyverbatim}
\begin{enumerate}
  \item Étudier les complexités des fonctions \texttt{u} et \texttt{S}, en fonction de $n$.
  \item Écrire une fonction donnant la valeur de $S_n$ en temps $O(n)$. 
\end{enumerate}