\textbf{Importation des modules nécessaires : }

\begin{lstlisting}
import scipy.optimize as so
import scipy.integrate as si
from math import sqrt, sin, cos, atan
from numpy import array
import numpy as np
\end{lstlisting}



\question{} Donner une approximation de $\int_{\alpha}^{\alpha+1} \cos(\sqrt{t})\,dt$ avec la méthode des trapèzes et 1000 subdivisions.

\begin{lstlisting}
def trapeze(f,a,b,n):
    h=(b-a)/n
    S=0.5*(f(a)+f(b))
    for k in range(1,n):
        S+=f(a+k*h)
    return S*h
    
print("Qu. 1 : ", trapeze(lambda x : cos(sqrt(x)),alpha,alpha+1,1000))
\end{lstlisting}


\question{}  Donnez une approximation (à $10^-5$ près) de l'unique réel positif solution de
  l'équation $x^{2}+\sqrt{x}-10 = \alpha$ avec la méthode de votre choix.
  
  
  
\begin{lstlisting}
def newton(f, fp, x0, epsilon):
    """Zéro de f par la méthode de Newton
       départ : x0, f' = fp, critère d'arrêt epsilon"""
    u = x0
    v = u - f(u)/fp(u)
    k=0
    while abs(v-u) > epsilon:
        u, v = v, v - f(v)/fp(v)
        k+=1
    return u,k

def f2(t):
    return t**2+t**0.5-10-alpha

def f2p(t):
    return 2*t+0.5*t**(-0.5)
    
print("Qu. 2 : méthode de Newton ", newton(f2,f2p,1,1e-5)[0])
\end{lstlisting}
  
\question{}
  Donnez le nombre d'itérations nécessaires pour obtenir ce résultats avec la méthode de Newton en prenant pour valeur initiale $\alpha$.
  
\begin{lstlisting}
print("Qu. 3 : méthode de Newton ", newton(f2,f2p,alpha,1e-5)[1])
\end{lstlisting}
  
  
  
\question{}
  Donnez le nombre d'itérations nécessaires pour obtenir ce résultats avec la méthode de Dichotomie sur l'intervalle $\left[0,12+\alpha\right]$.

\begin{lstlisting}
def dichotomie(f, a, b, epsilon):
    """Zéro de f sur [a,b] à epsilon près, par dichotomie
       Préconditions : f(a) * f(b) <= 0
                       f continue sur [a,b]
                       epsilon > 0"""
    c, d = a, b
    fc, fd = f(c), f(d)
    k=0
    while d - c > 2 * epsilon:
        m = (c + d) / 2.
        fm = f(m)
        if fc * fm <= 0:
            d, fd = m, fm
        else:
            c, fc = m, fm
        k+=1
    return (c + d) / 2.,k
    
print("Qu. 4 : méthode de dichotomie ", dichotomie(f2, 0, 12+alpha, 1e-5)[1])

\end{lstlisting} 
  


\question{}
  Donnez à l'aide une approximation (à $10^-5$ près) de l'unique réel positif $t$ tel que
  \begin{equation*}
    \int_{\alpha}^{\alpha+t} (2+\sqrt{x}+\cos x)\, dx = 10
  \end{equation*}

  
\begin{lstlisting}
print("Qu. 5 : ", dichotomie(lambda t : trapeze(lambda x : 2+sqrt(x)+cos(x),alpha,alpha+t,1000)-10,0,50,1e-5))
\end{lstlisting}
  

\question{}
  Donner une valeur approchée de $x(1)$ avec $x$ l'unique fonction
  vérifiant $x(0)=\alpha$ et pour tout $t\in \R$, $x'(t) = 3\cos(x(t))
  + t$.

\begin{lstlisting}
def F (x,t) :
    return 3*cos(x) + t

les_t = [i/10000 for i in range(10001)]

print("Qu. 6 : ", si.odeint(F,alpha,les_t)[-1,0])
\end{lstlisting}
  

\question{}
  Donner une valeur approchée de $x(1+\frac{\alpha}{10})$ avec $x$ l'unique fonction
  vérifiant $x(0)=0$, $x'(0)=0$ et pour tout $t\in \R$, $x''(t) = 1 + \sin(t+x(t))$.

\begin{lstlisting}
def G (X,t) :
    a,b = X[0],X[1]
    return array([b,1+sin(t+a)])

les_t = [i*(1+alpha/10)/10000 for i in range(10001)]

print("Qu. 7 : ", si.odeint(G,array([0,0]),les_t)[-1,0])

\end{lstlisting}
  
\question{}
  Donner une approximation de $\beta\in \R$, pour que l'unique
  solution de l'équation différentielle non linéaire $x''(t)=
  1+\arctan(t+x(t))$ avec les conditions initiales $x(0)=0$ et
  $x'(0)=\beta$ vérifie
  \begin{equation*}
    x(1+\frac{\alpha}{10}) = 1 + \frac{2}{3}\alpha
  \end{equation*}
  
  
\begin{lstlisting}
def H (X,t) :
    a,b = X[0],X[1]
    return array([b,1+atan(t+a)])

les_t = [i*(1+alpha/10)/10000 for i in range(10001)]

f = lambda beta : si.odeint(H,array([0,beta]),les_t)[-1,0]-1-(2/3)*alpha

print("Qu. 8 : ",so.brentq(f,-2,2))

\end{lstlisting}
  
