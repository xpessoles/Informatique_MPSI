\begin{obj}
On souhaite déterminer l’évolution d’une population d’une proie en fonction de celle de son prédateur.
\end{obj}

On suppose un milieu où existe une population « u » de proies (lapins) interagissant avec une unique population « v » de prédateurs (renards).

\subsection*{Modèle sans prédation}

Sans prédateur, l'évolution du nombre de proies est donné par l'équation différentielle suivante : $\dfrac{\dd u(t)}{\dd t} = a \times u(t)$ avec $a$ le taux re reproduction des proies. 

On prendra  $u_0 = \alpha$ et $a=\alpha 10^{-2}$. 

\question{Donner la population de lapins après 20 unités de temps.}


\subsection*{Modèle avec prédation}
Le modèle avec prédateur est donné par : 
$\left\{
\begin{array}{l}
u'(t)=u(t)\left(a - b\times v(t)\right)
v'(t)=-v(t)\left(c - d\times u(t)\right)
\end{array}
\right.
$
avec :
\begin{itemize}
\item $a $: taux de reproduction des proies;
\item $b$ : taux de mortalité des proies à cause des prédateurs;
\item $c$ : taux de mortalité des prédateurs;
\item $d$ : taux de reproduction des prédateurs.
\end{itemize}

On donne $u_0=\alpha$, $a=\alpha 10^{-2}$, $b=\alpha 10^{-3}$ ainsi que 
$v_0=\alpha+10$, $c=\alpha 10^{-2}$, $d=\dfrac{\alpha}{5} \times  10^{-3}$.


\question{Donner la population de lapins après 300 unités de temps.}


\question{Donner la population de renards après 300 unités de temps.}