\question{}
 
 Problème de cinétique chimique: avec trois réactifs, $A$, $B$ et
$C$. $A$ se transforme en $B$ qui se transforme en $C$ (réactions d'ordre $1$). 
Les concentrations de $A$, $B$ et $C$ suivent donc les équations suivantes.
\begin{align*}
  \frac{\mathrm{d}[A]}{\mathrm{d}t}&=-\alpha[A]\\
\frac{\mathrm{d}[B]}{dt}&=\alpha[A]-\beta[B]\\
\frac{\mathrm{d}[C]}{\mathrm{d}t}&=\beta[B]
\end{align*}

On supposera $\alpha=1\text{s}^{-1}$, $\beta=10\text{s}^{-1}$ et au
temps $t=0$, $[A] = 1\text{mol}/\text{l}$, $[B]=[C]=0$. On veut
regarder l'évolution
jusqu'à $t=6\text{s}$.

\question{}
Tracer l'évolution des concentrations des réactifs, par la méthode d'Euler et avec scipy.