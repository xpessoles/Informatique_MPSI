En utilisant \python{} on peut tracer de nombreux types de graphiques. 
Nous allons utiliser une bibliothèque regroupant de (très) nombreuses fonctions de tracé : \pyv{matplotlib}.
En fait, cette bibliothèque est bien trop vaste, nous n'utiliserons que sa sous-bibliothèque \pyv{matplotlib.pyplot}.

Commencez par ouvrir votre IDE, puis créez un script nommé \pyv{tp05_ex_sin.py}. Recopiez dedans le script suivant. 
\begin{pyverbatim}
import matplotlib.pyplot as plt
from numpy import sin

n = 20
x = [k*10/n for k in range(n)]
y = [sin(t) for t in x]

plt.clf()
plt.plot(x,y,label='sin(x)')
plt.xlabel('x')
plt.legend(loc=0)
plt.title('Tracé du sinus sur [0,10]')
plt.savefig('tp05_ex_sin.png')
\end{pyverbatim}
\'Exécutez ce script et vérifiez que la figure créée est en tout point semblable à celle présente sur le site de classe. 

\medskip{}

\question{} Quel est le type de \pyv{x} ?

\medskip{}

\question{} Comment \python{} représente-t-il graphiquement une fonction ? 

\medskip{}

\question{} Où le tracé s'arrête-t-il ? Pourquoi ? 

\medskip{}

On rappelle que l'on peut simplement créer une liste d'abscisses en utilisant la fonction \pyv{linspace} de la bibliothèque de calcul \pyv{numpy}. 

Chargez cette fonction dans l'interpréteur interactif par la commande 
\begin{pyverbatim}
from numpy import linspace
\end{pyverbatim}
puis consultez son manuel par la commande \pyv{help(linspace)}. 

\medskip{}

\question{}\label{tp05:qu:sin2} Écrire une fonction \pyv{ex_sin(nom_de_fichier)} permettant tracer de manière plus appropriée la courbe de l'exemple précédent et qui enregistre l'image produite dans le fichier \pyv{nom_de_fichier}. 
Vous produirez alors une image, que vous enverrez à votre enseignant. 

\emph{Indication :} Attention au type de \pyv{nom_de_fichier} ! Notamment, on supposera que l'extension du fichier est déjà présente dans \pyv{nom_de_fichier}.

\medskip{}

\question{}\label{tp05:qu:transitoire} \'Ecrire une fonction \pyv{transitoire(A,nom_de_fichier)} qui enregistre dans le fichier \pyv{nom_de_fichier} le graphe des fonctions 
\begin{equation*}
  t\mapsto A\p{1-\e^{-\sfrac{t}{\tau}}}
\end{equation*}


sur $[0,10]$, pour chaque $\tau\in\left\{\dfrac{1}{2};1;2;4;8\right\}$. 
Vous produirez une image (vous choisirez $A$), que vous enverrez à votre enseignant. 

\emph{Indication :} Attention au type de \pyv{nom_de_fichier} !