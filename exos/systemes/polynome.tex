%\subsection{Résolution des systèmes avec NumPy.}

On rappelle le principe de la méthode des moindres carrés.

On souhaite déterminer une parabole passant les points d'équation $y=a\,x^2+b\,x+c$ de la parabole passant par les points $(-1,9)$, $(1,3)$ et $(2,3)$.

\question{} Poser le système d'équations à résoudre.

\question{} Déterminer les coefficients $a$, $b$ et $c$ par l'utilisation du pivot de Gauss.

\question{} Sur un même graphe faire apparaître les 3 points ainsi que la paraboles obtenue par interpolation.