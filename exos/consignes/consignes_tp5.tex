Attention : suivez précisément ces instructions. 
Votre fichier portera un nom du type 
\begin{center}
  \texttt{tp05\_durif\_kleim.py},
\end{center}
 où les noms de vos enseignants sont à remplacer par ceux des membres du binôme. Le nom de ce 
fichier ne devra comporter ni espace, ni accent, ni apostrophe, ni majuscule.
Dans ce fichier, vous respecterez les consignes suivantes.
\begin{itemize}
  \item \'Ecrivez d'abord en commentaires (ligne débutant par \#), le titre du TP, les noms et prénoms des étudiants du groupe.
  \item Commencez chaque question par son numéro écrit en commentaires.
  \item Les questions demandant une réponse écrite seront rédigées en commentaires.
  \item Les questions demandant une réponse sous forme de fonction ou de script respecteront pointilleusement les noms de variables et de fonctions demandés.
\end{itemize}
Les figures demandées porteront toutes un nom du types \texttt{tp05\_durif\_kleim\_num.png}, où les noms de vos enseignants sont à remplacer par ceux des membres du binôme et où
\begin{itemize}
  \item \texttt{num} vaut \texttt{q04} pour la question~\ref{tp05:qu:sin2} ;
  \item \texttt{num} vaut \texttt{q05} pour la question~\ref{tp05:qu:transitoire} ;
  \item \texttt{num} vaut \texttt{q08} pour la question~\ref{tp05:qu:creneau} ;
  \item \texttt{num} vaut \texttt{q11} pour la question~\ref{tp05:qu:triangle} (question facultative).
\end{itemize}