%\section*{Fonctionnement du devoir}

\begin{enumerate}
\item Ce devoir est à réaliser seul, en utilisant SQLite3, soit en utilisant sqlite db browser soit en utilisant Python avec le module sqlite3.
\item Nous vous conseillons de commencer par créer un dossier au nom du DS dans le répertoire dédié à l'informatique sur votre PC. 
% \item La documentation Python officielle est disponible à l'adresse suivante. 
% \begin{center}
%  file:///usr/share/doc/python3.4/html/index.html 
% \end{center}
%\item Nous vous rappelons qu'il est possible d'obtenir de l'aide dans l'interpréteur d'idle en tapant \\
%\texttt{help(nom\_fonction)}.
%\item Vous inscrirez vos réponses sur la feuille réponse fournie. Attention : lisez attentivement le paragraphe suivant.



\item Vos réponses dépendent d'un paramètre $\alpha$, unique pour chaque étudiant, qui vous a déjà été donné.



\item Vous trouverez un fichier \texttt{hotel\_}$\alpha$\texttt{.db} reçu par mail ou sur le site de classe.
Enregistrez la base de données portant votre numéro $\alpha$ dans le dossier du devoir. 

%\medskip{}
%
%Si vous êtes un utilisateur de Windows et si sqlitebrowser n'est pas installé sur votre ordinateur, vous trouverez un fichier \texttt{SQLiteDatabaseBrowserPortable\_3.10.1\_English.paf.exe} sur le site de classe ainsi qu'à l'adresse suivante (où \texttt{X} 
%est à remplacer par \texttt{1} ou \texttt{2}).
%\begin{center}
%  $\sim$\texttt{/groupes/mpsX/données/d04s/SQLiteDatabaseBrowserPortable\_3.10.1\_English.paf.exe}
%\end{center}
%Enregistrez ce fichier sur votre ordinateur, puis double-cliquez dessus. Installez sqlitebrowser dans votre dossier «Mes Documents». Vous devriez normalement avoir ainsi une version fonctionnelle de sqlitebrowser.
%
%\medskip{}



\item Ouvrir cette base de donnée avec DB Browser for SQLite.
\item Tester vos requêtes sql dans ce logiciel.
\item Les réponse seront à compléter dans le fichier.py fourni avec le sujet (DS08$\_$eleve.py). 
\item Le sujet sera renommé en \texttt{DS\_08\_nom.py}.
\item Dans ce fichier on vous demandera de saisir vos requêtes ou résultats ou les deux.

\end{enumerate}

