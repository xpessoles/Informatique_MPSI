\question{}
\begin{pyverbatim}
def reste_a_payer(p,t,m,d):
    """p = montant du pret en euros
       t = taux mensuel
       m = mensualites
       d = duree en annees
       Calcule le montant restant a payer a l'echeance   du pret"""
    dette = p
    for mois in range(d*12):
        # Inv : dette est dû au début du mois
        dette = dette*(1+t)-m
    return dette
    
\question{}    
    
def somme_totale_payee(p,t,m,d):
    """p = montant du pret
       t = taux
       m = mensualites
       d = duree en annees
       Calcule le montant total paye"""
    return reste_a_payer(p,t,m,d) + 12*d*m
    
\question{}

def cout_total(p,t,m,d):
    """p = montant du pret
       t = taux
       m = mensualites
       d = duree en annees
       Calcule le cout total du credit"""
    return somme_totale_payee(p,t,m,d) - p
\end{pyverbatim}

\question{}

\begin{pyverbatim}
def duree_mensualite(p,t,m):
    """Durée du prêt
       p = montant prêté
       t = taux mensuel
       m = mensualité"""
    emprunt = p
    d = 0
    while (1+t)*emprunt >= m:
        # Inv : emprunt est dû au début du mois d
        d = d+1
        emprunt = (1+t)*emprunt-m
    return d
\end{pyverbatim}

\question{}

\question{}