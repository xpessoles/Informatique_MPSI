On considère la fonction suivante.
\begin{Verbatim}[gobble=0,numbers=left]
def mystere(L) :
    """Précondition : L est une liste de nombres"""
    x,n,i = L[0],len(L),1
    while i<n and x > L[i] :
        L[i-1],L[i] = L[i],L[i-1]
        i = i+1
    return None
\end{Verbatim}

\question{}Montrer que «$\texttt x = \texttt L[\texttt i-1]$» est un invariant de boucle pour la boucle \texttt{while} de la fonction \texttt{mystere}.

\question{}Donner un variant de boucle pour la boucle \texttt{while} de la fonction \texttt{mystere}. Que peut-on en déduire ? 

\question{}Si \texttt{L} est une liste de nombres, que fait \texttt{mystere(L)} ? Le justifier, notamment à l'aide des questions précédentes (vous pourrez cependant écrire un ou plusieurs autres invariants, au besoin).

\bigskip{}

\emph{Remarque :} vous avez tout intérêt à utiliser cette fonction et à observer son fonctionnement \emph{avant} de répondre à ces questions. 


