\question{}
\begin{lstlisting}
def somme1(n):
    """n = entier naturel.
       Calcule la somme des 1/(i+j**2), pour 1<=i,j<=n"""
    s = 0
    # Inv : s = somme des 1/(k+l**2) pour 1<=k<=0 et 1 <= l <= n (somme vide)
    for i in range(1,n+1):
        # Inv : s = somme des 1/(k+l**2) pour 1<=k<=(i-1) et 1 <= l <= n 
        for j in range(1,n+1):
            # Inv : s = somme des 1/(k+l**2) pour 1<=k<=(i-1) et 1 <= l <= n 
            #           + somme des 1 / (i + l**2) pour 1 <= l <= j-1
            s = s + 1 / (i + j**2)
            # Inv : s = somme des 1/(k+l**2) pour 1<=k<=(i-1) et 1 <= l <= n 
            #           + somme des 1 / (i + l**2) pour 1 <= l <= j
        # Au dernier tour de boucle, j = n, donc 
	# Inv : s = somme des 1/(k+l**2) pour 1<=k<=i et 1 <= l <= n
    # Au dernier tour de boucle, i = n, donc 
    # s = somme des 1/(k+l**2) pour 1<=k<=n et 1 <= l <= n 
    return s
\end{lstlisting}

\begin{lstlisting}
def somme2(n):
    """n = entier naturel.
       Calcule la somme des 1/(i+j**2), pour 1<=i<j<=n"""
    s = 0
    # Inv : s = somme des 1/(k+l**2) pour 1<=k<=0 et k < l <= n (somme vide)
    for i in range(1,n+1):
        # Inv : s = somme des 1/(k+l**2) pour 1<=k<=i et i < l <= n 
        for j in range(i+1,n+1):
            # Inv : s = somme des 1/(k+l**2) pour 1<=k<=i-1 et k < l <= n 
            #           + somme des 1 / (i + l**2) pour i+1 < l <= j-1 
            s = s + 1 / (i+j**2)
            # Inv : s = somme des 1/(k+l**2) pour 1<=k<=i-1 et k < l <= n 
            #           + somme des 1 / (i + l**2) pour i+1 < l <= j 
        # Au dernier tour de boucle, j = n, donc 
	# Inv : s = somme des 1/(k+l**2) pour 1<=k<=i et k < l <= n
    # Au dernier tour de boucle, i = n, donc 
    # s = somme des 1/(k+l**2) pour 1<=k<=n et k < l <= n 
    return s
\end{lstlisting}