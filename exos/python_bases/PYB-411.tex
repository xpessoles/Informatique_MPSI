\question{Indenter de deux manières différentes la suite d'instructions suivante afin que la variable \pyv{t} contienne \pyv{True} pour une indentation, puis \pyv{False} pour l'autre. }
%Le tester avec IDLE.
\begin{pyverbatim}
x = 0
y = 5
t = False
if x>=1:
t = True
if y <= 6:
t = True
\end{pyverbatim}

\question{Réécrire la suite d'instructions suivante de manière plus appropriée.}
\begin{pyverbatim}
from random import randrange
# Un entier aléatoire entre 0 et 99
n = randrange(100) 
if n <= 10:
    print("Trop petit")
else:
    if n >= 50:
        print("Trop grand")
    else:
        print("Juste comme il faut")
\end{pyverbatim}


\question{Que fait la fonction suivante ? La corriger pour qu'elle coïncide avec le but annoncé. }

\begin{pyverbatim}
def inv(n):
    """Somme des inverses des n premiers 
       entiers naturels non nuls"""
    s = 0
    for k in range(n):
        x = 1/k
    s = s+x
    return s
\end{pyverbatim}