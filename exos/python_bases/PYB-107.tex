% \question 
\begin{enumerate}[label = \emph{\alph*)}]
  \item Affecter à \pyv{v} la liste \pyv{[2,5,3,-1,7,2,1]}
  \item Affecter à \pyv{L} la liste vide.
  \item Vérifier le type des variables crées.
  \item Calculer la longueur de \pyv{v}, affectée à \pyv{n} et celle de \pyv{L}, affectée à \pyv{m}.
  \item Tester les expressions suivantes : \pyv{v[0]}, \pyv{v[2]}, \pyv{v[n]}, \pyv{v[n-1]}, \pyv{v[-1]} et \pyv{v[-2]}.
  \item Changer la valeur du quatrième élément de \pyv{v}.
  \item Que renvoie \pyv{v[1:3]} ? Remplacer dans \pyv{v} les trois derniers éléments par leurs carrés.
  \item Que fait \pyv{v[1] = [0,0,0]} ? Combien d'éléments y a-t-il alors dans \pyv{v} ?
\end{enumerate}