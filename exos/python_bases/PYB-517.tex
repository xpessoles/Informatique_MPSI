On considère la fonction suivante. 
\begin{Verbatim}[gobble=0,numbers=left]
def mystere(a,b) :
    """Précondition : a,b sont des entiers, a>0, b>1"""
    k,p = 0,1
    while a % p == 0 :
        k = k+1
        p = p*b
    return k-1
\end{Verbatim}

\question{} Dresser un tableau de valeurs décrivant les valeurs des variables \pyv{k} et \pyv{p} en entrée des trois premiers tours de la boucle \texttt{while} de la fonction \pyv{mystere(a,b)}. 

On pourra au besoin faire intervenir les variables \pyv{a} et \pyv{b}. 

\medskip{}

\question{} En s'aidant de la question précédente, écrire un invariant de boucle pour la boucle \texttt{while} de la fonction \pyv{mystere(a,b)}. On justifiera la réponse. 

\medskip{}

\question{} Donner un variant de boucle pour la boucle \texttt{while} de la fonction \pyv{mystere(a,b)}. On justifiera la réponse. 

\medskip{}

\question{} Déduire des questions précédentes qu'un appel de la fonction \pyv{mystere(a,b)} renvoie un résultat et déterminer le résultat alors renvoyé. On justifiera la réponse. 
