% \question 
\begin{enumerate}[label = \emph{\alph*)}]
  \item Affecter à la variable  \pyv{mon_age} l'âge que vous aviez il y a 13 ans.
  \item \'Ecrire l'opération qui vous permet d'actualiser votre âge, tout en conservant la même variable.
  \item Que donne l'interpréteur après exécution des expressions suivantes ? Pourquoi ? 
\begin{pyverbatim}
mon-age = 18
2013_mon_age = 18
True = 18
\end{pyverbatim}
  \item \`A partir d'une nouvelle session d'IDLE, exécuter les expressions suivantes et commenter le résultat.
\begin{pyverbatim}
age = 5
age = Age + 14
\end{pyverbatim}
\end{enumerate}