\question Un banquier vous propose de vous prêter $p$ euros, à un taux de $12t\%$ par an ---  ce qui, dans  le langage commercial  des banquiers,
veut  dire $t\%$  par mois  --- avec des  mensualités de  $m$ euros. Autrement  dit,  vous   contractez  une  dette  de  $p$
euros. Chaque mois, cette dette  augmente de $t\%$ puis est diminuée du  montant  de  votre  mensualité. Lorsque votre dette, augmentée du taux, est inférieure à la mensualité, il suffit de régler le solde en une fois.

\'Ecrire une fonction \pyv{duree_mensualite(p,t,m)} renvoyant le nombre de mensualités nécessaires au remboursement total du prêt.

Attention : que se passe-t-il si la mensualité est trop petite ? 

\emph{Indice : dans le cas où le prêt est $\texttt{p}=4\times10^5$, le taux est $\texttt{t}=0,25\times10^{-2}$ et la mensualité est $\texttt{m}=1431,93$, on trouvera une durée de remboursement de $480$ mois.}
