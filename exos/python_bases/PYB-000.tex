\question{Évaluer les expressions suivantes en repérant auparavant celles qui donnent des résultats de type \py{int}.}%
\begin{multicols}{2}
  \begin{enumerate}[label=\emph{\alph*)}]
    \item \pyv{4+2}
    \item \pyv{25-3}
    \item \pyv{-5+1}
    \item \pyv{117*0}
    \item \pyv{6*7-1}
    \item \pyv{52*(3-5)}
    \item \pyv{5*(-2)}
    \item \pyv{22/(16-2*8)}
    \item \pyv{42/6}
    \item \pyv{18/7}
    \item \pyv{(447+3*6)/5}
    \item \pyv{0/0}
  \end{enumerate}
\end{multicols}%

%\begin{multicols}{4}
%  \begin{enumerate}[label=\emph{\alph*)}]
%    \item \py{4+2}
%    \item \py{25-3}
%    \item \py{-5+1}
%  \end{enumerate}
%\end{multicols}%

\question{Calculer les restes et les quotients des divisions euclidiennes suivantes : }
\begin{multicols}{2}
  \begin{enumerate}[label=\emph{\alph*)}]
    \item $127 \div 8$
    \item $54 \div 3$
    \item $58 \div 5$
    \item $58 \div (-5)$
    \item $-58 \div 5$
    \item $-58 \div (-5)$
    \item $17583 \div 10$
    \item $17583 \div 100$
    \item $17583 \div 10^4$
    \item $(2^7+2^4+2) \div 2^5$
    \item $(2^7+2^4+2) \div 2^7$
    \item $(2^7+2^4+2) \div 2^{10}$
  \end{enumerate}
\end{multicols}
\question{Calculer les nombres suivants avec une expression Python en repérant auparavant ceux qui donnent un résultat de type \py{int}.}
\begin{multicols}{2}
  \begin{enumerate}[label=\emph{\alph*)}]
    \item $3^5$
    \item $2^{10}$
    \item $(-3)^7$
    \item $-3^7$
    \item $5^{-2}$
    \item $7^{{5^4}}$
    \item ${7^5}^4$
    \item $5^{7+6}$
    \item $5^7+6$
    \item $2^{{10^4}}$
  \end{enumerate}
\end{multicols}



\question{Évaluer les expressions suivantes.}
\begin{multicols}{2}
  \begin{enumerate}[label=\emph{\alph*)}]
    \item \pyv{4.3+2}
    \item \pyv{2.5-7.3}
    \item \pyv{42+4.}
    \item \pyv{42+4}
    \item \pyv{42.+4}
    \item \pyv{12*0.}
    \item \pyv{11.7*0}
    \item \pyv{2,22/(1.6-2*0.8)}
    \item \pyv{42/6}
    \item \pyv{1,8/7}
    \item \pyv{(447+3*6)/5}
    \item \pyv{0/0}
  \end{enumerate}
\end{multicols}

\question{Calculer, sans utiliser la fonction \pyv{sqrt} ni la division flottante \pyv{/}, les nombres suivants.}
\begin{multicols}{2}
  \begin{enumerate}[label=\emph{\alph*)}]
    \item $\dfrac{1}{7,9}$
    \item $\sqrt{6,2}$
    \item $\dfrac{1}{\sqrt{3,5}}$
    \item $2\sqrt{2}$
  \end{enumerate}
\end{multicols}
De base, on ne peut réaliser que des calculs élémentaires avec Python. Cependant, il est possible d'avoir accès à des possibilités de calcul plus avancées en utilisant une \emph{bibliothèque}. 
Par exemple, la bibliothèque \pyv{math} permet d'avoir accès à de nombreux outils mathématiques. 
On peut donc saisir
\begin{pyverbatim}
from math import sin, cos, tan, pi, e
from math import sin, cos, tan, pi, e 
\end{pyverbatim}
pour avoir accès à toutes ces fonctions. \\

\question{Calculer les nombres suivants (on n'hésitera pas à consulter l'aide en ligne).}
\begin{multicols}{2}
  \begin{enumerate}[label=\emph{\alph*)}]
    \item $e^2$
    \item $\sqrt{13}$
    \item $\cos\p{\dfrac{\pi}{5}}$
    \item $e^{\sqrt{5}}$
    \item $\ln 2$
    \item $\ln 10$
    \item $\log_{2} 10$
    \item $\tan\p{\dfrac{\pi}{2}}$
  \end{enumerate}
\end{multicols}

\question{Les expressions suivantes sont-elles équivalentes ?}

\begin{multicols}{2}
  \begin{enumerate}[label=\emph{\alph*)}]
    \item \pyv{8.5 / 2.1}
    \item \pyv{int(8.5) / int(2.1)}
    \item \pyv{int(8.5 / 2.1)}
  \end{enumerate}
\end{multicols}

Et celles-ci ?

\begin{multicols}{2}
  \begin{enumerate}[label=\emph{\alph*)}]
    \item \pyv{float(8 * 2)}
    \item \pyv{8 * 2}
    \item \pyv{8. * 2.}
  \end{enumerate}
\end{multicols}

Prévoir la valeur des expressions suivantes puis vérifier cela (avec IDLE).

\begin{multicols}{2}
  \begin{enumerate}[label=\emph{\alph*)}]
    \item \pyv{1.7 + 1.3}
    \item \pyv{2 - 1}
    \item \pyv{2. - 1}
    \item \pyv{2 - 1.}
    \item \pyv{(2 - 1).}
    \item \pyv{.5 + .5}
    \item \pyv{4 / (9 - 3**2)}
    \item \pyv{4 / (9. - 3**2)}
  \end{enumerate}
\end{multicols}

\question{Déterminer de tête la valeur des expressions suivantes avant de le vérifier (avec IDLE).}

\begin{multicols}{2}
  \begin{enumerate}[label=\emph{\alph*)}]
    \item \pyv{0 == 42}
    \item \pyv{1 = 1}
    \item \pyv{3 == 3.}
    \item \pyv{0 != 1}
    \item \pyv{0 < 1}
    \item \pyv{4. >= 4}
    \item \pyv{0 !< 1}
    \item \pyv{2*True + False}
    \item \pyv{-1 <= True}
    \item \pyv{1 == True}
    \item \pyv{False != 0.}
    \item \pyv{True and False}
    \item \pyv{True or False}
    \item \pyv{True or True}
    \item \pyv{(2 == 3-1) or (1/0 == 5)}
    \item \pyv{(1/0 == 5) or (2 == 3-1)}
    \item \pyv{True or True and False}
    \item \pyv{False or True and False}    
    \item \pyv{not (1 == 1 or 4 == 5)}
    \item \pyv{(not 1 == 1) or 4 == 5}
    \item \pyv{not True or True}
  \end{enumerate}
\end{multicols}

\question{Dans votre IDE, cliquer sur File/New File. Une nouvelle fenêtre apparaît. Dans cette fenêtre, taper les lignes suivantes.}
\begin{pyverbatim}
3*2
print(2*3.)
17*1.27
\end{pyverbatim}
Enregistrer le document produit puis, toujours dans cette fenêtre, exécutez-le. 
Observez le résultat dans l'interpréteur interactif. 
Modifiez les instructions pour que tous les résultats de calcul s'affichent dans l'interpréteur interactif.
