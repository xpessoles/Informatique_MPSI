%\question
\'Evaluer les expressions suivantes en repérant auparavant celles qui donnent des résultats de type \py{int}.%
\begin{multicols}{4}
  \begin{enumerate}[label=\emph{\alph*)}]
    \item \pyv{4+2}
    \item \pyv{25-3}
    \item \pyv{-5+1}
    \item \pyv{117*0}
    \item \pyv{6*7-1}
    \item \pyv{52*(3-5)}
    \item \pyv{5*(-2)}
    \item \pyv{22/(16-2*8)}
    \item \pyv{42/6}
    \item \pyv{18/7}
    \item \pyv{(447+3*6)/5}
    \item \pyv{0/0}
  \end{enumerate}
\end{multicols}%

\begin{multicols}{4}
  \begin{enumerate}[label=\emph{\alph*)}]
    \item \py{4+2}
    \item \py{25-3}
    \item \py{-5+1}
  \end{enumerate}
\end{multicols}%

%Calculer les restes et les quotients des divisions euclidiennes suivantes : 
%\begin{multicols}{4}
%  \begin{enumerate}[label=\emph{\alph*)}]
%    \item $127 \div 8$
%    \item $54 \div 3$
%    \item $58 \div 5$
%    \item $58 \div (-5)$
%    \item $-58 \div 5$
%    \item $-58 \div (-5)$
%    \item $17583 \div 10$
%    \item $17583 \div 100$
%    \item $17583 \div 10^4$
%    \item $(2^7+2^4+2) \div 2^5$
%    \item $(2^7+2^4+2) \div 2^7$
%    \item $(2^7+2^4+2) \div 2^{10}$
%  \end{enumerate}
%\end{multicols}
%
%Calculer les nombres suivants avec une expression Python en repérant auparavant ceux qui donnent un résultat de type \py{int}.
%\begin{multicols}{5}
%  \begin{enumerate}[label=\emph{\alph*)}]
%    \item $3^5$
%    \item $2^{10}$
%    \item $(-3)^7$
%    \item $-3^7$
%    \item $5^{-2}$
%    \item $7^{\p{5^4}}$
%    \item $\p{7^5}^4$
%    \item $5^{7+6}$
%    \item $5^7+6$
%    \item $2^{\p{10^4}}$
%  \end{enumerate}
%\end{multicols}