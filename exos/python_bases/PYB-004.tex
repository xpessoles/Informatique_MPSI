\question{}
\begin{enumerate}
	\item Ecrire une fonction qui à un nombre entier associe le chiffre des unités.
	\item Ecrire une fonction qui à un nombre entier associe le chiffre des dizaines.
	\item Ecrire une fonction qui à un nombre entier associe le chiffre des unités en base 8.
\end{enumerate}

\question{Ouvrir votre IDE, écrire la fonction suivante dans un fichier, l'enregistrer, taper \emph{run} (F5) puis utiliser la fonction dans l'interpréteur interactif. 
Décrire ensuite précisément ce que réalise cette fonction.}

\begin{pyverbatim}
def split_modulo(n):
  """A vous de dire ce que fait cette fonction !"""
  return (n%2,n%3,n%5)
\end{pyverbatim}

\question{\'Ecrire une fonction \pyv{norme} qui prend en argument un vecteur de $\R^2$ donnée par ses coordonnées et renvoie sa norme euclidienne. 
Vous devrez spécifier clairement le type de l'argument à l'utilisateur via la \emph{docstring}.}

\question{\'Ecrire une fonction \pyv{lettre} qui prend en argument un entier \pyv{i} et renvoie la \pyv{i}\ieme\ lettre de l'alphabet.}

\question{\'Ecrire une fonction \pyv{carres} qui prend en argument un entier naturel \pyv{n} et qui renvoie la liste des \pyv{n} premiers carrés d'entiers, en commençant par $0$.}

