
Lancer PYZO ou IDLE. Un \emph{interpréteur de commandes}, ou \texttt{shell}, s'affiche. 
Le symbole \texttt{>}\texttt{>}\texttt{>} signifie que \python{} attend vos instructions. 

Sitôt une instruction tapée et validée (par la touche \og Entrée \fg{}), le \texttt{shell} effectue le calcul demandé puis affiche un résultat, ou un message d'erreur. 
Il est extrêmement important de bien lire ces messages d'erreur, et de les comprendre ! 

\medskip{}

\question{} Taper dans le \emph{shell} les instructions suivantes. 
\begin{verbatim}
x = 42
y = 42.
type(x)
type(y)
x = x+y
x
type(x)
\end{verbatim}
Que se passe-t-il ? Qu'est-ce que cela signifie ?

\medskip{}

\question{} Décrire ce que font les opérations suivantes, après les avoir étudiées sur des exemples numériques.
\begin{center}
  \texttt{+}\qquad \texttt{-}\qquad \texttt{*}\qquad \texttt{**}\qquad \texttt{/}\qquad \texttt{//}\qquad \texttt{\%}
\end{center}


\question{} Taper dans le \emph{shell} les instructions suivantes. 
\begin{verbatim}
B = 42 > 41.
type(B)
\end{verbatim}
Que se passe-t-il ? Qu'est-ce que cela signifie ?

\medskip{}

\question{} Décrire ce que font les opérations suivantes, après les avoir étudiées sur des exemples numériques.
\begin{center}
  \texttt{==} {} \qquad{} \texttt{!=}\qquad\texttt{<}\qquad\texttt{>}\qquad\texttt{<=}\qquad\texttt{>=}
\end{center}

\medskip{}

\question{} Taper dans le \emph{shell} les instructions suivantes. 
\begin{verbatim}
3/0 > 5
\end{verbatim}
Que se passe-t-il ? Qu'est-ce que cela signifie ?
\medskip{}

\question{} Taper dans le \emph{shell} les instructions suivantes. 
\begin{verbatim}
B = (42 > 41) or ( 3/0 > 5).
type(B)
\end{verbatim}
Que se passe-t-il ? Qu'est-ce que cela signifie ?

\medskip{}

\question{} Décrire ce que font les opérations suivantes, après les avoir étudiées sur des exemples logiques.
\begin{center}
  \texttt{or}\qquad\texttt{and}\qquad\texttt{not}
\end{center}


\medskip{}

\question{} Taper dans le \emph{shell} les instructions suivantes. 
\begin{verbatim}
x = -3
abs(x)
help(abs)
\end{verbatim}
Que se passe-t-il ? Qu'est-ce que cela signifie ?

\medskip{}

\question{} Taper dans le \emph{shell} les instructions suivantes. 
\begin{verbatim}
import math as m
import numpy as np
m.sin(m.pi)
np.sin(np.pi)
np.sin([0,np.pi])
m.sin([0,m.pi])
\end{verbatim}
Que se passe-t-il ? Qu'est-ce que cela signifie ?


\medskip{}

\question{} Taper dans le \emph{shell} les instructions suivantes. 
\begin{verbatim}
L = [0,1,2,3,4,5,6]
type(L)
L[0]
L[6]
L[-1]
L[-2]
L[7]
L[1:4]
L[2:8]
L.append(7)
L
L = L.append(8)
L
\end{verbatim}
Que se passe-t-il ? Qu'est-ce que cela signifie ?

\medskip{}

Nous avons vu comment utiliser des fonctions et des bibliothèques. Nous pouvons bien entendu créer nos propres fonctions (et bibliothèques). 

Dans PYZO ou IDLE, ouvrir un nouveau fichier (CTRL+N ou File / New file). L'enregistrer (CTRL + S ou File / Save) sous le nom \texttt{TP02.py}. 

\medskip{}

\question{} Taper dans cette fenêtre le script suivant. 
\begin{lstlisting}
"""TP n02"""
def somme(n) : 
    """Renvoie 0 + 1 + 2 + ... + n
    Precondition : n entier naturel"""
    return n*(n+1) // 2 
\end{lstlisting}
Enregistrer puis appuyer sur la touche Crtl+MAJ+E (sous PYZO) ou F5 (sous IDLE).
Le \emph{shell} doit s'afficher. 
Taper dans le \emph{shell} les instructions suivantes. 
\begin{lstlisting}
somme(42)
somme(42.)
somme(-1515)
help(somme)
\end{lstlisting}
Que se passe-t-il ? Qu'est-ce que cela signifie ?

\medskip{}

\question{} Comment peut-on utiliser la fonction écrite précédemment dans un \emph{autre} script \python{} ? 