\subsection{Consignes}
\begin{enumerate}
\item  \textbf{Lisez attentivement  tout l'énoncé
    avant de commencer.}
% \item Commencez la séance en créant un dossier au nom du TP dans le répertoire dédié à l'informatique de votre compte. 
\item Après la séance, vous devez rédiger un compte-rendu de TP et
l'envoyer au format électronique à votre enseignant.
\item Le seul format accepté pour l'envoi d'un texte de compte-rendu est le
format PDF. Votre fichier s'appellera impérativement \texttt{tp02\_kleim\_durif.pdf}, où \og \texttt{kleim}\fg\ et \og \texttt{durif}\fg\ sont à remplacer par les noms des membres du binôme. 
\item Ce TP est à faire en binôme, vous ne rendrez donc qu'un compte-rendu pour deux.
\item Vous préciserez en préambule de votre compte-rendu les noms des membres du binôme ainsi que le système d'exploitation sur lequel vous avez travaillé. 
\item Ayez toujours un crayon et un papier sous la main. Quand vous réfléchissez à une question, utilisez-les !
\item Vous devez être autonome. Ainsi, avant de poser une question à l'enseignant, merci de commencer par :
\begin{itemize}
  \item relire l'énoncé du TP (beaucoup de réponses se trouvent dedans) ;
  \item relire les passages du cours\footnote{Dans le cas fort 
improbable où vous ne vous en souviendriez pas.} relatifs à votre problème ;
  \item effectuer une recherche dans l'aide disponible sur votre ordinateur (ou sur internet) concernant votre question.
\end{itemize}
  Il est alors raisonnable d'appeler votre enseignant pour lui demander des explications ou une confirmation !
\end{enumerate}

%Le but de ce TP est de vous faire de manipuler un système d'exploitation par lignes de commandes, puis de prendre en main \python{}.
%
%{\bf Attention :} les étudiants travaillant sous Unix (Linux et MacOS) répondront d'abord aux questions des parties \ref{tp02:sec:unix1} et \ref{tp02:sec:unix2}, ceux travaillant sous Windows répondront d'abord aux questions des parties \ref{tp02:sec:win1} et \ref{tp02:sec:win2}. Tous les étudiants finiront par répondre aux questions de la partie~\ref{tp02:sec:python}. 
%
%Pour les manipulations de fichier, vous utiliserez le code source du logiciel
%\python{}, que vous pourrez trouver dans le dossiers groupes. à l'adresse suivante (où \texttt{X} 
%est à remplacer par \texttt{1} ou \texttt{2})  :
%\begin{center}
%  $\sim$\texttt{/groupes/mpsX/données/TP02/cpython-4243df51fe43}
%\end{center}


\subsection{Prise en main élémentaire de \python{}.} \label{tp02:sec:python}

Lancer IDLE (via un terminal, n'oubliez pas l'autocomplétion, ou le menu). Un \emph{interpréteur de commandes}, ou \texttt{shell}, s'affiche. 
Le symbole \texttt{>}\texttt{>}\texttt{>} signifie que \python{} attend vos instructions. 

Sitôt une instruction tapée et validée (par la touche \og Entrée \fg{}), le \texttt{shell} effectue le calcul demandé puis affiche un résultat, ou un message d'erreur. 
Il est extrêmement important de bien lire ces messages d'erreur, et de les comprendre ! 

\medskip{}

\question{} Taper dans le \emph{shell} les instructions suivantes. 
\begin{verbatim}
x = 42
y = 42.
type(x)
type(y)
x = x+y
x
type(x)
\end{verbatim}
Que se passe-t-il ? Qu'est-ce que cela signifie ?

\medskip{}

\question{} Décrire ce que font les opérations suivantes, après les avoir étudiées sur des exemples numériques.
\begin{center}
  \texttt{+}\qquad \texttt{-}\qquad \texttt{*}\qquad \texttt{**}\qquad \texttt{/}\qquad \texttt{//}\qquad \texttt{\%}
\end{center}


\question{} Taper dans le \emph{shell} les instructions suivantes. 
\begin{verbatim}
B = 42 > 41.
type(B)
\end{verbatim}
Que se passe-t-il ? Qu'est-ce que cela signifie ?

\medskip{}

\question{} Décrire ce que font les opérations suivantes, après les avoir étudiées sur des exemples numériques.
\begin{center}
  \texttt{==} {} \qquad{} \texttt{!=}\qquad\texttt{<}\qquad\texttt{>}\qquad\texttt{<=}\qquad\texttt{>=}
\end{center}

\medskip{}

\question{} Taper dans le \emph{shell} les instructions suivantes. 
\begin{verbatim}
3/0 > 5
\end{verbatim}
Que se passe-t-il ? Qu'est-ce que cela signifie ?
\medskip{}

\question{} Taper dans le \emph{shell} les instructions suivantes. 
\begin{verbatim}
B = (42 > 41) or ( 3/0 > 5).
type(B)
\end{verbatim}
Que se passe-t-il ? Qu'est-ce que cela signifie ?

\medskip{}

\question{} Décrire ce que font les opérations suivantes, après les avoir étudiées sur des exemples logiques.
\begin{center}
  \texttt{or}\qquad\texttt{and}\qquad\texttt{not}
\end{center}


\medskip{}

\question{} Taper dans le \emph{shell} les instructions suivantes. 
\begin{verbatim}
x = -3
abs(x)
help(abs)
\end{verbatim}
Que se passe-t-il ? Qu'est-ce que cela signifie ?

\medskip{}

\question{} Taper dans le \emph{shell} les instructions suivantes. 
\begin{verbatim}
import math as m
import numpy as np
m.sin(m.pi)
np.sin(np.pi)
np.sin([0,np.pi])
m.sin([0,m.pi])
\end{verbatim}
Que se passe-t-il ? Qu'est-ce que cela signifie ?


\medskip{}

\question{} Taper dans le \emph{shell} les instructions suivantes. 
\begin{verbatim}
L = [0,1,2,3,4,5,6]
type(L)
L[0]
L[6]
L[-1]
L[-2]
L[7]
L[1:4]
L[2:8]
L.append(7)
L
L = L.append(8)
L
\end{verbatim}
Que se passe-t-il ? Qu'est-ce que cela signifie ?

\medskip{}

Nous avons vu comment utiliser des fonctions et des bibliothèques. Nous pouvons bien entendu créer nos propres fonctions (et bibliothèques). 

Dans IDLE, ouvrir un nouveau fichier (CTRL+N ou File / New file). L'enregistrer (CTRL + S ou File / Save) sous le nom \texttt{TP02.py}. 

\medskip{}

\question{} Taper dans cette fenêtre le script suivant. 
\begin{lstlisting}
"""TP n02"""
def somme(n) : 
    """Renvoie 0 + 1 + 2 + ... + n
    Precondition : n entier naturel"""
    return n*(n+1) // 2 
\end{lstlisting}
Enregistrer puis appuyer sur la touche F5 ou (Run / Run Module).
Le \emph{shell} doit s'afficher. 
Taper dans le \emph{shell} les instructions suivantes. 
\begin{lstlisting}
somme(42)
somme(42.)
somme(-1515)
help(somme)
\end{lstlisting}
Que se passe-t-il ? Qu'est-ce que cela signifie ?

\medskip{}

\question{} Comment peut-on utiliser la fonction écrite précédemment dans un \emph{autre} script \python{} ? 