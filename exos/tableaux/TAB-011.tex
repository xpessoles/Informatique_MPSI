On définit la fonction de Kaprekar comme suit (on la note $K$). 
Si $n\in\mathbb{N}$, on écrit $n$ en base $10$ (sans zéros inutiles), on prend $c$ le nombre obtenu en écrivant les chiffres de $n$ dans l'ordre croissant et $d$ celui obtenu en écrivant les chiffres de $n$ dans l'ordre décroissant. 
On pose alors $K(n) = d-c$.
\begin{exemple}
  $K(6384) = 8643 - 3468 = 5175$, $K(5175) = 5994$, $K(5355) = 5994$, $K(5994) = 5355$, $K(5355) = 1998$, $K(1998)=8082$, $K(8082) = 8532$, $K(8532) = 6174$ et $K(6174) = 6174$.
\end{exemple}
Toutes les suites récurrentes ainsi construites bouclent. Ici, nous avons un cas particulier : la suite est stationnaire.

\question Construire une fonction \pyv{Kaprekar(n)} prenant en argument un entier \pyv{n} et donnant en sortie les valeurs de la suite décrite plus haut, jusqu'à ce qu'elle ne boucle. 

\begin{exemple}
  L'appel de \pyv{Kaprekar(6384)} devra donner comme résultat\\ \pyv{[6384, 5175, 5994, 5355, 1998, 8082, 8532, 6174]}.
\end{exemple}

\emph{Indice : on pourra écrire une fonction convertissant un entier en la liste de ses chiffres en base 10, une fonction convertissant une liste de chiffres en entier, une fonction calculant $K$ et utiliser la méthode \pyv{sort}}.
