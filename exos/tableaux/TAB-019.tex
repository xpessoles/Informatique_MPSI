Soit $\texttt{t} = [t_0,\dots,t_{n-1}]$ un tableau de nombres de longueur $n\in\N^\ast$, soit $0 \leq i,j \leq n-1$. 

On dit qu'il y a un \emph{ascension} dans le tableau \texttt{t} aux indices $i$ et $j$ si 
\begin{equation*}
  i < j \quad\textrm{et}\quad t_i < t_j. 
\end{equation*}
La \emph{hauteur} de cette ascension est alors $t_j - t_i$.

\bigskip{}

On propose la fonction suivante. 

\begin{Verbatim}[gobble=0,numbers=left]
def plus_haute_ascension(t):
    """Plus haute ascension de t, ou 0 s'il n'y en a pas.
    Précondition : t est un tableau de nombres"""
    M = 0
    n = len(t)
    for i in range(n):
        m = 0
        for j in range(i+1,n):
            if t[j] - t[i] > m :
                m = t[j] - t[i]
        if m > M :
            M = m
    return M
\end{Verbatim}

\bigskip{}

\question{} Montrer que \og si $\texttt{m}>0$, alors \texttt{m} est la hauteur de la plus haute ascension de \texttt{t[i:j]}, sinon $m=0$ \fg{} est un invariant de boucle pour la boucle \texttt{for} portant des lignes 8 à 10 de la fonction \texttt{plus\_haute\_ascension(t)}.

\medskip{} 

\question{} Justifier qu'un appel de la fonction \texttt{plus\_haute\_ascension(t)} renvoie bien le résultat demandé, à l'aide notamment d'un invariant.

\medskip{}

\question{} Écrire une fonction \texttt{nb\_ascensions(t)} renvoyant le nombre d'ascensions présentes dans le tableau de nombres \texttt{t} passé en argument. 