
La percolation \footnote{du latin \emph{percolare} : couler à travers.} désigne le passage d'un fluide à
travers un solide poreux. Ce terme fait bien entendu référence au café
produit par le passage de l'eau à travers une poudre de café comprimée,
mais dans un sens plus large peut aussi bien s'appliquer à
l'infiltration des eaux de pluie jusqu'aux nappes phréatiques ou encore
à la propagation des feux de forêt par contact entre les feuillages des
arbres voisins.

L'étude scientifique des modèles de percolation s'est développée à
partir du milieu du XXe siècle et touche aujourd'hui de nombreuses
disciplines, allant des mathématiques à l'économie en passant par la
physique et la géologie.

\setcounter{section}{1}
\subsection{Choix d'un modèle}\label{choix-dun-moduxe8le}


Nous allons aborder certains phénomènes propres à la percolation par
l'intermédiaire d'un modèle très simple : une grille carrée \emph{n n},
chaque case pouvant être ouverte (avec une probabilité \emph{p}) ou
fermée (avec une probabilité 1 \emph{p}). La question à laquelle nous
allons essayer de répondre est la suivante : est-il possible de joindre
le haut et le bas de la grille par une succession de cases ouvertes
adjacentes ?
\begin{figure}[!htb]
\begin{center}
\includegraphics[width=0.8\textwidth]{illustration_perco.jpg}
\caption{deux exemples de grilles $10\times10$ ; la percolation
n'est possible que dans le second cas (les cases ouvertes sont les cases
blanches)}
\end{center}
\end{figure}




On conçoit aisément que la réussite ou non de la percolation dépend
beaucoup de $p$ : plus celle-ci est grande, plus les chances de
réussite sont importantes. Nous auront l'occasion d'observer l'existence
pour de grandes valeurs de \emph{n} d'un seuil critique
$p_0$ au delà duquel la percolation a toutes les
chances de réussir et en deçà duquel la percolation échoue presque à
chaque fois.

\subsection{Création et visualisation de la grille}\label{cruxe9ation-et-visualisation-de-la-grille}

Les deux modules essentiels dont nous aurons besoin sont les modules
\pyv{numpy} (manipulation de tableaux bi-dimensionnels) et \pyv{mathplotlib.pyplot}
(graphisme), qu'il convient d'importer :

\begin{lstlisting}
import numpy as np
import matplotlib.pyplot as plt
\end{lstlisting}

Nous aurons aussi besoin de la fonction rand du module \pyv{numpy.random}
(fonction qui retourne un nombre pseudo- aléatoire de l'intervalle
$\left[0,1\right[$ et accessoirement de la fonction \pyv{ListedColormap} du
module \pyv{matplotlib.colors} (pour choisir l'échelle chromatique à utiliser
pour la représentation graphique). Ces deux fonctions seront importées
directement, puisque nous n'aurons pas besoin des modules entiers :

\begin{lstlisting}
from numpy.random import rand
from matplotlib.colors import ListedColormap
\end{lstlisting}

La grille de percolation sera représentée par le type \pyv{np.array}. La
fonction \pyv{np.zeros((n, p))} renvoie un tableau de \emph{n} lignes et
\emph{p} colonnes contenant dans chacune de ses cases le nombre flottant
0.0. Une fois un tableau \pyv{tab} créé, la case d'indice (\emph{i, j}) est
référencée indifféremment par \pyv{tab[i][j]} ou par \pyv{tab[i][j]} et
peut être lue et modifiée (comme d'habitude, les indices débutent à 0).
Enfin, on notera que si tab est un tableau, l'attribut \pyv{tab.shape}
retourne le couple $(n, p)$ de ses dimensions verticale et
horizontale (le nombre de lignes et de colonnes, \pyv{tab} étant vu comme une
matrice).

Dans la suite de ce document, on représentera une grille de percolation
par un tableau $n\times n$, les cases fermées contenant le nombre
flottant $0.0$ et les cases ouvertes le nombre flottant $1.0$.

\question{Définir une fonction \python, \pyv{creationgrille(p, n)} à deux
paramètres : un nombre réel \emph{p} (qu'on supposera dans l'intervalle
$[0,1[$ et un entier naturel \emph{n}, qui renvoie un tableau
$(n,n)$ dans lequel chaque case sera ouverte avec la probabilité
$p$ et fermée sinon.}
%
%Pour visualiser simplement la grille, nous allons utiliser la fonction
%plt.matshow : appliquée à un tableau, celle-ci présente ce dernier sous
%forme de cases colorées en fonction de leur valeur \textsuperscript{2}.
%Les couleurs sont choisies en fonction d'une échelle chromatique que
%vous pouvez visualiser à l'aide de la fonction plt.colorbar(). Celle
%utilisée par défaut va du bleu au rouge ; puisque nos grilles ne
%contiennent pour l'instant que les valeurs 0 ou 1, les cases fermées
%apparaîtrons en bleu, et les cases ouvertes, en rouge.
%
%Changer l'échelle chromatique
%
%L'argument par défaut cmap de la fonction plt.matshow permet de modifier
%l'échelle chromatique utilisée. La fonction ListedColormap va nous
%permettre de créer l'échelle de notre choix. Puisque nous n'aurons que
%trois états possibles (une case pleine représentée par la valeur 0.0),
%une case vide représentée par la valeur 1.0 et plus tard une case
%remplie d'eau représentée par la valeur 0.5) une échelle à trois
%couleurs suffit. Vous pouvez utiliser celle-ci :
%
%ou celle de votre choix, dans la limite du bon goût.
%
%(Vous trouverez la liste des couleurs prédéfinies à l'adresse
%http://www.python−simple.com/img/img36.png)
%\end{quote}
%
%\section{Percolation}\label{percolation}
%
%\begin{quote}
%\includegraphics[width=0.82943in,height=0.47396in]{media/image1.png}\includegraphics{media/image2.png}Une
%fois la grille crée, les cases ouvertes de la première ligne sont
%remplies par un fluide, ce qui sera représenté par la valeur 0.5 dans
%les cases correspondantes. Le fluide pourra ensuite être diffusé à
%chacune des cases ouvertes voisines d'une case contenant déjà le fluide
%jusqu'à remplir toutes les cases ouvertes possibles.
%
%Figure 2 -- \emph{les deux grilles de la figure 1, une fois le processus
%de percolation terminé (le fluide est représenté par des hachures).}
%
%Question 2. Écrire une fonction percolation qui prend en argument une
%grille et qui remplit de fluide celle-ci, en appliquant l'algorithme
%exposé ci-dessous.
%\end{quote}
%
%\begin{enumerate}
%\def\labelenumi{(\roman{enumi})}
%\item
%  Créer une liste contenant initialement les coordonnées des cases
%  ouvertes de la première ligne de la grille et remplir ces cases de
%  liquide.
%\item
%  Puis, tant que cette liste n'est pas vide, e ffectuer les opérations
%  suivantes :
%\end{enumerate}
%
%\begin{quote}
%--extraire de cette liste les coordonnées d'une case quelconque ;
%
%--ajouter à la liste les coordonnées des cases voisines qui sont encore
%vides, et les remplir de liquide.
%
%L'algorithme se termine quand la liste est vide.
%
%Rédiger un script vous permettant de visualiser une grille avant et
%après remplissage, et faire l'expérience avec quelques valeurs de
%\emph{p} pour une grille de taille raisonnable (commencer avec \emph{n}
%= 64 pour vérifier visuellement que votre algorithme est correct, puis
%augmenter la taille de la grille, par exemple avec \emph{n} = 512).
%
%On dit que la percolation est réussie lorsqu'à la fin du processus au
%moins une des cases de la dernière ligne est remplie du fluide.
%
%Question 3. Écrire une fonction teste\_percolation qui prend en argument
%un réel \emph{p} {[}0\emph{,} 1{[} et un entier \emph{n}
%N\textsuperscript{∗}, crée une grille, effectue la percolation et
%retourne :
%\end{quote}
%
%\begin{itemize}
%\item
%  True lorsque la percolation est réussie, c'est-à-dire lorsque le bas
%  de la grille est atteint par le fluide ;
%\item
%  False dans le cas contraire.
%\end{itemize}
%
%\section{Seuil critique}\label{seuil-critique}
%
%\begin{quote}
%Nous allons désormais travailler avec des grilles de taille 128 128
%\textsuperscript{3}. Faire quelques essais de percolation avec
%différentes valeurs de \emph{p}. Vous observerez assez vite qu'il semble
%exister un seuil \emph{p}\textsubscript{0} en deçà duquel la percolation
%échoue presque à chaque fois, et au delà duquel celle-ci réussit presque
%à chaque fois. Plus précisément, il est possible de montrer que pour une
%grille de taille infinie, il existe un seuil critique
%\emph{p}\textsubscript{0} en deçà duquel la percolation échoue toujours,
%et au delà duquel la percolation réussit toujours. Bien évidemment, plus
%la grille est grande, plus le comportement de la percolation tend à se
%rapprocher du cas de la grille théorique infinie.
%
%Question 4. Notons P(\emph{p}) la probabilité pour le fluide de
%traverser la grille.
%\end{quote}
%
%\begin{enumerate}
%\def\labelenumi{\alph{enumi})}
%\item
%  Proposer une démarche expérimentale simple pour estimer cette quantité
%  en fonction de \emph{p}, et rédiger la fonction
%\end{enumerate}
%
%\begin{quote}
%proba correspondante.
%
%P(\emph{p})
%\end{quote}
%
%1
%
%\begin{quote}
%\emph{p}
%
%\emph{p}0 1
%
%Figure 3 -- \emph{L'allure théorique du graphe de la fonction} P\emph{.}
%\end{quote}
%
%\begin{enumerate}
%\def\labelenumi{\alph{enumi})}
%\item
%  En utilisant une recherche dichotomique, chercher à estimer le plus
%  précisément possible la valeur numérique du seuil
%  \emph{p}\textsubscript{0}.
%\end{enumerate}
%
%\section{Propriétés
%macroscopiques}\label{propriuxe9tuxe9s-macroscopiques}
%
%\begin{quote}
%A l'instar de la thermodynamique, on peut décrire le comportement d'un
%système lors d'une transition de phase en introduisant des propriétés
%macroscopiques. Dans le cas de la percolation, on peut par exemple
%définir la densité moyenne \emph{d}(\emph{p}) de cases ouvertes
%atteintes par le fluide.
%
%Question 5.
%\end{quote}
%
%\begin{enumerate}
%\def\labelenumi{\alph{enumi})}
%\item
%  Définir une fonction densite qui prend en argument une grille dans
%  laquelle la percolation a eu lieu et qui retourne la valeur de sa
%  densité.
%\item
%  À l'aide d'un nombre raisonnable d'échantillons, définir alors la
%  fonction d qui à une probabilité \emph{p} ∈ {[}0\emph{,} 1{[} associe
%  la densité moyenne de la percolation dans une grille 128 × 128.
%\item
%  Tracer le graphe de \emph{d}(\emph{p}) pour \emph{p} ∈ {[}0\emph{,}
%  1{[}.
%\end{enumerate}
%
%\section{Et pour les plus rapides}\label{et-pour-les-plus-rapides}
%
%\begin{quote}
%Recommencez toute cette étude, mais cette fois-ci avec une grille
%hexagonale. Il est possible de prouver que pour une grille hexagonale
%carrée le seuil critique \emph{p}\textsubscript{0} est exactement égal à
%\emph{p}\textsubscript{0} = 1\emph{/}2 ; le vérifier expérimentalement
%et chercher à le démontrer (en utilisant un argument de symétrie).
%\end{quote}

\end{document}
