Étant donnés deux vecteurs \texttt{u} et \texttt{v} de même taille, de
coordonnées $( u_0,u_1,\cdots,u_{n-1})$ et
$(v_0,v_1,\cdots,v_{n-1})$, on définit la somme
\texttt{u+v} de coordonnées $(u_0+v_0,u_1+v_1,\cdots,u_{n-1}+v_{n-1})$, et le
produit scalaire \texttt{u.v}, qui est le réel
$\displaystyle\sum_{k=0}^{n-1}u_kv_k$.\\
On choisit de représenter tout vecteur $(x_0,x_1,\cdots,x_{n-1})$ par le tableau
\texttt{$[x_0,x_1,\cdots,x_{n-1}]$}. Écrire alors deux fonctions calculant la
somme et le produit scalaire de deux vecteurs.