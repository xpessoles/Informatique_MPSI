 \question{\'Ecrire une fonction \pyv{indice(x, t)} renvoyant un indice
\pyv{i} tel que \pyv{t[i]==x} si \pyv{x} apparaît dans le tableau \pyv{t} et $-1$ sinon.}
%\end{exo}

%\begin{exo}
 \question{\'Ecrire une fonction \pyv{tous_les_indices(e,t)} renvoyant la liste de tous les indices des occurences de \pyv{e} dans le tableau \pyv{t}.}
%\end{exo}

%\begin{exo}
 \question{Écrire une fonction \pyv{compte(e,t)} renvoyant le nombre d'occurences de \pyv{e} dans le tableau \pyv{t}.}
%\end{exo}

%\begin{exo}
 \question{\'Ecrire une fonction \pyv{ind_appartient_dicho(e,t)} renvoyant l'indice d'une occurence de \pyv{e} dans le tableau \pyv{t} (\pyv{None} si \pyv{e} n'est pas dans \pyv{t}), en supposant que \pyv{t} est trié par ordre croissant.}
%\end{exo}


%\begin{exo}
 \question{\'Ecrire une fonction \pyv{dec_appartient_dicho(e,t)} renvoyant un booléen indiquant si \pyv{e} est dans le tableau \pyv{t}, en supposant que \pyv{t} est trié par ordre décroissant.}
%\end{exo}

%\begin{exo}
 \question{Écrire une fonction \pyv{compte(e,t)} comptant le nombre d'occurences de l'élément \pyv{e} dans le tableau \texttt{t}.}