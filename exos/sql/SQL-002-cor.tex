\question{} Le couple (\texttt{id\_etudiant}, \texttt{id\_interro}) est une clef primaire pour la table \texttt{interros}. 
  Ainsi, tout système de gestion de bases de données devrait refuser deux enregistrements ayant les mêmes valeurs pour ce couple d'attributs. 
  
  \fbox{On ne peut donc pas donner plusieurs notes pour un même étudiant et une même interrogation.}
  
\medskip{}  
  
\question{}
\begin{verbatim}
SELECT sujet
FROM interros
WHERE id = 1 ;
\end{verbatim}

  
\medskip{}  
  
\question{}
\begin{verbatim}
SELECT nom, prenom
FROM etudiants ;
\end{verbatim}
  
\medskip{}  
  
\question{}
\begin{verbatim}
SELECT DISTINCT prenom
FROM etudiants ;
\end{verbatim}
  
\medskip{}  
  
\question{}
\begin{verbatim}
SELECT id
FROM etudiants
WHERE nom = "reinedesneiges" ;
\end{verbatim}
  
\medskip{}  
  
\question{}
\begin{verbatim}
SELECT max(date_naissance)
FROM etudiants ;
\end{verbatim}
  
\medskip{}  
  
\question{}
\begin{verbatim}
SELECT DISTINCT nom, prenom
FROM etudiants, notes 
WHERE id = id_etudiant
      AND
      note = 20 ;
\end{verbatim}
Ou bien, avec une jointure : 
\begin{verbatim}
SELECT DISTINCT nom, prenom
FROM etudiants JOIN notes ON id = id_etudiant
WHERE note = 20 ;
\end{verbatim}

\medskip{}  
  
\question{}

\begin{verbatim}
SELECT nom, prenom, titre
FROM etudiants, interros, notes
WHERE etudiants.id = id_etudiant
      AND
      interros.id = id_interro
      AND 
      note = 0;
\end{verbatim}

\medskip{}  
  
\question{}
\begin{verbatim}
SELECT nom, prenom, AVG(note) AS moyenne
FROM etudiants, notes
WHERE id = id_etudiant
GROUP BY id;
\end{verbatim}
Ou bien, avec une jointure : 
\begin{verbatim}
SELECT nom, prenom, AVG(note) AS moyenne
FROM etudiants JOIN notes ON id = id_etudiant
GROUP BY id;
\end{verbatim}

\medskip{}

\question{}
\begin{verbatim}
SELECT nom, prenom, COUNT(*) AS nb_copies
FROM etudiants notes
WHERE etudiants.id = id_etudiant
GROUP BY etudiants.id
\end{verbatim}
Ou bien, avec une jointure : 
\begin{verbatim}
SELECT nom, prenom, COUNT(*) AS nb_copies
FROM etudiants JOIN notes ON id = id_etudiant
GROUP BY etudiants.id
\end{verbatim}

\medskip{}  
  
\question{}
\begin{verbatim}
SELECT nom, prenom, COUNT(*) AS nb_max_notes
FROM etudiants, notes
WHERE id = id_etudiant
GROUP BY id
ORDER BY nb_max_notes DESC
LIMIT 1;
\end{verbatim}
Ou bien, avec une jointure : 
\begin{verbatim}
SELECT nom, prenom, COUNT(*) AS nb_max_notes
FROM etudiants JOIN notes ON id = id_etudiant
GROUP BY id
ORDER BY nb_max_notes DESC
LIMIT 1;
\end{verbatim}

\medskip{}  
  
\question{}
\begin{verbatim}
SELECT nom, prenom, COUNT(*) AS nb_copies
FROM etudiants, notes
WHERE id = id_etudiant
GROUP BY id
HAVING nb_copies >= 10;
\end{verbatim}
Ou bien, avec une jointure : 
\begin{verbatim}
SELECT DISTINCT titre, COUNT(*) AS nb_copies
FROM etudiants JOIN notes ON id = id_etudiant
GROUP BY id
HAVING nb_copies >= 10;
\end{verbatim}

\medskip{}

\question{}
\begin{verbatim}
SELECT nom, prenom, COUNT(*) AS nb_copies_non_rendues
FROM interros, etudiants,
WHERE NOT EXISTS (SELECT *
                  FROM notes
                  WHERE etudiants.id = id_etudiant
                        AND
                        interros.id = id_interro
                 )
GROUP BY interros.id;
\end{verbatim}

\medskip{}

\question{}
\begin{verbatim}
SELECT titre
FROM interros

EXCEPT 

SELECT DISTINCT titre 
FROM interros, etudiants
WHERE NOT EXISTS (SELECT *
                  FROM notes
                  WHERE etudiants.id = id_etudiant
                        AND
                        interros.id = id_interro
                 )
;
\end{verbatim}


