Les classes de MPSI du lycée LMM organisent un tournoi d'intelligences artificielles de morpion. 
Une partie de morpion se déroule sur un damier de trois cases par trois cases. Un premier joueur choisit une case. 
Puis, le deuxième joueur choisit une autre case. 
Puis, le premier joueur choisit une troisième case, et l'on continue ainsi jusqu'à ce que :
\begin{itemize}
  \item soit un joueur aligne trois cases, dans ce cas il gagne ;
  \item soit le damier est rempli, dans ce cas il y a match nul. 
\end{itemize}
\begin{figure}[!h]
  \begin{center}
    \begin{tabular}{|c|c|c|}
      \hline
      1 & 2 & 3 \\
      \hline
      4 & 5 & 6 \\
      \hline
      7 & 8 & 9 \\
      \hline
    \end{tabular}
    \caption{Repérage des cases dans un damier de morpion.}
    \label{morpion:cases}
  \end{center}
\end{figure}
Les professeurs de ces classes ont donc créé une base de données afin de gérer ce tournoi. 
Cette base contient deux tables : 
\begin{itemize}
  \item une table contenant les informations des joueurs : identifiant du joueur, pseudonyme, date de naissance, classe ; 
  \item une table contenant les informations des parties : identifiant de la partie, identifiant du joueur \no 1, identifiant du joueur \no 2, résultat de la partie, coups joués, date de la partie. 
\end{itemize}
Voici les instructions SQL ayant permis de créé ces deux tables. 
\begin{verbatim}
CREATE TABLE joueurs (
          idj INTEGER,
          pseudo VARCHAR(50),
          datenaiss DATE,
          classe VARCHAR(50),
          PRIMARY KEY (idj)
          );

CREATE TABLE parties (
          idp INTEGER,
          idj1 INTEGER, 
          idj2 INTEGER,
          res INTEGER,
          coups INTEGER,
          date DATETIME,
          PRIMARY KEY(idp),
          FOREIGN KEY(idj1) REFERENCES joueurs,
          FOREIGN KEY(idj2) REFERENCES joueurs
          );
\end{verbatim}
Quelques conventions et rappels.
\begin{itemize}
  \item Si le résultat d'une partie vaut 0, il y a match nul, s'il vaut 1 le joueur 1 a gagné et s'il vaut 2 le joueur 2 a gagné.
  \item Pour une partie, on lit dans l'entier \texttt{coups} les coups effectués par les joueurs. Ainsi, si \texttt{coups = 54892}, alors 
    \begin{itemize}
      \item le joueur 1 joue dans la case 5 ;
      \item le joueur 2 joue dans la case 4 ;
      \item le joueur 1 joue dans la case 8 ;
      \item le joueur 2 joue dans la case 9 ;
      \item le joueur 1 joue dans la case 2.
    \end{itemize}
    Ainsi, dans cette partie, le joueur 1 a gagné (alignement sur la deuxième colonne). 
  \item Les dates (type \texttt{DATE}) sont au format \texttt{AAAA-MM-JJ}.
  \item Les dates-heures (type \texttt{DATETIME}) sont au format \texttt{AAAA-MM-JJ HH:MM:SS}. 
  \item Les dates et dates-heures sont écrites comme des chaînes de caractères (exemple : \texttt{"1515-09-13"}) et peuvent être comparées entre elles. 
  \item Si une requête ne donne pas qu'une réponse (exemple : plusieurs lignes vérifient un critère), on inscrira toutes les réponses. 
  \item Pour effectuer une jointure d'une table \texttt{T} avec elle même, on pourra lui donner un alias avec l'instruction \texttt{AS}. Voici un exemple.
\begin{verbatim}
SELECT *
FROM T
JOIN T AS Tbis ON T.[...] = Tbis.[...] ;
\end{verbatim}
\end{itemize}
% Enfin, voici quelques rappels concernant le fonctionnement de \texttt{sqlite3}.
% \begin{itemize}
%   \item Ouvrez un terminal dans le dossier contenant la base de données. 
%   \item Avec ce terminal, lancez \texttt{sqlite3} sur cette base de données avec la commande suivante.
% \begin{verbatim}
% sqlite3 mabdd.sqlite
% \end{verbatim}
%   Vous pouvez aussi lancer \texttt{sqlite3} sans argument et ouvrir ensuite une base de donnée avec la fonction \texttt{.open}.
%   \item Nous vous conseillons \emph{fortement} d'écrire vos requêtes SQL dans un script (fichier texte, via un éditeur de texte), enregistré dans le même dossier que votre base de données. 
%     Vous pouvez ensuite exécuter ce script dans \texttt{sqlite3} par la commande suivante. 
% \begin{verbatim}
% .read monscript.sql
% \end{verbatim}
% \end{itemize}

\bigskip

\question\ Quel est le pseudonyme de l'étudiant d'identifiant \no 42 ?

\medskip

\question\ Quel est l'identifiant de l'étudiant dont le pseudonyme est \texttt{"Galois"} ?

\medskip

\question\ Combien de parties ont été jouées ? 

\medskip

\question\ Combien y a-t-il eu de matchs nuls ? 

\medskip

\question\ Combien de parties différentes (\emph{i.e.} de successions de coups différentes) ont été jouées ? 

\medskip

\question\ Quel est l'identifiant de l'étudiant le plus jeune ? Les donner tous s'il y en a plusieurs.  

\medskip

\question\ Combien de parties ont été jouées par l'étudiant de pseudonyme \texttt{"Tux"} ? 

\medskip

\question\ Combien de parties ont été perdues par l'étudiant de pseudonyme \texttt{"Tux"} ? 

\medskip

\question\ Combien de parties ont été gagnées par des étudiants de MPS1 contre des étudiants de MPS2 ?

\medskip

\question\ Quels sont les pseudonymes des joueurs ayant joué le plus de parties en tant que joueur \no 1 ? Les donner tous s'il y en a plusieurs.  

\medskip

\question\ Combien de parties ont été jouées en sept coups ou moins, en 2018 et entre deux étudiants d'une même classe ?