
\documentclass[10pt]{article}

%----------------------------------------------------------------------------------------
%	Packages francais
%----------------------------------------------------------------------------------------


\usepackage[utf8]{inputenc} 
\usepackage[french]{babel}
\usepackage[T1]{fontenc}
%\usepackage[sf]{charter} % Style de police
\usepackage{lmodern}


%----------------------------------------------------------------------------------------
%	Titre du document
%----------------------------------------------------------------------------------------
\newcommand{\DocMatiere}{Informatique} % Matiere du document
\newcommand{\DocTitle}{A la recherche du meurtrier} % Titre du document
\newcommand{\DocType}{TD} % Type d'activité
\newcommand{\DocNumber}{} % Numero du projet
\newcommand{\DocCorrige}{Non} % Affiche le corrigé
\newcommand{\DocDate}{} % Date
\newcommand{\DocAuthorName}{Informatique PT - PT*} % Nom de l'auteur
%----------------------------------------------------------------------------------------
% Reglages des marges
%----------------------------------------------------------------------------------------
\usepackage{geometry}
\geometry{a4paper}
\geometry{top=2.5cm, bottom=2.5cm, left=2cm , right=2cm}
%----------------------------------------------------------------------------------------
%	Packages classiques
%----------------------------------------------------------------------------------------
\usepackage{fancyhdr} % Required for custom headers
\usepackage{lastpage} % Required to determine the last page for the footer
\usepackage{graphicx} % Required to insert images
\usepackage{subfigure}
\RequirePackage{algorithm}
\RequirePackage{algorithmic}
\usepackage{epstopdf}

%----------------------------------------------------------------------------------------
%	Packages Maths
%----------------------------------------------------------------------------------------
\usepackage{amsmath}
\usepackage{amsfonts}
\usepackage{amssymb}
\usepackage{stmaryrd}
%----------------------------------------------------------------------------------------
%	Packages divers
%----------------------------------------------------------------------------------------
\usepackage{enumerate} % choix style enumeration
\usepackage[table]{xcolor}
\usepackage{enumitem}
\usepackage{amssymb} 	
\usepackage{xargs}
\usepackage{xinttools}
\usepackage{tikz}
\usepgflibrary{arrows}
\usepackage{empheq}
\usepackage{multicol}
\usepackage[locale=FR]{siunitx}
\usepackage{url}
%----------------------------------------------------------------------------------------
%	Mise en forme code python
%----------------------------------------------------------------------------------------
% Default fixed font does not support bold face
\DeclareFixedFont{\ttb}{T1}{txtt}{bx}{n}{8} % for bold
\DeclareFixedFont{\ttm}{T1}{txtt}{m}{n}{8} % for normal
% Custom colors
\usepackage{color}
\definecolor{deepblue}{rgb}{0,0,0.5}
\definecolor{deepred}{rgb}{0.6,0,0}
\definecolor{deepgreen}{rgb}{0,0.5,0}
\definecolor{mygray}{rgb}{0.5,0.5,0.5}
\definecolor{mysoftgray}{rgb}{0.8,0.8,0.8}
\definecolor{grisclair}{rgb}{0.95,0.95,0.95}
%\definecolor{ocre}{rgb}{49,133,156} % Couleur ''bleue''
%\definecolor{violetf}{rgb}{112,48,160} % Couleur ''violet''

\usepackage{listings}

% Python style for highlighting
\newcommand\pythonstyle{\lstset{
language=Python,
%backgroundcolor=\color{grisclair},
%frame=single,
%frame=tb,              % Any extra options here
%numbers=left, % where to put the line-numbers; possible values are (none, left, right)
%numbersep=5pt, % how far the line-numbers are from the code
%numberstyle=\color{blue},
%stepnumber=1,
basicstyle=\tt,
otherkeywords={self},             % Add keywords here
keywordstyle=\ttb\color{deepblue},
emph={MyClass,__init__},          % Custom highlighting
emphstyle=\ttb\color{deepred},    % Custom highlighting style
stringstyle=\color{deepgreen},
showstringspaces=true,            % 
showtabs=true,
tabsize=5,
showspaces=false,  % indique les espace dans le code
commentstyle=\color{olive},
breaklines=true
}}


% Python environment
\lstnewenvironment{python}[1][]
{
\pythonstyle
\lstset{#1}
}
{}


% Python for external files
\newcommand\pythonexternal[1]{{
\pythonstyle
\lstinputlisting{#1}}}

% Python for inline
\newcommand\pythoninline[1]{{\pythonstyle\lstinline!#1!}} 
%-------------------listing et python------------------------------



%-------------------------sql
\newcommand\sqlstyle{\lstset{
language=SQL,
keywordstyle=\ttb\color{deepblue},
showspaces=false,
commentstyle=\color{olive},
showstringspaces=false,  
breaklines=true}}

% Python environment
\lstnewenvironment{sql}[1][]
{
\sqlstyle
\lstset{#1}
}
{}




% commandes d'espaces % Les espaces verticaux et horizontaux.
\newcommand{\sk}{\smallskip}
\newcommand{\mk}{\medskip}
\newcommand{\bk}{\bigskip}
\newcommand{\vk}{\vskip 1.5 cm}
\newcommand{\hk}{\hskip 1 cm\relax}


%----------------------------------------------------------------------------------------
%	Mise en page du document
%----------------------------------------------------------------------------------------
\linespread{1.1} % Line spacing
% Set up the header and footer
\pagestyle{fancy}
\lhead{\DocMatiere} % Top left header
\chead{\DocTitle} % Top center header
\rhead{\DocType \ \DocNumber} % Top right header
\lfoot{\DocAuthorName}% Bottom left footer
\cfoot{Page\ \thepage\ /\ \pageref{LastPage}} % Bottom center footer
\rfoot{\DocDate} % Bottom right footer
\renewcommand\headrulewidth{0.4pt} % Size of the header rule
\renewcommand\footrulewidth{0.4pt} % Size of the footer rule
\setlength\parindent{0pt} % Removes all indentation from paragraphs


%question exo
\newcounter{cexo}
\newenvironment{qexo}{
\refstepcounter{cexo}
\vspace{3 pt}
\noindent
\begin{minipage}[t]{0.15\textwidth}
\textbf{\noindent Question \arabic{cexo}. }
\end{minipage}\noindent
\begin{minipage}[t]{0.85\textwidth}}{\vspace{3 pt}
\end{minipage}}%\vspace{2 pt}
	
	
%	%objectifs , attention une virgule crée un tiret
%\newlength{\LongObj}
%\settowidth{\LongObj}{\textbf{ Objectif(s) }}
%\newlength{\largeurObj}
%\setlength{\largeurObj}{0.9\textwidth - \LongObj}
%\newcommand{\Obj}[1]{%\vspace{45 pt}%~\\ 
%\noindent
%\rule[4.5 pt]{0.1\textwidth}{0.4pt}\textbf{ Objectif(s) }\rule[4.5 pt]{\largeurObj}{0.4pt}
%\begin{itemize}[noitemsep,topsep=3pt]
%\foreach \x in {#1} {\item \textit{\x}}
%\end{itemize}
%\noindent\rule[4.5 pt]{1\textwidth}{0.4pt} \vspace{-1\baselineskip }
%}
	
%titre
\newenvironment{rem}{\par\vspace{10pt} % Vertical white space above the remark and (\small pour smaller font size)
\begin{list}{}{
\leftmargin=35pt % Indentation on the left
\rightmargin=25pt}\item\ignorespaces % Indentation on the right
\makebox[-2.5pt]{\begin{tikzpicture}[overlay]
\node[draw=orange!60,line width=1pt,circle,fill=orange!25,font=\sffamily\bfseries,inner sep=2pt,outer sep=0pt] at (-15pt,0pt){\textcolor{orange}{R}};\end{tikzpicture}} % Orange R in a circle
\advance\baselineskip -1pt}{\end{list}\vskip5pt} % Tighter line spacing and white space after remark

%titre
\newenvironment{obj}{\par\vspace{10pt} % Vertical white space above the remark and (\small pour smaller font size)
\begin{list}{}{
\leftmargin=35pt % Indentation on the left
\rightmargin=25pt}\item\ignorespaces % Indentation on the right
\makebox[-2.5pt]{\begin{tikzpicture}[overlay]
\node[draw=deepblue!60,line width=1pt,circle,fill=deepblue!25,font=\sffamily\bfseries,inner sep=2pt,outer sep=0pt] at (-15pt,0pt){\textcolor{deepblue}{Obj}};\end{tikzpicture}} % Orange R in a circle
\advance\baselineskip -1pt}{\end{list}\vskip5pt} % Tighter line spacing and white space after remark

%----------------------------------------------------------------------------------------
%	Repertoire images
%----------------------------------------------------------------------------------------
\graphicspath{{figures/}}






\usepackage{tikz}
\usetikzlibrary{shapes}
\usetikzlibrary{chains}
\usetikzlibrary{calc,patterns,decorations.pathmorphing,decorations.markings,intersections,
positioning}

\usepackage{lscape}

%----------------------------------------------------------------------------------------
%	Début du document
%----------------------------------------------------------------------------------------
\begin{document}


%----------------------------------------------------------------------------------------
%	TITLE PAGE & TABLE OF CONTENTS
%----------------------------------------------------------------------------------------
\begin{center}
\vspace{1cm}\textbf{\Large \DocType \ \DocNumber: \DocTitle}
\end{center}



%----------------------------------------------------------------------------------------
%	Début du doc
%----------------------------------------------------------------------------------------

\section{La police a besoin de vous}

Un crime a été commis et la police a besoin de vous. Celle-ci vous a donné le rapport de la scène 
de crime mais vous avez réussi malgré tout à le perdre... Vous vous rappelez vaguement que le crime
était un meurtre -- \emph{murder} -- qui a eu lieu à un moment de la journée du \emph{15 Janvier 2018} et qui s'est
déroulé à \emph{SQL City}. Commencez par consulter les rapports de scène de crime correspondants dans
la base de donnée du département de police.

\section{Consignes}

Le principal objet du TD est de travailler le langage SQL tout en abordant un sujet sympathique. Cela veut dire qu'il n'y a pas de difficultés/pièges dans la partie enquête policière. 

Toutes vos requêtes seront effectuées sur la base de données \texttt{sql-murder-mystery.db}. \'Evidemment, le but est
de franchir chaque étape de la résolution de l'enquête en effectuant la requête adaptée et non en lisant toute la
base de données (vous saurez assez vite de quoi je parle, parfois c'est tentant). Une fois que vous aurez trouvé le
criminel, vous pourrez tester votre réponse en utilisant la commande :

\begin{sql}
INSERT INTO solution VALUES (1, 'Insert the name of the person you found here');
SELECT value FROM solution;
\end{sql}


Afin de vous aider dans l'analyse de la base de données, la figure~\ref{fig:Schema} ci-dessous représente l'architecture et les liens de la base de données.

\newpage

%\begin{landscape}



\begin{figure}[!htbp]
\centering
\begin{tikzpicture}
\def\distance{1cm}
{\small
{\tt

%dl
\node[draw,rectangle split,rectangle split parts=2,rectangle split part fill={gray!10,white},
align=center] (dl) at (0,0){drivers\_license
\nodepart{two}
\begin{tabular}{lll}
K & \textcolor{red}{id} & INTEGER \\
&age & INTEGER \\
&height & INTEGER \\
&eye\_color & TEXT \\
&hair\_color & TEXT \\
&gender & TEXT \\
&plate\_number & TEXT \\
&car\_make & TEXT \\
&car\_model & TEXT  \\
\end{tabular}
};


%person
\node[draw,rectangle split,rectangle split parts=2,rectangle split part fill={gray!10,white},
align=center, below=\distance of dl] (p) {person
\nodepart{two}
\begin{tabular}{lll}
K & \textcolor{blue}{id} & INTEGER \\
&name & TEXT \\
&\textcolor{red}{license\_id} & INTEGER \\
&adress\_number & INTEGER \\
&adress\_street\_name & TEXT \\
&\textcolor{olive}{ssn} & INTEGER \\
\end{tabular}
};

%get fit now member
\node[draw,rectangle split,rectangle split parts=2,rectangle split part fill={gray!10,white},
align=center, below=\distance of p] (gfm) {get\_fit\_now\_member
\nodepart{two}
\begin{tabular}{lll}
K &\textcolor{orange}{id} & TEXT \\
&\textcolor{blue}{person\_id} & INTEGER \\
&name & TEXT \\
&membership\_start\_date & INTEGER \\
&membership\_status & TEXT \\
\end{tabular}
};


%interview
\node[draw,rectangle split,rectangle split parts=2,rectangle split part fill={gray!10,white},
align=center, left=\distance of gfm] (i) {interview
\nodepart{two}
\begin{tabular}{ll}
\textcolor{blue}{person\_id} & INTEGER \\
transcript & TEXT \\
\end{tabular}
};


%facebook
\node[draw,rectangle split,rectangle split parts=2,rectangle split part fill={gray!10,white},
align=center, right=\distance of gfm] (f) {facebook\_event\_checkin
\nodepart{two}
\begin{tabular}{ll}
\textcolor{blue}{person\_id} & INTEGER \\
event\_id & INTEGER \\
event\_name & TEXT \\
date & INTEGER \\
\end{tabular}
};


%get fit now check in
\node[draw,rectangle split,rectangle split parts=2,rectangle split part fill={gray!10,white},
align=center, below=\distance of gfm] (gfc) {get\_fit\_now\_check\_in
\nodepart{two}
\begin{tabular}{ll}
\textcolor{orange}{membership\_id} & TEXT \\
check\_in\_date & INTEGER \\
check\_in\_time & INTEGER \\
check\_out\_time & INTEGER \\
\end{tabular}
};


%income
\node[draw,rectangle split,rectangle split parts=2,rectangle split part fill={gray!10,white},
align=center, right=\distance of p] (in) {income
\nodepart{two}
\begin{tabular}{lll}
K & \textcolor{olive}{ssn} & INTEGER \\
& annual\_income & INTEGER \\
\end{tabular}
};


%crime_scene_report
\node[draw,rectangle split,rectangle split parts=2,rectangle split part fill={gray!10,white},
align=center, left=\distance of dl] (c) {crime\_scene\_report
\nodepart{two}
\begin{tabular}{ll}
date & INTEGER \\
type & TEXT \\
description & TEXT \\
city & TEXT \\
\end{tabular}
};


%solution
\node[draw,rectangle split,rectangle split parts=2,rectangle split part fill={gray!10,white},
align=center, right=\distance of dl] (s) {solution
\nodepart{two}
\begin{tabular}{ll}
user & INTEGER \\
value & TEXT \\
\end{tabular}
};



\draw[->,>=latex] (p) -- (dl);
\draw[->,>=latex] (in) -- (p);
\draw[->,>=latex] (i) |- (p);
\draw[->,>=latex] (gfm) -- (p);
\draw[->,>=latex] (f) |- (p.south east);
\draw[->,>=latex] (gfc) -- (gfm);

}
}
\end{tikzpicture}
\caption{Schéma entité-relation}
\label{fig:Schema}
\end{figure}


\begin{qexo}
Déterminer qui est l'assassin !
\end{qexo}


Si vous êtes coincés, voici un cheminement :
\begin{itemize}
\item Rédiger une requête qui identifie le premier témoin.
\item Écrivez une requête qui identifie le deuxième témoin.
\item Écrivez une requête qui montre les transcriptions des entretiens pour nos deux sujets.
\item Trouvez le meurtrier !
\end{itemize}


\section{Requêtes bonus pour s'exercer}
Ces requêtes demandées ne sont pas utiles à la résolution du crime précédent. Mais elles permettent de vous entraîner. C'est parti !

\begin{qexo}
Combien de personnes se trouvent dans cette base de données ?
\end{qexo}

\begin{qexo}
Que savons-nous de ces personnes ? On se limite aux 10 premières personnes.
\end{qexo}

\begin{qexo}
Quelles sont les valeurs possibles pour le type dans la base \texttt{crime`\_scene\_report} ?
\end{qexo}

\begin{qexo}
On souhaite tout connaître de \texttt{Kinsey Erickson}.
\end{qexo}


\begin{qexo}
On souhaite connaître tous les vols (\texttt{theft}) recensés à \texttt{Chicago}.
\end{qexo}

\begin{qexo}
On souhaite connaître toutes les villes où ont eu lieu des accidents avec un \texttt{i}.
\end{qexo}

\begin{qexo}
On souhaite connaître toutes les villes où ont eu lieu des accidents qui commencent par les lettres  de \texttt{W} à \texttt{Z}.
\end{qexo}


\begin{qexo}
L'âge le plus grand d'un possesseur d'un permis ?
\end{qexo}


\begin{qexo}
Toutes les infos sur les 10 plus vieux possesseur de permis. Tri par ordre croissant sur l'âge.
\end{qexo}


\begin{qexo}
Identifier les noms des personnes, ainsi que leurs revenus qui gagnent plus de \num{450000}.
\end{qexo}


\begin{qexo}
Le nom, les revenus, le genre, la couleur des yeux et la couleur des cheveux des personnes qui gagnent plus de \num{450000}.
\end{qexo}


\section{Solution}

\setcounter{cexo}{0}
	
\begin{qexo}
\end{qexo}
Commencez par récupérer le rapport de scène de crime correspondant dans la base de données du service de police.

\begin{sql}
SELECT description FROM crime_scene_report
WHERE date = '20180115' AND type = 'murder' AND city = 'SQL City'
\end{sql}
qui renvoie :

%\begin{tabular}{p{10cm}}
%"description" \\ \hline
{\tt "Security footage shows that there were 2 witnesses. The first witness lives at the last house on ""Northwestern Dr"". The second witness, named Annabel, lives somewhere on ""Franklin Ave""."}
%\end{tabular}

On creuse  alors la piste du témoin 1 :
\begin{sql}
SELECT * FROM person
WHERE address_street_name = 'Northwestern Dr'
ORDER BY address_number DESC 
\end{sql}
qui renvoie :

\begin{table}[!htp]
\centering
{\tt 
\begin{tabular}{rrrrrr}
id & name& license\_id& address\_number& address\_street\_name& ssn \\ \hline
14887& Morty Schapiro& 118009& 4919& Northwestern Dr& 111564949 \\
17729& Lasonya Wildey& 439686& 3824& Northwestern Dr& 917817122 \\
53890& Sophie Tiberio& 957671& 3755& Northwestern Dr& 442830147 \\
... &...&...&...&...&... 
\end{tabular}
}
\end{table}

Si on ne veut que le premier on peut indiquer une limite (bonus) :

\begin{sql}
SELECT * FROM person
WHERE address_street_name = 'Northwestern Dr'
ORDER BY address_number DESC LIMIT 1
\end{sql}
qui renvoie :

\begin{table}[!htp]
\centering
{\tt
\begin{tabular}{rrrrrr}
id& name& license\_id& address\_number& address\_street\_name& ssn \\ \hline
14887& Morty Schapiro& 118009& 4919& Northwestern Dr& 111564949 \\
\end{tabular}
}
\end{table}

En ce qui concerne le deuxième témoin :
\begin{sql}
SELECT * FROM person
WHERE name like '%Annabel%' AND address_street_name = 'Franklin Ave' 
\end{sql}
qui renvoie :

\begin{table}[!htp]
\centering
{\tt
\begin{tabular}{rrrrrr}
id& name& license\_id& address\_number& address\_street\_name& ssn \\ \hline
16371& Annabel Miller& 490173& 103& Franklin Ave& 318771143
\end{tabular}
}
\end{table}


Le témoignage de ces deux témoins donne avec le mot-clé \texttt{in} (mais c'est un exemple), à l'aide de la requête :
\begin{sql}
SELECT * FROM interview where person_id in (14887, 16371)
\end{sql}
donne : {\tt \begin{itemize}
\item 14887 -- I heard a gunshot and then saw a man run out. He had a "Get Fit Now Gym" bag. The membership number on the bag started with "48Z". Only gold members have those bags. The man got into a car with a plate that included "H42W".
\item 16371 -- I saw the murder happen, and I recognized the killer from my gym when I was working out last week on January the 9th.
\end{itemize}
}



On s'attaque aux personnes du côté de la gym, avec le bon sac, le statut est présent le 09 janvier.

\begin{sql}
SELECT m.person_id, m.name FROM get_fit_now_member AS m 
JOIN get_fit_now_check_in AS c ON m.id=c.membership_id 
WHERE m.id like '48Z%' and m.membership_status = 'gold' 
AND c.check_in_date = '20180109'

\end{sql}

ce qui renvoie deux candidats :

\begin{table}[!htp]
\centering
{\tt
\begin{tabular}{rr}
person\_id& name\\ \hline
28819& Joe Germuska \\
67318& Jeremy Bowers \\
\end{tabular}
}
\end{table}

Il faut donc croiser avec les plaques minéralogiques.

\begin{sql}
SELECT p.id, p.name FROM person as p JOIN drivers_license as d
ON d.id = p.license_id
WHERE d.plate_number like '%H42W%'
\end{sql}
qui nous renvoie 3 possibilités :

\begin{table}[!htp]
\centering
{\tt
\begin{tabular}{rrr}
id& name \\ \hline
51739& Tushar Chandra \\
67318& Jeremy Bowers \\
78193& Maxine Whitely\\
\end{tabular}
}
\end{table}



Remarquons la possibilité de conclure l'histoire sur ce personnage :
\begin{sql}
INSERT INTO solution VALUES (1, "Jeremy Bowers");
SELECT value FROM solution;
\end{sql}
qui renvoie : 

{\tt Congrats, you found the murderer! But wait, there's more... If you think you're up for a challenge, try querying the interview transcript of the murderer to find the real villian behind this crime. If you feel especially confident in your SQL skills, try to complete this final step with no more than 2 queries. }



Le tueur est donc : \texttt{Jeremy Bowers}. Regardons alors ce qu'il a à dire pour sa défense :

\begin{sql}
SELECT transcript FROM interview WHERE person_id = 67318
\end{sql}
ce qui donne :

{\tt I was hired by a woman with a lot of money. I don't know her name but I know she's around 5'5" (65") or 5'7" (67"). She has red hair and she drives a Tesla Model S. I know that she attended the SQL Symphony Concert 3 times in December 2017.}


Du côté de la voiture, du genre et de la couleur de cheveux :
\begin{sql}
SELECT * FROM drivers_license
WHERE gender = 'female' AND hair_color = 'red' 
AND car_make = 'Tesla' AND car_model = 'Model S' 
AND height BETWEEN 65 AND 67 
\end{sql}
nous permet d'obtenir 3 numéros :

\begin{table}[!htp]
\centering
{\tt
\begin{tabular}{rrrrrrrrr}
id& age& height& eye\_color& hair\_color& gender& plate\_number& car\_make& car\_model \\ \hline
202298& 68& 66& green& red& female& 500123& Tesla& Model S \\
291182& 65& 66& blue& red& female& 08CM64& Tesla& Model S \\
918773& 48& 65& black& red& female& 917UU3& Tesla& Model S \\
\end{tabular}
}
\end{table}

Est-ce que l'une d'entre elles est riche ?
\begin{sql}
SELECT i.annual_income, p.id, p.name, d.age FROM income AS i
JOIN person AS p 
ON i.ssn = p.ssn
JOIN drivers_license AS d
ON p.license_id = d.id
WHERE d.id in (202298,291182,918773)
ORDER BY i.annual_income DESC
\end{sql}
Oui !

\begin{table}[!htp]
\centering
{\tt
\begin{tabular}{rrrr}
annual\_income	&id&	name&	age \\ \hline
310000	&99716&	Miranda Priestly	&68\\
278000	&78881	&Red Korb&	48\\
\end{tabular}
}
\end{table}

Mais qui est allé au concert de musique 3 fois au mois de décembre ?

\begin{sql}
SELECT person_id, COUNT(*) AS nbFois
FROM facebook_event_checkin
WHERE event_name = 'SQL Symphony Concert'
AND date between 20171201 and 20171231
GROUP BY person_id
HAVING nbFois = 3 
\end{sql}

\begin{table}[!htp]
\centering
{\tt
\begin{tabular}{rr}
person\_id& nbFois \\ \hline
24556& 3 \\
99716& 3 \\
\end{tabular}
}
\end{table}

Je crois que Miranda est coincée !
\begin{sql}
INSERT INTO solution VALUES (1, "Miranda Priestly");
SELECT value FROM solution;
\end{sql}
Et félicitations, vous avez trouvé le cerveau du meurtre ! Tout le monde à SQL City vous salue comme le plus grand détective SQL de tous les temps. Il est temps de sabrer le champagne !
%\end{landscape}






\underline{Remarque :}

Il est demandé de le faire en deux requêtes ce qui n'est pas très pédagogique, mais bien sûr c'est possible. On peut proposer cela\footnote{https://gist.github.com/bearloga/cfc8099223d1dace2604c8737dcbb4c3} :
\begin{sql}
WITH red_haired_tesla_drivers AS (
    SELECT id AS license_id
    FROM drivers_license
    WHERE gender = 'female' AND hair_color = 'red'
    AND car_make = 'Tesla' AND car_model = 'Model S'
    AND height BETWEEN 65 AND 67 
), rich_suspects AS (
    SELECT person.id AS person_id, name, annual_income
    FROM red_haired_tesla_drivers AS rhtd
    LEFT JOIN person ON rhtd.license_id = person.license_id
    LEFT JOIN income ON person.ssn = income.ssn
), symphony_attenders AS (
    SELECT person_id, COUNT(*) AS nbFois
    FROM facebook_event_checkin
    WHERE event_name = 'SQL Symphony Concert'
	AND date between 20171201 and 20171231 
    GROUP BY person_id
    HAVING nbFois = 3 
)
SELECT name, annual_income
FROM rich_suspects
JOIN symphony_attenders ON rich_suspects.person_id = symphony_attenders.person_id
\end{sql}




\section{Solution 2}

\begin{qexo}
\end{qexo}
La requête
\begin{sql}
SELECT count(*) FROM person;
\end{sql}
donne :
\begin{table}[!htp]
\centering
{\tt
\begin{tabular}{r}
count(*) \\ \hline
10011
\end{tabular}
}
\end{table}


\begin{qexo}
\end{qexo}
La requête
\begin{sql}
SELECT * FROM person LIMIT 10;
\end{sql}
donne
\begin{table}[!htp]
\centering
{\tt
\begin{tabular}{rrrrrr}
id &	name	&license\_id	&address\_number	&address\_street\_name	&ssn \\ \hline
 10000  &  Christoper Peteuil  &  993845  &  624  &  Bankhall Ave  &  747714076  \\
 10007  &  Kourtney Calderwood  &  861794  &  2791  &  Gustavus Blvd  &  477972044 \\
 10010  &  Muoi Cary  &  385336  &  741  &  Northwestern Dr  &  828638512 \\
 10016  &  Era Moselle  &  431897  &  1987  &  Wood Glade St  &  614621061 \\
 10025  &  Trena Hornby  &  550890  &  276  &  Daws Hill Way  &  223877684 \\
 10027  &  Antione Godbolt  &  439509  &  2431  &  Zelham Dr  &  491650087 \\
 10034  &  Kyra Buen  &  920494  &  1873  &  Sleigh Dr  &  332497972 \\
 10039  &  Francesco Agundez  &  278151  &  736  &  Buswell Dr  &  861079251 \\ 
 10095  &  Leslie Thate  &  729987  &  2772  &  Camellia Park Circle  &  127944356  \\
 10122  &  Alva Conkel  &  779002  &  116  &  Diversey Circle  &  148521773 \\
\end{tabular}
}
\end{table}


\begin{qexo}
\end{qexo}
La requête
\begin{sql}
SELECT DISTINCT type FROM crime_scene_report;
\end{sql}
donne
\begin{table}[!htp]
\centering
{\tt
\begin{tabular}{r}
type \\ \hline
robbery \\
murder\\
theft\\
fraud\\
arson\\
bribery\\
assault\\
smuggling\\
blackmail\\
\end{tabular}
}
\end{table}


\begin{qexo}
\end{qexo}
La requête
\begin{sql}
SELECT * FROM person WHERE name = 'Kinsey Erickson'
\end{sql}
donne
\begin{table}[!htp]
\centering
{\tt
\begin{tabular}{rrrrrr}
id&	name&	license\_id	&address\_number	&address\_street\_name	&ssn \\ \hline
89906&	Kinsey Erickson&	510019&	309&	Northwestern Dr	&635287661 \\
\end{tabular}
}
\end{table}

\begin{qexo}
\end{qexo}
La requête
\begin{sql}
SELECT * FROM crime_scene_report 
WHERE type = 'theft' 
AND city = 'Chicago';
\end{sql}
donne
\begin{table}[!htp]
\centering
{\tt
\begin{tabular}{rr>{\raggedleft\arraybackslash}p{10cm}r}
date&	type	&description	&city \\ \hline
20180115&	theft&	Big Bully stole my lunch money!&	Chicago \\
20170101&	theft&	‘Yes,’ said Alice, ‘we learned French and music.’ &	Chicago\\
20171227&	theft&	silence, and then another confusion of voices--‘Hold up his head--Brandy &	Chicago \\
\end{tabular}
}
\end{table}


\begin{qexo}
\end{qexo}
La requête
\begin{sql}
SELECT DISTINCT city 
FROM crime_scene_report 
WHERE city LIKE 'I%';
\end{sql}
donne
\begin{table}[!htp]
\centering
{\tt
\begin{tabular}{r}
city \\ \hline
Irving\\
Indianapolis\\
Irvine\\
Inglewood\\
Independence\\
\end{tabular}
}
\end{table}



\begin{qexo}
\end{qexo}
La requête
\begin{sql}
SELECT DISTINCT city 
FROM crime_scene_report 
WHERE city BETWEEN 'W%' AND 'Z%';
\end{sql}
donne
\begin{table}[!htp]
\centering
{\tt
\begin{tabular}{r}
city \\ \hline
Wilmington\\
Waterbury\\
West Valley City\\
Winter Haven\\
Youngstown\\
Wichita\\
West Covina\\
Yakima\\
Washington\\
Winston\\
Westminster\\
Waco\\
Yonkers\\
Warren\\
Worcester\\
Waterloo\\
York\\
\end{tabular}
}
\end{table}


\begin{qexo}
\end{qexo}
La requête
\begin{sql}
SELECT max(age) FROM drivers_license;
\end{sql}
donne
\begin{table}[!htp]
\centering
{\tt
\begin{tabular}{r}
max(age) \\ \hline
89
\end{tabular}
}
\end{table}



\begin{qexo}
\end{qexo}
La requête
\begin{sql}
SELECT * FROM drivers_license ORDER BY age ASC LIMIT 10
\end{sql}
donne
\begin{table}[!htp]
\centering
{\tt
\begin{tabular}{rrrrrrrrr}
 id  &  age  &  height  &  eye\_color  &  hair\_color  &  gender  &  plate\_number  &  car\_make  &  car\_model  \\ \hline
 101255  &  18  &  79  &  blue  &  grey  &  female  &  5162Z1  &  Lexus  &  GS  \\
 108374  &  18  &  63  &  brown  &  red  &  male  &  X2KE6N  &  Ford  &  Escape  \\
 112201  &  18  &  57  &  green  &  green  &  male  &  HW66XJ  &  BMW  &  325 \\
 115674  &  18  &  74  &  blue  &  blue  &  female  &  2OGQIP  &  Mitsubishi  &  Diamante \\ 
 122161  &  18  &  73  &  black  &  black  &  male  &  2H3Y1S  &  BMW  &  M5  \\
 127288  &  18  &  59  &  black  &  black  &  female  &  A20YSP  &  Ford  &  Freestar \\ 
 131246  &  18  &  57  &  brown  &  white  &  female  &  VOU6R8  &  Suzuki  &  Grand Vitara \\ 
 141220  &  18  &  70  &  green  &  blue  &  male  &  215AK2  &  Lexus  &  LX \\
 152848  &  18  &  58  &  brown  &  blonde  &  male  &  4Y00IK  &  Mazda  &  Millenia \\ 
 160151  &  18  &  67  &  blue  &  grey  &  female  &  O1G724  &  Mitsubishi  &  Montero \\
\end{tabular}
}
\end{table}


\begin{qexo}
\end{qexo}
La requête
\begin{sql}
SELECT person.name, income.annual_income 
FROM income 
JOIN person 
ON income.ssn = person.ssn 
WHERE annual_income > 450000
\end{sql}
donne
\begin{table}[!htp]
\centering
{\tt
\begin{tabular}{rr}
name	& annual\_income \\ \hline
Claudio Carlan	& 473100 \\
Felice Prudden	& 486600 \\
Buena Cosimini	& 475700 \\
Dianna Eyster	& 476300 \\
Numbers Cranker	& 498500 \\
Truman Haaker	& 489800 \\
\end{tabular}
}
\end{table}





\begin{qexo}
\end{qexo}
La requête
\begin{sql}
SELECT name, annual_income as income, 
gender, eye_color as eyes, hair_color as hair
FROM income i
JOIN person p
ON i.ssn = p.ssn 
JOIN drivers_license dl
ON p.license_id = dl.id
WHERE annual_income > 450000
\end{sql}
donne
\begin{table}[!htp]
\centering
{\tt
\begin{tabular}{rrrrr}
name & income & gender & eyes & hair \\ \hline
Claudio Carlan & 473100 & male & black & brown \\
Felice Prudden & 486600 & female & green & green \\
Buena Cosimini & 475700 & female & brown & blonde \\
Dianna Eyster & 476300 & female & brown & black \\
Numbers Cranker & 498500 & male & brown & green \\
Truman Haaker & 489800 & male & brown & grey\\
\end{tabular}
}
\end{table}














%%%%%%%%%%%%%%%%%%%%%%%%%%%%%%%%%%%%%%%%%%%%%%%%%%%%%%%%%%%%%%%%%%%%%%%%%%%%%%%
%%%%%%%%%%%%%%%%%%%%%%%%%%%%%%%%%%%%%%%%%%%%%%%%%%%%%%%%%%%%%%%%%%%%%%%%%%%%%%%

\end{document}