
Le fichier \texttt{gilberte.txt} contient un extrait du premier volume d'\emph{À l'ombre des jeunes filles 
en fleurs} de Marcel Proust, où il évoque la fin de sa relation avec Gilberte Swann, son 
premier amour.

Dans toute la suite, les questions porteront sur les lignes allant du numéro $\alpha$ inclus au 
numéro $\alpha+300$ exclu. Comme toujours en \python, la première ligne du texte porte le numéro 
0.

On pourra remarquer qu'aucun mot n'est coupé lors d'un retour à la ligne, ce qui simplifiera les 
choses.

\question\ Combien de fois apparaît le mot \og Gilberte \fg\ ?

\vspace{1cm}

\noindent On appellera ici \emph{paragraphe} toute partie du texte telle que :
\begin{itemize}
 \item le premier caractère est une tabulation ;
\item le caractère précédant cette tabulation est un retour à la  ligne, sauf s'il s'agit du début 
du texte ; 
\item le dernier caractère est un retour à la ligne ;
\item le caractère suivant ce retour à la ligne est une tabulation, sauf s'il s'agit de la fin du 
texte.
\end{itemize}

\question\ Combien y a-t-il de paragraphes ?



\noindent Proust est célèbre pour la longueur de ces phrases. Vous allez donc chercher la phrase la plus 
longue de la portion du texte que vous devez étudier.\\
On rappelle que l'on appelle \emph{blanc} un caractère qui est un espace, un retour chariot ou une 
tabulation.\\
On appellera \emph{point} les caractères \og.\fg, \og!\fg\ et \og?\fg. \\
On appellera ici \emph{phrase} toute portion du texte telle que :
\begin{itemize}
 \item le premier caractère est un blanc ;
 \item le caractère précédant ce blanc est un retour à la ligne ou un point, sauf s'il s'agit du 
début du texte ;
 \item le dernier caractère est un point ;
 \item le caractère suivant ce point est un blanc, sauf s'il s'agit de la fin du texte.
\end{itemize}

Attention, une phrase peut contenir un point autre que son dernier caractère, par exemple dans les 
phrases :
\begin{center}
\emph{Je m'écriai \og Gilberte !\fg\ en la voyant.}
\end{center}

Ou
\begin{center}
\emph{J'étais perplexe ...}
\end{center}

Un autre exemple pour être sûr que tout est compris : dans le texte :\\

\emph{
Bonjour. Ça va ?\\
Comment te portes-tu ?\\
\indent Très bien,\\ je te remercie.\\}

il y a quatre phrases, la phrase la plus longue est la dernière, et elle comporte 28 caractères (ne 
pas oublier les deux blancs au début (retour chariot et tabulation), le retour chariot au milieu et le point à la fin 
!).\\

Et une dernière remarque : votre portion de texte risque de commencer au milieu d'une phrase et de finir au milieu 
d'une autre : pour simplifier les choses, la portion de votre texte commençant au premier caractère de votre texte 
et finissant là où commence la phrase suivante ne sera pas prise en compte, même si par hasard c'était une phrase 
complète. Et si votre texte ne finit pas par un point suivi d'un blanc, vous ne tiendrez pas compte de la 
phrase incomplète qui clôt votre portion de texte.


\question\ Quel est le nombre de caractères de la phrase la plus longue ?

\medskip

\question\ Quel est le nombre de caractères de la phrase la plus courte ?



\noindent Le texte d'\emph{À la recherche du temps perdu} étant tombé dans le domaine public depuis 1987, 
vous décidez de publier vous-mêmes la partie du texte qui vous est échue aujourd'hui, dans une 
édition au format particulier : ayant hérité de votre oncle épicier d'une grande quantité de 
rouleaux pour caisse enregistreuse, vous décidez de faire imprimer le texte sur ce papier. Dans un 
souci de lisibilité, vous décidez également d'imprimer ce texte dans une police de taille standard 
: chaque ligne comportera au plus neuf caractères, sans compter les retours chariot.\\
La règle est la suivante : les retours à la ligne du texte de base sont tous supprimés, sauf ceux 
marquant un changement de paragraphe. Les retours à la ligne de la nouvelle édition seront donc 
ceux des changements de paragraphe et ceux imposés par la taille maximale de neuf caractères de 
chaque ligne. S'il le faut, les mots seront coupés par un retour à la ligne sans scrupule. Par 
contre, les espaces en début de ligne seront supprimés.\\
Par exemple, le texte :\\

\emph{Marcel Proust est un enfant à la santé fragile. Toute sa vie il a des difficultés 
respiratoires.\\
\indent Très jeune, il fréquente des salons aristocratiques.}\\

sera réédité de la manière suivante :\\

\emph{Marcel P\\
roust est\\
un enfant\\
à la sant\\
é fragile\\
. Toute s\\
a vie il \\
a des dif\\
ficultés \\
respirato\\
ires.\\
\indent Très jeu\\
ne, il fr\\
équente d\\
es salons\\
aristocra\\
tiques.}\\
 
\question\ Donner la 71-ème ligne (donc la ligne numéro 70) de votre texte réédité : on signalera les 
tabulations en début de ligne et les espaces par le symbole.