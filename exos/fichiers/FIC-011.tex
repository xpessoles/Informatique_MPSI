Ouvrir le fichier \texttt{complot$\_$contre$\_$lamerique.txt} dans python (on prendra soin de nommer la variable contenant cet objet). 

\question{} Que fait chacune des méthodes  \pyv{read()}, \pyv{readline()} et \pyv{readlines()} ? Quels sont les types des valeurs que chacune des ces fonctions renvoient ? 

\question{} Que représentent les symboles \texttt{\textbackslash t} et \texttt{\textbackslash n} ?


\question{} \'Ecrire une fonction \python\ \pyv{carac(nom_de_fichier)} qui renvoie un tableau contenant le nombre de caractères de chaque ligne du fichier \pyv{nom_de_fichier}, retour chariot exclu. 

\emph{Indication :} attention au type de \pyv{nom_de_fichier} !

\question{} \'Ecrire une fonction \python\ \pyv{compte_carac(carac,nom_de_fichier)} qui renvoie pour un caractère de l'alphabet son nombre d'occurence sans tenir compte de la casse.

\question{} \'Ecrire une fonction \python\ \pyv{stat_carac(nom_de_fichier)} qui renvoie une liste de 26 éléments donnant le nombre d'occurence de chaque lettre de l'alphabet.

\question{ } \'Ecrire une fonction \python\ \pyv{trace_stat_carac(carac,nom_de_fichier)} qui trace en fonction du numéro de la lettre dans l'alphabet le nombre d'occurence.
