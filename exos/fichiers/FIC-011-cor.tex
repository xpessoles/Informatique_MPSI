\question{} Que fait chacune des méthodes  \pyv{read()}, \pyv{readline()} et \pyv{readlines()} ? Quels sont les types des valeurs que chacune des ces fonctions renvoient ? 

La méthode read lit un caractère depuis la position courante, renvoie une chaine.

La méthode readline lit le reste de la ligne depuis la position courante, renvoie une chaine.

La méthode readlines lit toutes les lignes depuis la position courante, renvoie une liste de chaines.

\question{} Que représentent les symboles \texttt{\textbackslash t} et \texttt{\textbackslash n} ?

Ce sont les caractères \og tabulation\fg{} et \og nouvelle ligne\fg{} .


\question{} \'Ecrire une fonction \python\ \pyv{carac(nom_de_fichier)} qui renvoie un tableau contenant le nombre de caractères de chaque ligne du fichier \pyv{nom_de_fichier}, retour chariot exclu. 

\begin{pyverbatim}
def carac(nom_de_fichier):
    """Renvoie une liste contenant le nombre de caractères 
       de chaque ligne de nom_de_fichier"""
    with open(nom_de_fichier,'r',) as f:
        lignes = f.readlines()
    return [len(x.strip('\n')) for x in lignes]
\end{pyverbatim}
