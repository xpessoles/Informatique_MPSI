On suppose que l'on dispose d'un fichier \texttt{nombres.txt} contenant des entiers naturels séparés pas des blancs (on 
rappelle qu'un blanc est soit un espace, soit une tabulation \texttt{\textbackslash t} soit un retour à la ligne 
\texttt{\textbackslash n}).

Pour faciliter votre travail, un exemple de tel fichier est disponible sur le site de classe. Nous vous encourageons fortement à tester vos fonctions sur des exemples dont vous aurez calculé le résultat à la main.  

\medskip

\question\ Écrire une fonction \texttt{somme()}, sans argument, renvoyant la somme de tous les entiers contenus dans ce 
fichier.

\medskip

\question\ Pour chaque ligne du fichier, on effectue le produit des entiers de cette ligne. Écrire une fonction \texttt{moyenne()}, sans 
argument, renvoyant la moyenne de tous ces produits.

\medskip

\question\ Si $i$ est un entier inférieur au nombre de lignes du fichier, on note $s_i$ la somme des entiers de la ligne 
$i$. Écrire une fonction \texttt{inversion()}, sans argument, retournant le nombre de couples $(i, j)$ tels que $i < 
j$, et $s_j<s_i$. 

Les invariants de boucle seront impérativement précisés en commentaires, dans le script renvoyé.

\medskip

\question\label{qu:complexité} Dans cette dernière question, on suppose que le fichier contient $n$ lignes, et que chaque ligne 
ne contient qu'un entier. Quelle est en fonction de $n$ la complexité de l'algorithme \texttt{inversion()} ? 

On supposera que les opérations de lecture dans le fichier se font en temps constant. Ainsi, l'opération consistant à lire tous les 
entiers du fichier et à les stocker un par un dans un tableau sera supposée avoir une complexité en $O(n)$.

Pour plus de clarté, vous recopierez le code de la fonction \texttt{inversion()} avant d'étudier sa complexité. 
