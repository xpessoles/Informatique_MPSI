\question\ \'Ecrire une fonction \pyv{resume(nom_de_fichier)} qui prend en argument une chaîne de caractères \pyv{nom_de_fichier} et qui renvoie le triplet \pyv{(L,m,c)} où, pour le fichier dont le chemin est \pyv{nom_de_fichier} ;: 
    \begin{itemize}
      \item \pyv{L} est le nombre de lignes du fichier ;
      \item \pyv{m} est le nombre de mots (sous chaîne maximale non vide de caractères consécutifs, sans blanc)  du fichier  ;
      \item \pyv{c} est le nombre de caractères  du fichier (blancs compris).
    \end{itemize}
On rappelle qu'un blanc est un retour chariot (\texttt{\textbackslash{}n}), une tabulation (\texttt{\textbackslash{}t}) ou une espace.
